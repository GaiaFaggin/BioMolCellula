\chapter{Chimica e bioenergetica della cellula}
\section{Le componenti chimiche della cellula}
Gli organismi viventi sono composti da un piccolo sottoinsieme di elementi: carbonio (C), idroceno (H), azoto (N) e ossigeno (O) formano il $96.5\%$ del peso della cellula. Gli atomi di
questi elementi sono legati da legami covalenti in modo da formare molecole in quanto sono pi\`u resistenti delle energie termiche all'interno della cellula e sono rotti solo durante
specifiche reazioni con altri atomi e molecole. Due molecole diverse possono essere tenute insieme da legami non covalenti molto pi\`u deboli.
\subsection{L'acqua \`e mantenuta da legami a idroeno}
Le reazioni all'interno della cellula avvengono in ambiente acquoso, pertanto la vita si basa sulle propriet\`a chimiche dell'acqua. In ogni molecola d'acqua ($H_20$) i due atomi di 
H sono legati all'atomo O da due legami covalenti altamenti polari, pertantosi trova una distribuzione inequale di elettroni che causa una regione carica positivamente verso gli atomi H
e negativamente verso l'O. Quando una parte carica positivamente si avvicina a una negativa si formano legami a idrogeno, molto meno forti di quelli covalenti e facilmente rotti 
dall'energia termica delle molecole. Questi legami durano pertanto un periodo breve. Questi legami sono responsabili dello stato liquido dell'acqua, dell'alta tensione superficiale e 
punto di ebollizione. Alcune molecole come gli alcoli che possiedono legami polari possono formare legami a idrogeno con l'acqua si dissolvono facilmente in acqua e sono chiamate 
idrofile (zuccheri, DNA, RNA e la maggior parte delle proteine). Le molecole idrofobiche invece sono apolari e non formano legami a idrogeno e pertanto non si dissolvono nell'acqua, un
importante esempio sono gli idrocarburi, in cui gli H sono legati con gli atomi di C attraverso legami non polari, questa propriet\`a \`e sfruttata dalle cellule le cui membrane sono 
costruite da molecole con lunghe catene idrocarburiche.
\subsection{Quattro tipi di attrazione non covalente aiuta a unire le molecole nelle cellule}
Molto della biologia dipende dagli specifici legami causati da legami non covalenti: attrazione elettrostatica (legami ionici), legami a idrogeno e attrazioni di van der Waals e un 
quarto fattore che \`e la forza idrofobica. Nonostante ognuna di queste forze da sola sarebbe troppo debole per essere efficace si sommano tra loro in modo da creare una forte attrazione
tra due molecole separate. Si noti anche come formando un'interazione competitiva con queste molecole l'acqua riduca fortemente la forza delle attrazioni elettrostatiche e dei legami a 
idrogeno.
\subsection{Alcune molecole polari formano acidi e basi in acqua}
Una delle reazioni chimice pi\`u significative nella cellula occorre quando una molecola con un legame covalente altamente polare tra un idrogeno e un altro atomo si dissolve in acqua.
Tale idrogeno ha quasi completamente perso il proprio elettrone e pertanto esiste quasi come nucleo di idrogeno caricato positivamente ($H^+$). Quando la molecola polare viene circondata
da molecole d'acqua il protone viene attratto dalla loro carica parzialmente negativa e si pu\`o dissociare dalla molecola originale formando uno ione idronio ($H_3O^+$). La reazione
inversa accade molto velocemente, pertanto in soluzione acquosa i protoni continuano a spostarsi tra una molecola e l'altra. Le sostanza che compiono questa reazione sono dette acidi e 
maggiore la concentrazione di $H_3O^+$, pi\`u acida la soluzione. Questo ione risulta presente anche in acqua pura a causa del continuo movimento di protoni, in una concentrazione 
$10^{-7} M$. Per convenzione la concentrazione di $H_3O^+$ \`e riferita come la concentrazione di $H^+$ e espressa utilizzando la scala del pH, logaritmica. L'acqua pura ha un valore di 
$7$ ed \`e detta neutra. Per valori di pH maggiori di $7$ \`e detta basica, per valori minori detta acida. Gli acidi si caratterizzano in forti o deboli in base a quanto facilmente 
perdono i protoni in acqua. Molti degli acidi importanti per la cellula sono deboli. A causa dell'effetto sulla natura delle molecole dei protoni liberi l'acidit\`a all'interno della
cellula deve essere regolata. L'opposto di un acido \`e una base, molecole che accettano un protone in soluzione acquosa, ancora una volta nelle cellula sono presenti per la maggior
parte basi deboli. Acidi e basi hanno azioni contrastanti e tendono ad annullare reciprocamente il loro effetto, pertanto l'interno della cellula \`e mantenuto vicino alla neutralit\`a 
da buffer, acidi e basi deboli che tendono a compiere scambi di protoni ad un pH vicino a $7$, mantenendo l'ambiente della cellula costante. 
\subsection{La cellula \`e formata da composti di carbonio}
Senza considerare l'acqua e gli ioni inorganici come il potassio, la maggior parte delle molecole nella cellula sono basate sul carbonio, un atomo con la capacit\`a di formare grosse
molecole. Siccome il cabonio \`e piccolo e ha possiede quattro elettroni liberi nel livello esterno pu\`o formare legami covalenti con altri atomi e con s\`e stesso, in modo da formare
catene ed anelli in modo da generare molecole grandi e complesse. I composti del carbonio sono detti molecole organiche. Alcune combinazioni di atomi compaioni ricorrentemente nella
cellula e ognuno di questi gruppi possiede propriet\`a chimiche e fisiche proprie che influiscono sul loro comportamento. 
\subsection{Le cellule contengono quattro principali famiglie di piccole molecole organiche}
Le molecole della cellula sono basate sul carbonio e hanno pesi molecolari tra $100$ e $1000$ e contengono circa $30$ atomi di carbonio. Sono solitamente trovate libere in soluzione.
Alcune sono usate come monomeri per costruire macromolecole polimeriche, altre agiscono come fonti di energia e sono divise e trasformate in altre piccole molecole con pi\`u ruoli nella
cellula. Sono molto meno presenti rispetto alle macromolecole. Tutte le molecole organiche sono sintetizzate e divise dallo stesso insieme di componenti, pertanto i composti nella 
cellula sono chimicamente simili e possono essere classificati per la maggior parte in zuccheri, acidi grassi, nucleotidi e amminoacidi. 
\subsection{La chimica della cellula \`e dominata da macromolecole con propriet\`a notevoli}
Le macromolecole sono le molecole organiche pi\`u abbondanti per peso nella cellula. Sono le strutture principali per la costruzione e la cellula e definiscono le propriet\`a degli 
organismi viventi. Le macromolecole nella cellula sono polimeri che sono costruiti legando covalentemente piccole molecole organiche (monomeri) in lunghe catene. Le proteine sono 
abbondanti e versatili, alcune servono da enzimi, che catalizzano tutte le reazioni interne alla cellula, altre hanno funzione strutturale, o per compattare il DNA nei cromosomi, altre
ancora agiscono come produttrici di forza motile. Nonostante le reazioni chimiche per la formazione di polimeri varino tra proteine, acidi nucleici e polisaccaridi in tutte la 
molecola cresce grazie all'addizione di un monomero in una fine di una catena in una reazione di condensazione, in cui una molecola di acqua \`e persa con ogni subunit\`a aggiunta. 
Questa operazione richiede gli stessi enzimi per tutta la molecola ed \`e pertanto facilmente serializzabile. Tranne i polisaccaridi i monomeri che formano le macromolecole esistono
in diverse varianti e richiedono pertanto una sequenza precisa di addizione per formare la macromolecola corretta. 
\subsection{I legami non covalenti specificano la forma di una macromolecola e i suoi legami con le atre molecole}
La maggior parte dei legami covalenti nella macromolecola permettono una rotazione degli atomi che legano, permettendo grande flessibilit\`a garantendo ad essa un gran numero di 
conformazioni possibili quando l'energia termica causa rotazioni. Nonstante questo la maggior parte delle macromolecole biologiche sono altamente costrette a causa di un gran numero di
legami non covalenti che si formano tra diverse parti della stessa molecola e causano la macromolecola in una conformazione particolare, determinata dalla sequenza lineare dei monomeri.
Questo accade nella maggior parte delle proteine e in molte delle piccole molecole di RNA. Questi legami non covalenti possono anche creare forte attrazione tra moleocle diversi in modo
da creare un'interazione molecolare con alta specificit\`a e con vari gradi di affinit\`a, permettendo rapida dissociazione dove necessario. Questo processo \`e fondamentale per tutte le
catalisi biologiche, permettendo il loro funzionamento come enzimi e per la creazione di strutture cellulari complesse. 
\section{Catalisi e utilizzo dell'energia da parte delle cellule}
Una delle principali differenze tra organismi viventi e non viventi \`e che i primi creano e mantengono ordine. Per farlo necessitano di operare un insieme di reazioni chimiche in cui 
alcune molecole sono separate per mettere a disposizione altre molecole per costruirne altre.
\subsection{Il metabolismo della cellula \`e organizzato dagli enzimi}
Le reazioni chimiche che una cellula performa accadrebbero normalmente unicamente ad alte temperature, pertanto ogni reazione richiede un'accellerazione specifica nella sua reattivit\`a.
Questo fatto permette alla cellula di controllare la sua chimica. Il controllo \`e operato da catalizzatori biologici specializzati, proteine detti enzimi o RNA detto ribosomi. Ogni
enzima catalizza una delle possibili reazioni che sono connesse in serie in modo che il prodotto di una sia il substrato di un'altra. Questi cammini lineari sono legati uno con l'altro,
formando un insieme di reazioni interconnesse che permettono alla cellula di sopravvivere, crescere e riprodursi. Esistono due principali flussi di reazioni chimiche: quelle cataboliche
che separano i nutrienti in miuccole molecole, generando energia e alcune  piccole molecole fondamentali e anaboliche o biosintetiche in cui le piccole molecole e l'energia vengono
utilizzate per guidare la sintesi delle molecole che formano la cellula. Insieme costituiscono il metabolismo della cellula. 
\subsection{L'ordine biologico \`e reso possibile dal rilascio di calore dalla cellula}
Le cellule devono ridurre il proprio livello di entropia e pertanto deve recuperare energia dall'ambiente sotto forma di cibo o fotoni, che viene poi tulizzata per generare l'ordine
necessario. Nel corso di queste reazioni una parte dell'energia utilizzata viene trasformata in calore. L'energia, nel caso delle cellule animali viene ottenuta rompendo i legami dei 
nutrienti e viene trasformata in energia termica. La cellula non pu\`o beneficiare del calore rilasciato a meno che queste reazioni che generano energia siano accoppiati direttamente
con i progessi che generano l'ordine molecolare. 
\subsection{Le cellule ottengono energia ossidando molecole organiche}
Tutte le cellule animali e vegetali utilizzano l'energia conservata in legami chimici di molecole organiche, sia che siano zuccheri sintetizzati dalla fotosintesi sia che siano ottenuti
mangiando. L'energia \`e stratta da un processo di ossidazione graduale. L'atmosfera contiene molto ossigeno e in presenza di ossigeno la forma di carbonio pi\`u stabile \`e la $CO_2$ e
quella dell'idrogeno $H_2O$. Una cella \`e pertanto capace di ottenere energia permettendo a carbonio e idrogeno delle molecole di combinarsi con l'ossigeno per produrre $CO_2$ e $H_2O$.
Questo processo \`e chiamato respirazione aerobica. La fofosintesi e la respirazione sono processi complementari. Si nota pertanto come l'utilizzo di carbonio formi un grande ciclo che
coinvolge l'intera biosfera. 
\subsection{Ossidazione e riduzione coinvolgono il trasferimento di elettroni}
L'ossidazione nella cellula avviene attraverso l'uso di enzimi in cui il metabolismo prende le colecole attraverso un numero di reazioni che raramente coinvolgono la diretta addizione 
di ossigeno. L'ossidazione si riferisce a quel processo in cui elettroni sono trasferiti da un atomo all'altro (il processo inverso \`e la riduzione). Essendo che il numero di elettroni
deve essere conservato durante una reazione ossidazione e riduzione accadono contemporaneamente: una molecola guadagna un elettrone e un'altra lo perde. Questi termini si riferiscono
anche a un parziale spostamento di elettroni in un legame covalente: quando se ne crea uno polare l'atomo dalla parte del delta positivo acquisice una parziale carica positiva ed \`e
detto ossidato. Quando una molecola recupera un elettrone recupera anche un protone allo stesso momento e l'effetto netto \`e l'addizione di un atomo di idrogeno alla molecola:
$$ A+e^-+H^+\rightarrow AH$$
Queste reazioni, dette di idrogenazione sono riduzione, mentre quelle inverse, di deidroginazione sono dette ossidazioni. In una molecola organica avviene un'ossidazione quando il numero
di legami C-H diminuisce, una riduzione quando aumenta. Le cellule utilizzano gli enzimi per catalizzare le ossidazioni attraverso una sequenza di reazioni che permettono il raccolto
dell'energia prodotta. 
\subsection{Gli enzimi abbassano la barriera di energia di attivazione che blocca la reazione chimica}
Si noti come le reazioni chimiche procedono spontaneamente unicamente nella direzione che porta alla perdita di energia libera (energicamente favorevoli). Essendo che le molecole negli
esseri viventi si trovano in uno stato energetico relativamente stabile \`e necessario, affinch\`e una reazione inizi di un'energia di attivazione, creato da collisioni randomiche 
insolitamente energetiche, che diventano violente maggiore \`e l'energia. La chimica di una cellula \`e altamente controllata e il superamento del livello di energia \`e svolto da 
enzimi che si legano con un'altra molecola (substrato) in modo da ridurre l'energia di attivazione necessaria per la reazione. Questi enzimi sono detti catalizzatori e aumentano il 
tasso delle reaizoni chimice in quanto permettono maggiori collisioni randomiche con le molecole circostanti. 
\subsection{Gli enzimi possono guidare i substrati lungo specifici cammini di reazioni}
Un enzima non pu\`o cambiare il punto di equilibrio per una reazione in quanto aumenta anche il tasso della reazione inversa. Nonostante questo sono capaci di guidare le reazioni verso
un cammino specifico in quanto sono altamente selettivi e molto precisi, catalizzando un'unica reazione, pertanto ogni enzima selettivamente abbassa l'energia di attivazione di una delle
reazioni chimiche possibili  che il substrato pu\`o svolgere. In questo modo insiemi di enzimi possono direzionare ognuna delle molecole lungo cammini specifici. Ogni enzima possiede un
sito attivo, uno spazio in cui solo particolari substrati possono legarsi e dopo la reazione rimangono invariati e possono pertanto funzionare pi\`u volte. 
\subsection{Come gli enzimi trovano i loro substrati, la rapidit\`a dei movimenti molecolari}
Un enzima pu\`o catalizzare la reazione di migliaia di substrati al secondo. L'attacco veloce \`e possibile perch\`e il movimento causato dal calore \`e estremamente veloce. Questi 
movimenti molecolari sono calssificati in:
\begin{itemize}
	\item Movimento traslatorio: il movimento della molecola da un posto all'altro.
	\item Vibrazioni: il movimento di atomi legati da legami covalenti tra di loro.
	\item Rotazioni.
\end{itemize}
Questi movimenti aiutano ad unire le superfici di molecole che interagiscono. Il movimento delle molecole causa il processo di diffusione e in questo modo ogni molecola collide con un
grano numero di altre molecole al scondo. La distanza netta alla fine di una passeggiata casuale \`e proporzionale alla radice del tempo impiegato. Essendo che gli enzimi si muovono 
pi\`u lentamente dei substrati si possono considerare fermi. Il tasso di incontro dell'enzima con il suo substrato dipende alla concentrazione dell'ultimo. Una collisione del substrato 
con il sito attivo causa immediatamente la creazione del sistema enzima-substrato. L'alta specificit\`a \`e data dal fatto che una forma errata causa legami covalenti pi\`u deboli 
dell'agitazione termica. 
\subsection{Il cambio di energia libera di una reazione determina se pu\`o avvenire spontaneamente}
Nonostante gli enzimi velocizzino le reazioni non possono forzare reazioni energeticamente sfavorevoli. Questo tipo \`e per\`o necessario per alcune operazioni della cellula, pertanto
gli enzimi accoppiano reazioni energeticamente favorevoli in modo da produrre energia che viene utilizzata per le reazioni sfavorevoli e produrre ordine biologico. Il livello di energia
libera (G) esprime l'energia disponibile per fare lavoro e viene presa in considerazione quando il sistema subisce un cambiamento ($\Delta G$), critico in quanto \`e una diretta misura
della quantit\`a di disordine creata nell'universo da una reazione. Reazioni energeticamente favorevoli hanno un $\Delta G$ negativo, quelle favorevoli positivo e possono avvenire 
unicamente se accoppiate con reazioni favorevoli tale che il $\Delta G$ totale rimanga negativo. La concentrazione dei reagenti influenza il cambio di energia libera e la direzione di 
una reazione. A causa di questo per comparare le reazioni si deve utilizzare il cambio di energia libera standard o $\Delta G^{\circ}$, definito alla concentrazione per reagenti di
$1\frac{M}{L}$. Per una reazione $Y\rightarrow X$ a $37^\circ C$, $\Delta G^\circ$ \`e in relazione a $\Delta G$ come:
$$\Delta G = \Delta G^\circ + RT\ln\dfrac{[X]}{[Y]}$$
Dove $R$ \`e la costante dei gas reali e $R$ la temperatura assoluta. Si noti come al procedere della reazione il rapporto tra i reagenti cambia e avvicina $\Delta G$ a zero, dove si 
raggiunge l'equilibrio chimico e non esiste un cambio di energia libera per guidare la reazione in nessuna direzione e pertanto il rapporto di prodotto e substrato raggiunge un valore
costante $K$ detta costante di equilibrio. I $\Delta G$ delle reazioni accoppiate sono additivi










\begin{Huge}
	TO DO
\end{Huge}


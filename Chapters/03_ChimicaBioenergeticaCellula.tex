\chapter{Chimica e bioenergetica della cellula}
\section{Le componenti chimiche della cellula}
Gli organismi viventi sono composti da un piccolo sottoinsieme di elementi: carbonio (C), idroceno (H), azoto (N) e ossigeno (O) formano il $96.5\%$ del peso della cellula. Gli atomi di
questi elementi sono legati da legami covalenti in modo da formare molecole in quanto sono pi\`u resistenti delle energie termiche all'interno della cellula e sono rotti solo durante
specifiche reazioni con altri atomi e molecole. Due molecole diverse possono essere tenute insieme da legami non covalenti molto pi\`u deboli.
\subsection{L'acqua \`e mantenuta da legami a idroeno}
Le reazioni all'interno della cellula avvengono in ambiente acquoso, pertanto la vita si basa sulle propriet\`a chimiche dell'acqua. In ogni molecola d'acqua ($H_20$) i due atomi di 
H sono legati all'atomo O da due legami covalenti altamenti polari, pertantosi trova una distribuzione inequale di elettroni che causa una regione carica positivamente verso gli atomi H
e negativamente verso l'O. Quando una parte carica positivamente si avvicina a una negativa si formano legami a idrogeno, molto meno forti di quelli covalenti e facilmente rotti 
dall'energia termica delle molecole. Questi legami durano pertanto un periodo breve. Questi legami sono responsabili dello stato liquido dell'acqua, dell'alta tensione superficiale e 
punto di ebollizione. Alcune molecole come gli alcoli che possiedono legami polari possono formare legami a idrogeno con l'acqua si dissolvono facilmente in acqua e sono chiamate 
idrofile (zuccheri, DNA, RNA e la maggior parte delle proteine). Le molecole idrofobiche invece sono apolari e non formano legami a idrogeno e pertanto non si dissolvono nell'acqua, un
importante esempio sono gli idrocarburi, in cui gli H sono legati con gli atomi di C attraverso legami non polari, questa propriet\`a \`e sfruttata dalle cellule le cui membrane sono 
costruite da molecole con lunghe catene idrocarburiche.
\subsection{Quattro tipi di attrazione non covalente aiuta a unire le molecole nelle cellule}
Molto della biologia dipende dagli specifici legami causati da legami non covalenti: attrazione elettrostatica (legami ionici), legami a idrogeno e attrazioni di van der Waals e un quarto
fattore che \`e la forza idrofobica. Nonostante ognuna di queste forze da sola sarebbe troppo debole per essere efficace si sommano tra loro in modo da creare una forte attrazione tra due
molecole separate. Si noti anche come formando un'interazione competitiva con queste molecole l'acqua riduca fortemente la forza delle attrazioni elettrostatiche e dei legami a idrogeno.
\subsection{Alcune molecole polari formano acidi e basi in acqua}

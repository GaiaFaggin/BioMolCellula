\chapter{DNA, cormosomi e genomi}
La vida dipende dall'abilit\`a della cellula di conservare, recuperare e tradurre le istruzioni genetiche richieste per creare e mantenere un organismo vivente. Queste informazioni
ereditarie sono passate da una cellula a una sua figlia durante la divisione cellulare e da una generazione di organismi all'altra attraverso le cellule riproduttive, sono salvate come
geni. Le informazioni genetiche consistono principalmente su istruzioni per la costruzione delle proteine, macromolecole versatili che svolgono la maggior parte delle funzioni della 
cellula. Le informazioni genetiche sono trasportate su cromosomi, strutture a filo nel nucleo che diventano visibili durante la divisione e composti di DNA (acido desossiribonucleico) e 
proteine in egual misura. La determinazione della struttura a doppia elica del DNA ha solto il problema di come le informazioni nel DNA sono replicate e come una molecola di DNA utilizza
la sequenza dei suoi monomeri per produrre le proteine. 
\section{La struttura e la funzione del DNA}

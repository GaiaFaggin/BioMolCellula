\chapter{DNA, cormosomi e genomi}
La vita dipende dall'abilit\`a della cellula di conservare, recuperare e tradurre le istruzioni genetiche richieste per creare e mantenere un organismo vivente. Queste informazioni
ereditarie sono passate da una cellula a una sua figlia durante la divisione cellulare e da una generazione di organismi all'altra attraverso le cellule riproduttive, sono salvate come
geni. Le informazioni genetiche consistono principalmente su istruzioni per la costruzione delle proteine, macromolecole versatili che svolgono la maggior parte delle funzioni della 
cellula. Le informazioni genetiche sono trasportate su cromosomi, strutture a filo nel nucleo che diventano visibili durante la divisione e composti di DNA (acido desossiribonucleico) e 
proteine in egual misura. La determinazione della struttura a doppia elica del DNA ha risolto il problema di come le informazioni nel DNA sono replicate e come una molecola di DNA 
utilizza la sequenza dei suoi monomeri per produrre le proteine. 
\section{La struttura e la funzione del DNA}
Negli anni $50$ la struttura del DNA \`e stata determinata grazie a diffrazione a raggi X, che indicarono la composizione a due fili avvolti in un'elica fornendo un grande indizio a
Watson-Crick e al loro modello.
\subsection{Una molecola di DNA consiste di due catene di nucleotidi complementari}
Una molecola di acido desossiribonucleico cosiste di due catene polinucleotide lunghe note come strand e composte da quattro tipi di subunit\`a. Le catene sono antiparallele tra di loro
e tra le basi dei nucleotidi si formano legami a idrogeno che le tengono unite. I nucleotidi sono composti da zuccheri a 5 atomi di carbonio a cui sono attaccadi dei gruppi fosfati e 
una base contenente dell'azoto. Nel caso dei nucleotidi del DNA lo zucchero \`e il desossiribosio attaccato a un singolo gruppo fosfato (da cui il nome) e la base pu\`o essere adenina
(A), citosina (C), guanina (G) o timina (T). I nucleotidi sno legati covalentemente in unca catena attraverso gli zuccheri e i fosfati che formano un backbone di legami 
zucchero-fosfato-zucchero-fosfato alternati. I legami dei nucleotidi danno al filo del DNA una polarit\`a chimica. Il gruppo $5'$ fosfato si lega con il gruppo $3'$ idrossile di un
altro monomero e pertanto tutte le subunit\`a hanno lo stesso orientamento. Inoltre le due terminazioni dello strand sono facilmente distinguibili e ci si riferisce ad esse come alla
terminazione $3'$ e $5'$, indicando la polarit\`a. Grazie alla direzionalit\`a e linearit\`a del filo di DNA pu\`o essere letto facilmente. La doppia elica si genera dalla struttura 
chimica delle catene polinucleotidiche in quanto sono mantenute insieme da legami a idrogeno tra le basi sui fili diversi tutte le basi si trovano all'interno della doppia elica con i 
backbones verso l'esterno. In ogni caso una base a due anelli (una purina) \`e sempre legata con una ad anello singolo (una pirimidina): A si accoppia con T e G con C. Questo 
accoppiamento complementare delle basi permette a coppie di basi di essere ammassate nell'ordinamento pi\`u energeticamente favorevole all'interno della doppia elica in quanto ogni
coppia possiede la stessa lunghezza mantenendo i backbones a distanza costante. Per massimizzare l'efficienza spaziale i backbones si avvolgono formando una doppia elica destra con
un giro completo ogni $10$ paia. I membri di ogni base possono adattarsi insieme solo se i fili sono antiparalleli. Ogni strand della molecola contiene una sequenza complementare a
quella dell'altro strand. 
\subsection{La struttura del DNA mette a disposizione un meccanismo per l'ereditariet\`a}
La struttura a polimero lineare composto da quattro monomeri permette al DNA di trasportare informazioni in forma chimica, mentre la natura a doppio filamento permette la sua 
duplicazione. Essendo ogni filamento complementare all'altro pu\`o agire come stampo per la sintesi di un nuovo filamento complementare. Nominando i filamenti $S$ e $S'$, $S$ agisce
da stampo per la formazione di $S'$ e $S'$ per $S$, permettendo l'accurata copia dell'informazione nel DNA. Questa capacit\`a permette alla cellula di replicare il proprio genoma
e passarlo ai discendenti. Il DNA permette la codifica di proteine (le cui funzioni sono determinate dalla forma e pertanto in ultimo dalla sequenza di amminoacidi) attraverso una
corrispondenza esatta tra i quattro nucleotidi e i venti amminoacidi. L'intero gruppo di informazioni mantenute dal DNA di un organismo \`e detto genoma e specifica tutte le molecole
di RNA e proteine che l'organismo sintetizzer\`a. 
\subsection{Negli eucarioti il DNA \`e racchiuso nel nucleo cellulare}
Circa tutti il DNA della cellula si trova isolato nel nucleo, che occupa circa il $10\%$ del volume totale, delimitato da un involucro nucleare formato da due bistrati lipidi 
concentrici. Queste membrane sono puntuate a intervalli da pori nucleari in cui le molecole si trasferiscono dal nucleo al citosol. Tale involucro \`e conneso a un sistema di 
membrane intracellulari detto reticolo endoplasmatico che si estende nel citoplasma ed \`e meccanicamente supportato da una rete di filamenti intermedi detti lamine nucleari. L'involucro
permette alle proteine che agiscono sul DNA di essere concentrate dove necessario e mantiene gli enzimi nucleari e citosolici separati. 
\section{DNA cromosomico e il suo confezionamento nella fibra cromatina}
La funzione pi\`u importante del DNA \`e quella di trasportare i geni, informazioni che specificano tutte le molecole di RNA e le proteine che compongono l'organismo. Il DNA nucleare 
degli eucarioti \`e diviso in cromosomi. 
\subsection{Il DNA eucariotico \`e confezionato in un insieme di cromosomi}
Ogni cromosoma consiste di una singola molecola lineare di DNA estremamente lunga associata con proteine che piegano e impacchettano il filamento in una struttura compatta. Oltre a 
queste proteine si trovano altre proteine e molecole di RNA che si occupano dell'espressione genica e della riparazione e replicazione del DNA. Il complesso formato dal DNA e le proteine
strettamente legate prende il nome di cromatina. Nei batteri d'altra parte il DNA si trova in forma circolare. Con l'eccezione dei gameti, alcune cellule specializzate che non possono
moltiplicare o a cui manca il DNA e altre che replicano il loro DNA senza completare la divisione cellulare ogni nucleo cellulare umano contiene due coppie di ogni cromosoma, uno 
ereditato dal padre e uno dalla madre. I cromosomi di tale coppia sono detti omologhi, l'unica coppia non omologa nei maschi \`e quella dei cromoosomi Y (dal padre) e X (dalla madre).
Ogni uomo contienee 46 cromosomi: 22 paia comuni tra i sessi e una coppia detta dei cromosomi sessuali che possono essere distinti attraverso colorazione basata sull'ibridazione del DNA
in cui un piccolo filamento marcato con un colore fluorescente agisce da sonda che si lega alla sequenza complementare illuminando il cromosoma obiettivo. L'insieme dei 46 cromosomi
durante la mitosi \`e detto cariotipo.
\subsection{I cromosomi contengono lunghe stringhe di geni}
I cromosomi trasportano geni, definiti come segmenti di DNA che contengono le istruzioni per sintetizzare una particolare proteina (o una famiglia) o molecole di RNA significativamente 
funzionali. Oltre ai geni il genoma di organismi multicellulari e di molti altri eucarioti contiene grandi segmenti sparsi la cui funzione non \`e compresa. Alcuni di questi segmenti
sono fondamentali per la regolazione dell'espressione genica. Questi segmenti generano grandi diversit\`a nella dimensione genomica quando si comparano le specie e creano sequenze molto
diverse per oganismi strettamente imparentati, anche se contengono gli stessi geni. La divisione del genoma in cromosomi \`e unico per la specie, con numero e dimensione dei cromosomi
diversa. 
\subsection{La sequenza nucleotidica del genoma umano mostra come i geni sono ordinati}
Si nota come solo una piccola percentuale del genoma umano codifica le proteine e circa met\`a \`e composta da pezzi mobili che si sono inseriti nei cromosomi durante l'evoluzione. La
dimensione media di un gene \`e circa $27000$ coppie di basi. Siccome solo circa $1300$ paia sono richieste per codificare una proteina di dimensione media ($430$ amminoacidi negli 
umani) il resto della sequenza consiste di lunghezze di DNA non codificante che interrompe quello codificante. Le sequenze codificanti sono dette esoni, mentre quelle non codificanti
introni. Il DNA umano consiste di una lunga stringa di esoni e introni (che sono la maggioranza) che si alternano. Oltre a introni e esoni ogni gene \`e associato con sequenze di DNA
regolatorio responsabile che il gene sia espresso al momento e a livello giusto. Negli esseri umani occupano decine di migliaia di coppie di basi. Oltre ai geni codificanti le proteine
ne esistono altri che codificano molecole di RNA con funzione propria.
\subsection{Ogni molecola di DNA che forma un cromosoma lineare deve contenere un centromero, due telomeri e un'origine di replicazione}
Per formare un cromosoma funzionale una molecola di DNA deve poter essere abile di replicarsi e i replicati devono essere separati e partizionati in cellule figlie. Questo processo 
accade attraverso una serie ordinata di passaggi detti ciclo cellulare che fornisce la separazione temporale tra la duplicazione dei cromosomoi e la segregazione in due cellule figlie. 
Durante una lunga interfase i geni sono espressi e i cromosomi sono replicati rimanendo in un paio di cromatidi sorelle. In questo momento i cromosomi sono estesi in modo che la maggior
parte della cromatina esista come lunghi fili e i cromosomi individuali nonsiano distinguibili. In un momento successivo si condensano in modo da separare le cromatidi sorelle e 
distribuiti alle cellule figlie. I cromosomi altamente condensati sono dette cromosomi mitotici. Ogni cromosoma opera come un'unit\`a strutturale indipendente. Le operazioni di 
replicazione sono controllate da tre tipi di sequenze nucleotidiche nel DNA, ognuna delle quali si lega a proteine specifiche che guidano la macchina che replica e segrega i cromosomi.
Un tipo di sequenza nucleotidica agisce come origine della replicazione del DNA, ovvero il luogo dove la duplicazione inizia, ne esistono diverse per assicurarsi che l'intero cromosoma
venga replicato rapidamete. Dopo la replicazione del DNA le due cromatidi sorelle rimnangono attaccate a una fine mentre vengono condensate producendo cromosomi mitotici. Un'ulteriore
sequenza specializzata detta centromero permette a una copia di ogni cromosoma duplicato e condensato di essere portata nella cellula figlia quando si divide. Un complesso proteico detto
cinetocoro si forma al cnetromero e attacca i cromosomi al mandrino mitotico, permettendo la loro separazione. La terza sequenza di DNA specializzata forma i telomeri, le terminazioni 
dei cromosomi. Permettono una replicazione efficiente e formano strutture che proteggono la fine del cromosoma da essere confusa con una molecola di DNA in necessit\`a di riparazione.
\subsection{Le molecole di DNA sono altamente conservate nei cromosomi}
Tutti gli organismi eucariotici possiedono metodi per impacchettare il DNA in cromosomi. Questa compressione viene svolta da proteine che avvolgono e piegano il DNA in livelli sempre 
pi\`u alti di organizzazione. Nonostante molto meno condensati rispetto a cromosomi mitotici sono comunque compressi. La struttura dei cromosomi \`e dinamica: regioni specifiche del
interfase dei cromosomi decondensa per permettere l'accesso di sequenze specifiche del DNA per l'espressione genica, riparazione e replicazione ricondensandosi dopo. La condansazione 
avviene pertanto in modo da permettere accesso rapido e localizzato al DNA su richiesta.
\subsection{I nucleosomi sono unit\`a base della struttura del cromosoma eucariote}
Le proteine che si legano al DNA per formare i cromosomi sono divise in istoni e proteine non-istoniche che contribuiscono con la stessa massa ad un cromosoma. Il complesso di entrambe 
le classi di proteine \`e noto come cromatina. Gli istoni sono responsabili per il primo e basico livello di condensazine cromosomica detto nucleosoma, un complesso DNA-proteina. Si nota
come il nucleosoma sia formato da una stringa di DNA con parti di esso avvolte intorno a istone. Ogni nucleo istonico del nucleosoma consiste di un complesso di $8$ proteine istoniche
e DNA a doppio filamento lungo $147$ coppie di basi. Questo ottamero forma un nucleo proteico lungo il quale si lega. La regione di DNA linker che separa ogni nucleo di nucleosoma
pu\`o essere lunga fino a $80$ basi. In media i nucleosomi si ripetono ogni circa $200$ coppie di basi. 
\subsection{La struttura del nucleo del nucleosoma rivela come il DNA \`e condensato}
La forma a disco del nucleo istonico in cui il DNA \`e avvolto in una bobina sinistra di $1.7$ giri \`e composta da quattro istoni, proteine che condividono una piega istnica formata da
tre $\alpha$-eliche connesse da due anelli. Durante l'assemblamento del nucleosoma le pieghe istoniche si legano prima tra di loro formando dimeri, poi tetrameri e infine ottameri. 
L'interfaccia tra il DNA e l'istone forma $142$ legami a idrogeno in ogni nucleosoma. Circa met\`a di questi legami derivano dal backbone amminoacido e quello del DNA, le interazioni
idrofobiche e saline mantengono il DNA e la proteina insieme nel nucleosoma. Un quinto del nucleo istonico \`e formato da lisina o da arginina (amminoacidi basici) e le loro cariche 
positive possono neutralizzare il backbone del DNA. Queste interazioni spiegano perch\`e il legame con l'istone \`e indipendente dalla sequenza delle basi. Oltre alla piega istonica
ogni nucleo istonico possiede una coda \ce{N}-terminale che si estende all'esterno del nucelo DNA-istone che subisce cambi covalenti che controllano aspetti della struttura cromatina
e della funzione. Gli istoni sono proteine altamente conservate. 
\subsection{I nucleosomi hanno strutture dinamiche e sono soggetti a cambi catalizzati da complessi rimodellanti a cromatina ATP-dipendenti}
Il DNA in un nucleosoma isolato si disavvolge quattro volte al secondo, rimanendo esposto per intervalli di tempo, rendendolo disponibile per il legame con altre proteine. Dalla
cromatina nella cellula \`e richiesto un ulteriore allentamento \`e necessario in quanto le cellule contengono una variet\`a di complessi di rimodellamento a cromatina ATP-dipendenti.
Questi complessi includono una subunit\`a che idrolizza ATP che lega il nucleo proteico del nucleosoma e il DNA legato a esso. Utilizzando l'energia fornita dall'ATP muove il DNA
cambiando la struttura del nucleosoma temporaneamente allentando il legame tra il DNA e il nucleo istonico. Attraverso ripetuti cicli di idrolisi dell'ATP il complesso pu\`o catalizzare
uno scorrimento del nucleosoma in modo da riposizionarlo esponendo specifiche regioni di DNA rendendole disponibili ad altre proteine. Inoltre cooperando con altre altre proteine che si
legano agli istoni e servono come accompagnatrici di istoni questi complessi sono capaci di rimuovere parte o tutto il nucleo del nucleosoma. Un tipico nucleosoma \`e sostituito ogni
ora o due all'interno della cellula. Questi complessi esistono in molte varianti con ruoli diversi. La maggior parte sono proteine con $10$ o pi\`u subunit\`a altri creano specifici
cambi sugli istoni. Quando i geni sono attivati o disattivati questi complessi vengono portati a regioni specifiche del DNA dove agiscono sulla struttura cromatinica. La maggior 
influenza sul posizionamento del nucleosoma sembra essere altre proteine strettamente legate al DNA, che possono favorire la presenza di un nucleosoma adiacente a loro o forzare il 
suio allontanamento. 
\subsection{I nucleosomi sono tipicamente condensati in una fibra cromatinica}
I nucleosomi sono tipicamente condensati uno sull'altro, generando vettori in cui il DNA \`e altamente denso, pertanto la cromatina forma una fibra. Questa vicinanza \`e causata da 
legami tra nucleosomi che coinvolgono la coda degli istoni e un istone addizionale che svolge la funzione di linker (H1) che si lega a ogni nucleosoma e cambia il percorso del DNA 
quando esce dal nucleosoma. 
\section{La struttura e la funzione della cromatina}
I meccanismi che creano le strutture cromatiniche in diverse regioni della cellula hanno diverse funzioni e possono essere ereditate attraverso ereditariet\`a epigenetica, ovvero 
ereditariet\`a superimposta sull'ereditariet\`a genetica del DNA. 
\subsection{L'eterocromatina \`e altamente organizzata e limita l'espressione genica}
Esistono due tipi di cromatina nell'interfase dei nuclei di molte cellule eucariotiche: una forma altamente condensata detta eterocrmatina e una meno condensata detta eucromatina. 
L'eterocromatina \`e altamente concentrata in certe regioni specializzate: i centromeri e nei telomeri e presente in luoghi che variano in base allo stato fisiologico della cellula. 
Contiene tipicamente pochi geni e quando parti eucromatiche vengono convertite in eterocromatiche i rispettivi geni vengono disattivati. Descrive domini di cromatina compatti che sono
resistenti all'espressione genica.
\subsection{Lo stato eterocromatico \`e autopropagante}
Attraverso la rottura e la riunione dei cromosomi una parte eucromatica pu\`o essere trasformata in una eterocromatica in un effetto dettodi posizione. Riflette l'allargamento dello 
stato eterocromatico ed \`e il modo in cui l'eterocromatina \`e creata e mantenuta. QUando una condizione eterocromatica si stabilisce in una zona viene ereditata dalle cellule figlie
nella variegazione dell'effetto di posizione. Si nota pertanto come l'eterocromatina genera s\`e stessa in un feedback positivo che si espande spazialmente e temporalmente. Esistono 
geni che aumentano o diminuiscono questo processo che codificano proteine cromosomali non istoni che interagiscono con gli istoni coinvolti nella modifica o mantenimento della struttura
cromatinica.
\subsection{I nuclei istonici sono modificati covalentemente a molti siti diversi}
Le catene laterali degli amminoacidi dei quattro istoni del nucleosoma sono soggette a molti cambi covalneti, come l'acetilazione della lisina, la mono-, di- e triametilazione delle 
lisinee la fosforilazione delle serine. Un gran numero di queste modifiche avvengono sulle code istoniche \ce{N-} terminali che protrudono dal nucleosoma. Sono tutte reversibili e 
catalizzate da enzimi altamente specifici. Ogni enzima \`e reclutato su siti specifici della cromatina in tempi determinati in base alle proteine regolatorie della trascrizione o 
fattori di trascrizione che riconoscono e si legano a specifiche sequenze di DNA. Sono prodotte in tempi e luoghi diversi, determinando dove e quando tali enzimi agiscono. Il DNA 
determina pertanto come gli istoni sono modificati, ma in alcuni casi le modifiche covalenti possono persistere creando una memoria della storia di sviluppo della cellula che pu\`o 
essere trasmessa ereditariamente. Differenti gruppi di nucleosomi presentano diverse modifiche, che vengono controllate e hanno importanti conseguenze. L'acetilazione delle lisine 
allenta la struttura cromatinica in quanto il gruppo acetile rimuove la carica positiva della lisina, riducendone l'affinit\`a con le code dei nucleosomi adiacenti, ma gli effetti pi\`u 
importanti sono generati dall'abilit\`a di reclutare altre proteine in queste lunghezze di cromatina alterata. La trimetilazione di una specifica lisina sulla coda H3 attrae la proteina
HP1 eterocromatina-specifica contribuendo alla stabilizzazione ed espansione della cromatina. Le proteine reclutate agiscono con gli istoni modificati determinando dove e quando i geni
saranno espressi. Si nota pertanto come la struttura di ogni dominio cromatinico governa la lettura delle informazioni genetiche e infine struttura e funzioni della cellula.
\subsection{La cromatina acquisisce variet\`a addizionale attraverso inserzioni specifiche al sito di un insieme di varianti di istoni}
Nella cellula sono contenuti alcune varianti di istoni che si possono assemblare in nucleosomi. Gli istoni principali sono sintetizzati primariamente durante la fase S del ciclo 
cellulare e si assemblano in nucleosomi sulle eliche di DNA figlie dietro la forcella di replicazione, mentre le farianti sono sintetizzate nell'interfase e sono inseriti in cromatine
gi\`a formate con un processo di cambio istonico catalizzato dal complesso di rimodellazione della cromatina ATP-dipendente che contiene subunit\`a che causano il suo legame a siti 
specifici della cromatina e a accompagnatori di istoni che trasportano una particolare variante, inserendoli pertanto in luoghi con alta specificit\`a.
\subsection{Le modifiche covalenti e le varianti istoniche agiscono insieme per controllare le funzioni dei cromosomi}
Le modifiche degli istoni avvengono in insiemi ordinati. Alcune combinazioni determinano come e dove il DNA condensato possa essere acceduto e manipolato, creando un insieme di marcature
che determinano se la lunghezza di cromatina \`e stata appena replicata, o \`e danneggiata o come e dove l'espressione genica deve accadere. Molte proteine regolatorie contengono piccoli
domini che si legano a marcature specifiche come la lisina trimetilata sull'istone H3. Questi domini sono legati insieme in una grande proteina o complesso che riconosce una combinazione
specifica di modifiche all'istone. Si crea pertanto un complesso lettore che permette a particolari combinazioni di marcature di attrarre proteine per eseguire le funzioni biologiche
nel momento corretto. Queste marcature sono dinamiche e si creano sulle code istoniche che si trovano sull'esterno del nucleo del nucleosoma.
\subsection{Un complesso di proteine lettrici e scrittrici possono espandere modifiche alla cromatina lungo un cromosoma}
Il fenomeno dell'effetto di posizione richiede che forme modificate di cromatina abbiano l'abilit\`a di espandersi per distanze sostanziale lungo il cromosoma. Si nota come gli enzimi 
che aggiungono o rimuovono modifiche agli istoni fanno parte di complessi che possono essere trasportati dai regolatori di trascrizione su lunghezze specifiche di DNA, ma dopo che
l'enzima marca i nucleosomi pu\`o avvenire una reazione a catena in cui lavora insieme a una proteina di lettura nello stesso complesso che riconosce la marcatura e si lega a tale 
nucleosoma, attivando l'enzima di scrittura e posizionandolo su un nucleosoma vicino. Attraverso i cicli di lettura-scrittura la proteina pu\`o trasportare l'enzima di scrittura lungo
il DNA espandendo la marcatura lungo il cromosoma. Un processo simile avviene per rimuovere le modifiche agli istoni. 
\subsection{Barriere nella sequenza di DNA bloccano l'espansione dei complessi di lettura-scrittura separando domini di cromatina vicini}
Nella cellula esistono sequenze di DNA che marcano i confini di domini di cromatina e li separano. Questa sequenza di barriera contiene un insieme di siti di legame per gli enzimi
istoni acetilasi. Siccome la lisina acetilata \`e incompatibile con la metilazione richiesta per l'espansione dell'eterocromatina si blocca la sua espansione. Esistono comunque molte
altre modifiche alla cromatina che raggiungono questo scopo.
\subsection{La cromatina nei centromeri rivela come le varianti degli istoni possono creare strutture speciali}
Le varianti di istoni che trasportano nucleosomi possono produrre marcature nella cromatina persistenti. Un esempio di questo \`e la formazione e ereditariet\`a di una struttura di 
cromatina specializzata al centromero, la regione necessaria per l'attacco al mandrino mitotico e alla segregazione ordinata delle copie del genoma durante la divisione cellulae. Ogni
centromero \`e contenuto da una cromatina centromerica che persiste durante l'interfase che cotniene una variante dell'istone H3 nota come CENP-A (centromere protein-A) e altre 
proteine che condensano il nucelosoma in ordinamenti particolarmente densi e formano il cinetocoro, la struttura richiesta per l'attacco al mandrino mitotico. I centromeri consistono 
di sequenze di DNA corte e ripetute dette DNA alfa satellite. La presenza di queste sequenze in altre parti di DNA suggerisce come da sole non siano sufficienti per la formazione 
del centromero. Altri centromeri sono stati osservati su cromosomi frammentati e inizialmente eucromatici. I centromeri sono pertanto definiti da un insieme di proteine, piuttosto che
da una sequenza di DNA specifica. 
\subsection{Alcune strutture cromatiniche possono essere ereditate direttamente}
I cambi nell'attivit\`a del centromero devono essere trasmessi alle generazioni successive. Si nota come la formazione di un centromero de novo richiede un inizio evento di 
inseminazione con strutture specifiche al DNA che contengono nucleosomi con CENP-A, che avviene pi\`u rapidamente lungo il DNA alfa satellite. Essendo che i tetrameri di H3-H4 sono
ereditati si nota come un nucleosoma contenente CENP-A \`e stato assemblato su una lunghezza di DNA \`e banale come un nuovo centromero possa essere generato nello stesso luogo, 
asumendo che la presenza di un istone CENP-A recruti selettivamente pi\`u istoni CENP-A nei suoi vicini. Si nota inoltre come la generazione del centromero siaun processo altamente 
cooperativo.
\subsection{L'attivazione e la repressione di strutture di cromatina pu\`o essere ereditato epigeneticamente}
L'ereditariet\`a epigenetica \`e centrale nella creazione di organismi multicellulari in quanto permettono la creazione di tessuti diversi. Si nota come le strutture specifiche della
cromatina tendono a persistere e a essere trasmesse nei cicli di divisione.
\subsection{Le strutture di cromatina sono importanti per le funzioni del cromosoma eucariotico}
La condensazione del DNA in nucleosomi \`e fondamentale per l'evoluzione di organismi multicellulari in quanto cellule si devono specializzare cambiando l'accessibilit\`a e l'attivit\`a
di centinaia di geni, processo che richiede sulla memoria cellulare che si trova in parte nella struttura cromatinica che crea strutture che silenziano geni temporaneamente o in 
maniera persistente.
\section{La struttura globale dei cromosomi}
Come una fibra a $30nm$ il nucleo avrebbe una lunghezza di $1mm$ e sarebbe troppo grande per essere contenuto nel nucleo. Si rende pertanto necessario un secondo livello di piegamento
che coinvolge il piegamento della cromatina in una serie di anelli e bobine in maniera dinamica.
\subsection{I cromosomi sono piegati in grandi anelli di cromatina}
I cromosomi, accoppiati in preparazione per la meiosi si presentano in una serie di grandi anelli di cromatina che si emanano da un asse cromosomico lineare. In queste circostanze un
anello contiene sempre la stessa sequenza di DNA che rimane estesa nello steso modo. Questi cromosomi producono grandi quantit\`a di RNA e la maggior parte dei geni negli anelli sono
espressi. La maggior parte del DNA si trova comunque condensata nell'asse dove i geni non sono espressi. 
\subsection{Cromosomi politene sono utili per visualizzare le strutture di cromatina}


\chapter{Introduzione}
\section{Microscopia}
I microscopi possono essere divisi in due categorie principali: i microscopi ottici (composti, semplici o a fluorescenza) e i microscopi elettronici. Mentre
i primi utilizzano la luce e lenti ingrandenti i secondi utilizzano fasci di elettroni direzionati attraverso campi magnetici. Diverse tecniche di 
microscopia elettronica includono trascrizione, sezione, freeze fracture e scansione. I microscopi elettronici generano immagini unicamente in bianco e nero
e si rende pertanto necessario utilizzare dei falsi colori in un secondo momento, ma hanno come vantaggio una maggiore risoluzione delle immagini. Per un
microscopio ottico l'ingrandimento dell'immagine \`e dato dall'ingrandimento dell'obiettivo moltiplicato per l'ingrandimento degli eye-pieces: 
$$M_{microscope} = M_{objective}\cdot M_{eyepieces}$$
\subsection{Risoluzione}
La risoluzione di un immagine \`e una misura del dettaglio che essa contiene e se attraverso tecniche digitali l'ingrandimento \`e illimitato la risoluzione
no. Si indica con risoluzione la minor distanza tra due punti che possono essere distinti come separati. La risoluzione dipende da parametri del'utente 
finale e da parametri fisici. 
\subsubsection{Parametri fisici}
I parametri fisici che determinano la risoluzione sono:
\begin{itemize}
\item Il corretto allineamento del sistema ottico del microscopio.
\item La lunghezza d'onda della luce ($\lambda$): maggiore la lunghezza d'onda minore la risoluzione.
\item L'apertura numerica (NA) dell'obiettivo e del condensatore. Questo parametro indica la capacit\`a di un obiettivo di raccogliere luce e risolvere 
dettagli ad una distanza fissata dall'oggetto. Dipende dall'ingrandimento e dall'indice di rifrazione del medium tra microscopio e oggetto (aria, acqua, 
olio). Maggiore l'indice di rifrazione maggiore il numero di apertura e maggiore la risoluzione.
\end{itemize}
\subsection{Tecniche di microscopia ottica}
\subsubsection{Microscopia a cambio di fase}
Nella microscopia a cambio di fase viene sfruttato lo shift di fase della luce quando attraversa il corpo che si vuole osservare. L'interpretazione dello 
shift d\`a origine a diverse possibili considerazioni sull'oggetto.
\subsubsection{Microscopia a fluorescenza}
La microscopia a fluorescenza sfrutta il fenomeno della fluorescenza: certe molecole, quando colpite da certe lunghezze d'onda, si eccitano. Successivamente
quando la molecola ritorna dallo stato eccitato a quello basale emette a sua volta un'onda luminosa di lunghezza d'onda maggiore di quella con cui era
stata colpita (legge di Stokes). Per questa tecnica vengono tipicamente utilizzate delle proteine come GFP (green fluorescent protein), YFP, CFP tipicamente
estratte da organismi (nel caso della GFP da una medusa) che permettono pertanto la loro codifica nel DNA della cellula. Queste proteine vengono aggiunte ad
una proteina in modo da riuscire ad osservarne il comportamento. Si deve prestare attenzione al fatto che per massa o struttura questa aggiunta potrebbe 
creare una variazione in comportamento della proteina oggetto di studio. I fattori essenziali per la microscopia a fluorescenza sono:
\begin{itemize}
\item Eccitazione di alta intensit\`a.
\item Filtri di eccitazione e emissione appropriati.
\item Autoflorescenza minima nell'oggetto di studio.
\item Utilizzo di un olio di immersione non fluorescente. 
\item Antifade reagents.
\end{itemize}
Questa tecnica permette pertanto di osservare proteine in cellule vive a differenza dell'antibody staining. Un'importante applicazione \`e 
l'immunoistochimica.
\paragraph{FRET}
Le tecnica FRET si utilizza per determinare la prossimit\`a di due diverse proteine: si attaccano ad esse due molecole fluorescenti tali che l'emissione 
della prima eccita la seconda, che a sua volta emette luce. Se questo accade si \`e dimostrato che le due proteine sono vicine.
\paragraph{Photobleach}
Questa tecnica viene utilizzata per mostrare la velocit\`a di movimento di una proteina della cellula: si trova una cellula di controllo e quella oggetto di 
studio entrambe con la proteina fluorescente. La seconda viene sottoposta ad un raggio ad alta potenza (photoactivation) che distrugge la parte fluorescente
della proteina. Si osserva ora la velocit\`a con cui la luminescenza torna nella zona colpita, confrontandola con la cellula di controllo e si crea il grafo
di velocit\`a di movimento della proteina.
\section{Dogma centrale della biologia molecolare}
$$DNA\xrightarrow[\leftarrow]{}mRNA\rightarrow proteina$$
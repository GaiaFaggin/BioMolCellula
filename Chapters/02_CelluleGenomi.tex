\chapter{Cellule e genomi}
Tutti gli esseri viventi sono fatti di cellule e queste unit\`a di materia vivente hanno in comune lo stesso macchinario che svolge le stesse funzioni di 
base. Si nota pertanto il contrappunto tra enorme differenza tra gli individui se osservati all'esterno ma una straordinaria somiglianza nei meccanismi 
fondamentali. 
\section{Caratteristiche universali della vita sulla Terra}
Ciascuna specie presente sulla Terra \`e diversa e si riproduce fedelmente producendo una progenie che appartiene alla stessa specie. L'organismo genitore 
passa l'informazione che specifica in modo dettagliato le caratteristiche che la progenie avr\`a. Questo fenomeno, detto ereditariet\`a \`e ci\`o che 
distingue la vita da altri processi. L'organismo vivente inoltre consuma energia libera per creare e mantenere la sua organizzazione che spinge un sistema 
complesso di processi chimici nel modo specificato dall'informazione ereditaria. Sia che l'organismo sia costituito da una singola cellula o da 
raggruppamenti di cellule specializzate collegati da sistemi complessi di comunicazione \`e stato generato da divisioni cellulari di una singola cellula.
La singola cellula \`e perci\`o il il veicolo dell'informazione ereditaria che definisce la specie e include il meccanismo necessario alla sua copia. 
\subsection{Tutte le cellule conservano la loro informazione ereditaria nello stesso codice chimico lineare (DNA)}
Tutte le cellule viventi sulla Terra conservano le loro informazioni sotto forma di molecole a doppio filamento di DNA, lunghe catene polimeriche accoppiate
senza ramificazioni, formate sempre dagli stessi quattro tipi di monomeri, con nomi derivati da un alfabeto a quattro lettere (A, C, G, T). Questi monomeri
sono attaccati in una lunga sequenza lineare che codifica l'informazione genica. Essendo il DNA una struttura utilizzata da tutte le cellule viventi il DNA
di un essere umano \`e leggibile, copiabile e interpretabile da una cellula batterica (e viceversa). Utilizzando metodi chimici si pu\`o leggere la sequenza
completa di monomeri in qualunque molecola di DNA e decifrare cos\`i l'informazione ereditaria contenuta in qualsiasi organismo. 
\subsection{Tutte le cellule replicano la loro informazione ereditaria mediante polimerizzazione su uno stampo}
I meccanismi che rendono possibile la vita dipendono dalla struttura della molecola di DNA a doppio filamento. Ciacun monomero nel DNA, detto nucleotide, 
\`e composto da uno zucchero (deossiribosio) con un gruppo fosfato attaccato e una base che pu\`o essere adenina (A), guanina (G), citosina (C) o timina 
(T). I nucleotidi creano una catena polimerica composta da un'ossatura ripetitiva zucchero-fosfato con una serie di basi che sporgono da un lato. Essendo 
l'unit\`a zucchero-fosfato asimmetrica la catena ha una polarit\`a che determina l'ordine di lettura. Il polimero di DNA viene esteso aggiungendo monomeri 
ad una estremit\`a. Essendo la base uguale per tutti in teoria qualsiasi base pu\`o essere aggiunta in qualsiasi momento. Essendo che nella cellula vivente 
il DNA viene sintetizzato su uno stampo formato da un filamento preesistente di DNA le basi che sporgono dal filamento esistente si legano a basi del 
filamento che viene sintetizzato, secondo una regola rigida definita da strutture complementari delle basi: A si lega a T e C a G. Questi accompiamenti di 
basi mentengono i nuovi monomeri in posizione e controllano la scelta del nuovo monomero da aggiungere. In questo modo si crea una struttura a doppio 
filamento composta da due catene complementari di nucleotidi e un'ossatura con polarit\`a inversa. I nucleotidi di ciascun filamento si uniscono tra di loro 
attraverso legami covalenti, mentre con i corrispettivi nell'altro con legami ad idrogeno. I due filamenti si avvolgono l'uno sull'altro formando la 
struttura a doppia elica. Essendo i legami tra le basi deboli i due filamenti possono separarsi in modo da fornire lo stampo per una nuova replicazione. 
Questo processo di replicazione del DNA avviene con ritmi, controlli e molecole ausiliarie diverse, ma le basi sono universali: il DNA \`e il depositario 
dell'informazione e la polimerizzaione a stampo \`e il modo con cui l'informazione viene copiata e propagata. 
\subsection{Tutte le cellule trascivono porzioni della loro informazione ereditaria nella stessa forma intermedia (RNA)}
Si rende necessario esprimere le informazioni del DNA in modo da guidare la sintesi di altre molecole nella cellula. Anche questo processo \`e universale e 
produce principalmente RNA e proteine. Il processo inizia con una polimerizzazione su stampo detta trascrizione, in cui segmenti del DNA sono usati come
stampo per la sintesi di molecole pi\`u corte di acido-ribonucleico o RNA. In seguito durante la traduzione queste molecole dirigono la sintesi di proteine.
Nell'RNA l'ossatura \`e formata da ribosio e la timina viene sostituita dall'uracile (U). Durante la trascrizione i monomeri dell'RNA sono allineati e 
scelti per la polimerizzazione su un filamento stampo di DNA. Il risultato \`e un polimero che rappresenta una parte dell'informazione genetica della 
cellula. Lo stesso segmento di DNA pu\`o essere utilizzato per la sintesi di molti trascritti identici di RNA. Si noti pertanto come se il DNA rimane unico 
e stabile per la cellula l'RNA \`e monouso e prodotto in massa. Questi trascritti sono intermedi nel trasferimento dell'informazione genetica: servono 
soprattutto da RNA messaggero (mRNA) che guida la sintesi di proteine secondo le istruzioni conservate nel DNA. Le molecole di RNA possiedono anche 
strutture caratteristiche che possono conferire capacit\`a chimiche specializzate. Essendo a filamento singolo possono ripiegarsi all'indietro su s\`e 
stesse per formare legami deboli tra le basi, situzione che avviene quando segmenti della sequenza sono localmente complementari. La forma viene pertanto 
dettata dalla sequenza e pu\`o permettere alla molecola di essere scelta selettivamente e di catalizzare modificazioni chimiche cruciali per alcuni dei 
processi pi\`u antichi e fondamentali della cellula.
\subsection{Tutte le cellule usano proteine come catalizzatori}
Anche le proteine sono lunghe catene polimeriche non ramificate formate dall'unione in serie di monomeri comuni a tutti gli esseri viventi. Portano 
un'informazione sotto forma di una sequenza lineare di simboli. Dopo l'acqua sono l'elemento pi\`u presente nella cellula. I monomeri sono detti amminoacidi
e ne esistono 20, sono tutti formati dalla stessa struttura centrale standard che permette la formazione di catene a cui \`e attaccato un gruppo laterale 
che determina il carattere chimico specifico. Ciacuna molecola proteica o polipeptide si ripiega in una forma tridimensionale precisa con siti reattivi 
sulla sua superficie. Questi polimeri di amminoacidi si legano con alta specificit\`a ad altre molecole e agiscono da enzimi che catalizzano reazioni in cui
vengono rotti o creati legami covalenti e dirigono pertanto la maggioranza dei processi chimici. Hanno anche funzione strutturale, motile, di rilevazione
segnali, che viene determinata in base alla sequenza di amminoacidi creata dalla sequenza genica. Il circuito a feedback di catalizzazione del processo di
duplicazione del DNA da parte delle proteine, che viene poi utilizzato per produrre proteine e RNA \`e alla base del comportamento autocatalitico e capace 
di autoriprodursi degli organismi viventi. 
\subsection{Tutte le cellule traducono RNA nello stesso modo}
La traduzione dell'informazione genica \`e un processo complesso. L'informazione contenuta in un mRNA \`e letta in gruppo di tre nucleotidi alla volta: 
ciascuna tripletta di nucleotidi (codone) codifica un singolo amminoacido. Questa codifica porta a ridondanza in quanto ci sono 64 codoni ma 20 amminoacidi.
Il codice \`e letto dal tRNA (RNA transfer). Ogni tipo di tRNA posside ad un'estremit\`a un amminoacido specifico e all'altra estremit\`a una sequenza di
tre nucleotidi (anticodone) che gli permette tramite l'accoppiamento di basi di riconoscere un gruppo di codoni nell'mRNA. Per la sintesi proteica una 
successione di molecole di tRNA cariche degli amminoacidi deve legarsi all'mRNA e gli amminoacidi devono essere uniti per espandere la catena proteica e i 
tRNA liberati dal loro carico devono essere rilasciati. Questo insieme di processi viene eseguito dal ribosoma formato da due catene principali di RNA
detto rRNA (RNA ribosomiale) e da un gran numero di proteine diverse. Questa struttura si attacca ad un'estremit\`a dell'mRNA e si sposta lungo di essa
catturando molecole di tRNA cariche e mettendo insieme gli amminoacidi in modo da formare una nuova catena proteica. 
\subsection{Il frammento di informazione genica che corrisponde ad una proteina \`e un gene}
Le molecole di DNA contengono le specifiche per migliaia di proteuine. Sequenze speciali nel DNA servono come punteggiatora, indicando dove l'informazione di ciascuna proteina inizia e
finisce. Ogni segmento del DNA \`e trascritto in una molecola di mRNA, codifica di diverse proteine. Tali segmenti sono i geni. Le molecole di RNA trascritte possono essere processate
in modi diversi in modo da generare un insieme di versioni alternatice di una proteina. Oltre a questo alcune parti sono trascritte in RNA con funzioni catalitiche, strutturali o 
regolatorie. Un gene viene pertanto definito come il segmento della sequenza di DNA corrispondente a una singola proteina o un insime di proteine varianti o a unna singola molecola di 
RNA catalitica, strutturale o regolatoria. L'espressione dei geni \`e regolata in base alla necessit\`a della proteina o RNA che producono. Questo avviene grazie a lunghezze di DNA
regolatorio che intramezzano la sequenza che legano delle protine che controllano il tasso della trascrizione. In questo modo il genoma della cellula determina la natura delle proteine e
quando devono essere prodotte.
\subsection{La vita richiede energia libera}
Una cellula vivente \`e un sistema chimico dinamico che opera lontano dall'equilibrio chimico. Affinch\`e una cellula cresca deve prendere dall'ambiente energia e materiali per far
avvenire le riezioni. Questo consumo di energia \`e ci\`o che tiene la cellula in vita. L'energia \`e anche fonamentale per la propagazione dell'informazione genetica. Il processo che
guida la formazione dei legami che determinano le molecole all'interno della cellula richiede energia per formare legami abbastanza forti da resistere a pressioni esterne. 
\subsection{Tutte le cellule funzionano come fabbriche biochimiche che impiegano le stesse strutture molecolare base}
Siccome tutte le cellule creano DNA, RNA e proteine, devono contenere e manipolare una collezione simile di zuccheri semplici, nucleotidi e amminoacidi e altre sostanze come l'ATP 
(adenina trifosfato) per la sintesi di DNA e RNA e come trasportatore di energia libera. 
\subsection{Tutte le cellule sono chiuse in una membrana plasmatica attraverso la quale nutrienti e materiali di scarto devono passare}
Un'altra caratteristica universale \`e che ogni cellula \`e rinchiusa da una membrana citoplasmatica che agisce da barriera selettiva e permette alla cellula di concentrare i nutrienti e
di conservare i prodotti delle sintesi, escrescendo i materiali di scarto. Le molecole che formano la membrana sono anfipatiche, ovvero formate d auna parte idrofobica e una idrofila. 
Queste molecole poste in acqua si aggregano spontaneamente in modo da avvicinare tra di loro (e allontanare dall'acqua) la parte idrofobica. Tali tipi di molecole, come i fosfolipidi si
aggregano in acqua per formare vescicoli a bistrato. Tipicamente la coda idofobica \`e costituita da polimeri di idrocarburi e dimostra perfettamente la tendenza della cellula di formare
molecole le cui propriet\`a chimiche causano un auto-assemblamento nella strututtura necessaria alla cellula. Naturalmente il confine della cellula deve poter permettere il passaggio di 
alcuni elementi e sono pertanto presenti proteine di trasporto di membrana il cui compito di trasportare le molecole che entrano ed escono dalla cellula. 
\section{La diversit\`a dei genomi e l'albero della vita}
Il successo degli organismi viventi li ha portati ad occupare qualsiasi luogo sulla Terra. La maggior parte di questi organismi rimane per\`o microscopica e invisibile all'occhio nudo.
\subsection{Le cellule possono ottenere energia da una variet\`a di fonti di energia libera}
Alcuni organismi come animali, funghi e molti batteri ottengono energia cibandosi di altri organismi viventi o dei prodotti chimici organici che producono, chiamati organotrofi. Altri 
la ottengono dal mondo non vivente e si dividono in due categorie: quelli che raccolgono l'energia dalla luce solare (fototrofi) e quelli che la ottengono da sistemi inorganici ricchi di
essa (litotrofi). Gli organismi organotrofi non potrebbero esistere senza questi due. Gli organismi fototrofi includono batteri, alghe e piante, i litotrofi sono microscopici e vivono
in condizioni impossibili per l'uomo. Alcuni litotrofi ottengono energia da reazioni aerobiche, altri da reazioni anaerobiche. 
\subsection{Alcune cellule fissano azoto e anidride caronica per altre}
La maggior parte della materia che compone le proteine \`e composta di idrogeno, carbonio, azoto, ossigeno, zolfo e fosforo, molto presenti in ambienti non viventi, ma non in una forma
chimica facilmente incorporata in molecole biologiche. Una gran quantit\`a di energia \`e necessaria per guidare le reazioni che usano le molecole inorganiche di $N_2$ e $CO_2$ per 
fissare $N$ e $C$ e renderli disponibili agli organismi. Pertanto classi di cellule sono specializzate per fare questo lavoro. Si nota pertanto come le cellule viventi possano variare 
infinitamente in aspetti della loro biochimica. 
\subsection{La pi\`u grande diverit\`a biochimica esiste tra le cellule procariote}
Gli organismi viventi, in base alla loro struttura cellulare possono essere divisi in eucarioti e procarioti. Gli eucarioti mantengono il proprio DNA all'interno di una membrana 
intracellulare chiamata il nucleo, mentre i procarioti non possiedono questo compartimetno. La maggior parte delle cellule procariote sono piccole e vivono principalmente come individui
indipendenti o in comunit\`a vagamente organizzate. Possiedono tipicamente un rivestimento protettivo spesso detto parete cellulare sotto il quale si trova una membrana plasmatica
che contiene un singolo compartimento citoplasmatico che contiene tutte le molecole necessarie per la vita. Vivono in una variet\`a di nicchie ecologice e sono molto vari nelle loro
capacit\`a biochimiche. 
\subsection{L'albero della vita possiede tre rami principali: Batteri, Archei ed Eucarioti}
La classificazione delle cose viventi dipende da somiglianze comuni, che, come mostrato da Darwin, suggerisce un antenato comune relativamente recente. I procarioti sono classificati in
termini della loro biochimica e necessit\`a nutrizionali. L'analisi genomica fornisce un modo pi\`u semplice e diretto di determinazione di relazioni evolutive. La sequenza di DNA di 
un organismo definisce la sua natura in maniera esaustiva e permette una facile comparazione con le informazioni corrispondenti in altre forme viventi e pertanto una determinazione 
immediata della loro distanza evolutiva. Questo ha permesso la distinzione dei procarioti in batteri e archei e come la prima cellula eucariota si sia generata da una cellula archea 
penetrata in un batterio antico. 
\subsection{Alcuni geni evolvono rapidamente, altri sono altamente conservati}
Sia durante la conservazione che la copiatura dell'informazione genetica ci possono essere degli errori che alterno la sequenza nucleotidica creando delle mutazioni. Pertanto quando una
cellula si divide le sue figlie sono spesso non del tutto uguali tra di loro. Questa mutazione pu\`o essere del tutto ininfluente, miglirare la cellula o causare seri problemi. 
Questi cambiamenti possono essere manteuti grazie alla selezione naturale ed \`e immediato notare come il terzo tipo di mutazione raramente verr\`a propagato. Alcune parti del genoma
possono cambiare pi\`u facilmente: un segmento di DNA che non codifica proteine e non ha significativi ruoli regolatori pu\`o cambiare ad un tasso limitato unicamente dalla frequenza
degli errori casuali, mentre un gene che codifica una proteina essenziale o una molecola di RNA genera quasi sempre una cellula che viene eliminata. Geni di quest ultimo tipo sono detti
altamente conservati. Questi sono i geni da osservare se si vogliono determinare le relazioni tra gli organismi pi\`u lontani. La classificazione nei tre domini si basa sull'analisi
delle componenti di rRNA dei ribosomi. 
\subsection{Nuovi geni sono generati da geni preesistenti}
Il materiale dell'evoluzione sono sequenze di DNA preesistenti e l'innovazione pu\`o accadere in molti modi:
\begin{itemize}
	\item Mutazione intragenica: un gene casuale pu\`o essere modificato da cambiamenti nella sua sequenza di DNA attraverso errori che accadono principalmente nel processo della
		replicazione del DNA.
	\item Duplicazione genica: un gene esistente pu\`o essere accidentalmente duplicato in modo da creare un paio di geni identici all'interno della cellula che possono 
		successivamente divergere. 
	\item Mescolamento dei segmenti di DNA: due o pi\`u geni esistenti possono rompersi e raggrupparsi creando un gene ibrido consistente di un segmento che prima apparteneva a 
		geni diversi.
	\item Trasferimento intracellulare orizzontale: un segmento di DNA pu\`o essere trasferito dal genoma di una cellula ad un'altra. 
\end{itemize}
Ognuno di questi cambi lascia delle tracce caratteristiche ed \`e chiaro come siano avvenuti tutti.
\subsection{La duplicazione genica permette la creazione di famiglie di geni imparentati in una stessa cellula}
Una cellula duplica il suo intero genoma ogni volta che si divide, ma pu\`o accadere che la duplicazione abbia degli errori con una conservazione di segmenti originali e duplicati in
una singola cellula. Una volta che un gene viene cos\`i duplicato, una delle coppie \`e libera di mutare e specializzarsi in una funzione diversa. Diverse iterazioni danno origine 
a famiglie di geni che possono essere trivati nello stesso genoma. Attraverso questo processo gli individui di una specie possiedono diverse varianti di un gene primordiale. Si chiamano
ortologhi quei geni che si trovano in due specie diverse e derivano dallo stesso gene ancestrale nell'ultimo antenato comune, mentre si chiamano paraloghi quei geni che sono risultati
da una duplicazione genica in un singolo genoma e probabilmente hanno ora funzioni differenti. Entrambi si classificano come geni omologhi.
\subsection{I geni possono essere trasferiti tra organismi}
I procarioti forniscono un buon esempio di trasferimento orizzontale dei geni da una specie di cellule all'altra. I segni pi\`u ovvi derivano da sequenze virali (dei batteriofagi). I
virus sono piccoli pacchetti di materiale genetico evoluti come parassiti sui processi biochimici e riproduttivi della cellula. Non sono organismi viventi ma servono come vettori per il
trasferimento di geni. Un virus si replica in una cellula, emerge da essa con un involucro potettivo e penetra, infettandola, un'altra cellula, che pu\`o essere anche di un'altra specie.
Spesso la cellula infettata pu\`o morire, ma alcune volte il DNA virale potrebbe persistere nell'host per molte generazioni come un passeggero innoquo come plasmide o come sequenza
inserita nel genoma reoglare. Nei viaggi i virus possono recuperare frammenti di DNA dal genoma della cellula host e trasportarli in un'altra. Questi scambi sono comuni in procarioti ma
rari in eucarioti di specie diverse. Molti procarioti possono recuperare anche DNA non virale dall'ambiente. Attraverso questi metodi batteri e archei possono acquisire geni da cellule
vicine facilmente. Gli scambi orizzontali hanno un analogo tra gli eucarioti nell'attivit\`a sessuale. 
\subsection{Molti geni sono comuni tra tutti i rami primari dell'albero della vita}
La sequenza di un gene permette di capire la sua funzione, confrontandola con un database preesistente. Data la sequenza genomica di organismi rappresentativi per archei, batteri e 
eucarioti e considerando gli scambi orizzontali si nota come i geni in comune sono principalmente quelli del sistema di traduzione, trascrizione e trasporto di amminoacidi. 
\subsection{Le mutazioni rivelano la funzione dei geni}
L'analisi dei geni dipende da due approcci complementari: genetica e biochimica. La prima comincia con uno studio dei mutanti: si trova un organismo in cui il gene \`e alterato e si 
esaminano gli effetti sulla struttura e prestazioni. La biochimica analizza invece la funzione delle molecole. Combinando le due \`e possibile trovare quelle molecole la cui produzione
dipende da un dato gene determinando allo stesso tempo il ruolo delle molecole nelle operazioni dell'organismo. La biologia molecolare ha permesso un rapido progesso in quanto
si possono testare le contribuzioni dei geni all'attivit\`a del loro prodotto costuendo geni artificiali che combinano parte di un gene e parte di un altro. Gli organismi possono essere
ingegnerizzati per produrre l'RNA o la proteina specificata dal gene in grandi quantit\`a.
\section{Informazione genetica negli eucarioti}
Le cellule eucariote sono pi\`u grandi ed elaborate rispetto alle cellule eucariote, come i loro genomi. La dimensione maggiore \`e accoppiata con radicali differenze strutturali e
funzionali. Inoltre le cellule eucariote formano organismi multicellulari che arrivano a livelli di complessit\`a maggiori. 
\subsection{Le cellule eucariote potrebbero essersi originate come predatori}
Per definizione le cellule eucariote mantengono il proprio DNA nel nucleo. L'involucro nucleare, un doppio strato di membrana circonda il nucleo e separa il DNA dal citoplasma. Sono
circa 1000 volte pi\`u voluminose rispetto ai procarioti e possiedono un citosceletro elaborato, un sistema di filamenti proteici nel citoplasma che forma, insieme ad altre proteine
un sistema strutturale e motile. Esiste una serie di membrane simile alla membrana citoplasmatica che racchiudono diversi spazi nella cellula, molti dei quali riguardano digesione e
secrezione. Non hanno una parete cellulare gli organismi procarioti unicellulari sono chiamati protozoi e possono cambiare la loro forma e intrappolare altre cellule e piccoli oggetti
attraverso fagocitosi. Una delle ipotesi sull'origine delle cellule eucariote riguarda una cellula primordiale predatrice. I movimenti rapidi erano necessri alla caccia e il nucleo 
necessario alla protezione del genoma rispetto ai movimenti del citoscheleto.
\subsection{Le moderne cellule eucariote si sono evolute da una simbiosi}
Tutte le cellule eucariote contengono i mitocondri, piccoli corpi nel citoplasma incapsulati in un doppio strato di membrana che consumano ossigeno e intrappolano energia 
dall'ossidazione di molecole nutritive come gli zuccheri per produrre la maggior parte dell'ATP. Sono simili in dimensioni ai piccoli batteri e possiedono il proprio genoma nella forma
di una molecola di DNA circolare, i propri ribosomi e il proprio rRNA. \`E generalmente accettato che si sono originati da una forma di batterio aerobico fagocitato da una cellula
ancestrale anaerobica. Sfuggendo alla digestione questi batteri si sono evoluti in simbiosi con la cellula fagocitante e la sua progenie, ricevendo protezione e nutrimento in cambio di
energia. Recenti analisi genomiche suggeriscono che la prima cellula eucariota si sia formata rispetto a una cellula archea fagocitata da un batterio aerobico. La maggior parte delle
cellule eucaariote di piante e alge contengono anche un'altra classe di membrane chiamate i cloroplasti che svolgono la fotosintesi che come i mitocondri possiedono il proprio genoma.
Si sono quasi certamente generati come batteri simbiontici fotosintetici, acquisiti da cellule eucariote che avevano gi\`a i mitocondri. Una cellula eucariota equipaggiata con i 
cloroplasti non necessita di predare in quanto riceve nutrimento dalla fotosintesi e pertanto hanno perso la capacit\`a di muoversi, scambiandola con la formaizone di una parete 
cellulare protettiva. I funghi, come le cellule animali possiedono mitocondri ma non cloroplasti e una spessa parete esterna che limita la loro abilit\`a di muoversi rapidamente o di 
fagocitare altre cellule. Si sono evoluti in spazzini, nutrendosi degli scarti di altre cellule o delle cellule stesse morte.
\subsection{Gli eucarioti possiedono genomi ibridi}
L'informazione genetica degli eucarioti ha un origine ibrida: dall'ancestrale cellula archea anaerobica e dal batterio che vi si \`e adattato come simbionte. La maggior parte 
dell'informazione \`e conservata nel nucleo, ma una piccola quanti\`a rimane all'interno del mitocondrio e nei cloroplasto. Quando il DNA di mitocondri e cloroplasto \`e separato dal 
DNA nucleare e analizzato e sequenziato si nota come siano versioni degeneri e ridotte dei corrispondenti genomi batterici. La ragione di questo \`e che molti geni si sono spostati da
essi al DNA del nucleo che mostra pertanto chiare prove dell'origine batterica.
\subsection{I genomi eucatiori sono grandi}
La selelzione naturale ha favorito mitocondri con piccoli genomi, mentre il genoma nucleare sembra essersi ingrandito, probabilmente la dimensione maggiore era un vantaggio nella vita
predatoria. Graze all'accumulazione di segmenti di DNA derivati dagli elementi trasportabili parassitici il genoma della maggior parte degli eucarioti \`e molti ordini di grandezza 
maggiore rispetto ai batteri e agli archei. Questa maggiore disponibilit\`a porta ad avere pi\`u geni e maggior parti di DNA non codificante ($98\%$ per gli esseri umani).
Molto del DNA non codificante \`e quasi certamente inutile, ma una parte svolge l'attivit\`a di regolare l'espressione dei geni adiacenti, cruciale per la formazione di sistemi 
multicellulari complessi. 
\subsection{Il genoma definisce il programma dello sviluppo multicellulare}
Le cellule in piante e animali sono estremamente varie, ma comunque derivano tutte dalla stessa cellula e contengono per la maggior parte copie identiche dello stesso genoma. Le 
differenze risultano dal modo in cui le cellule fanno uso selettivo delle proprie istruzioni genice secondo gli indizi che ricevono dal emprione sviluppante. La cellula si comporta come
una macchina multifunzione con sensori e abilit\`a di esprimere diversi insiemi di geni secondo la sequenza di segnali che riceve. Il genoma in ogni cellula \`e grande abbastanza da
accomodare le informazioni per l'intero sistema multicellulare, di cui viene usata un'unica parte. Un numero di geni codifica proteine che regolano l'attivit\`a di altri geni detti
regolatori di trascrizione e si legano, direttamente o indirettamente al DNA regolatorio adiacenti ai geni da controllare. Le cellule sono inoltre in grado di inviare segnali con i 
propri vicini, pertanto lo stesso sistema di controllo governa ogni cellula, con diverse conseguenze in base ai messaggi scambiati. Il risultato \`e un preciso insieme di cellule in 
stati diversi. 
\subsection{Molti eucarioti vivono come cellule solitarie}
Molte specie di cellule eucariote formano organismi unicellulari come predatori (protozoi) o come fotosintetizzatori (algae unicellulari) o come spazzini
(funghi unicellulari o lieviti). L'anatomia dei protozoi \`e elaborata e include strutture complesse come sensori, organi motili e di attacco. Nei termini
di lignaggio e sequenze di DNA questi organismi presentano molte pi\`u differenze rispetto alle controparti multicellulari. 

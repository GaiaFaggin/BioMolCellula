\chapter{Cellule e genomi}
Tutti gli esseri viventi sono fatti di cellule e queste unit\`a di materia vivente hanno in comune lo stesso macchinario che svolge le stesse funzioni di 
base. Si nota pertanto il contrappunto tra enorme differenza tra gli individui se osservati all'esterno ma una straordinaria somiglianza nei meccanismi 
fondamentali. 
\section{Caratteristiche universali della vita sulla Terra}
Ciascuna specie presente sulla Terra \`e diversa e si riproduce fedelmente producendo una progenie che appartiene alla stessa specie. L'organismo genitore 
passa l'informazione che specifica in modo dettagliato le caratteristiche che la progenie avr\`a. Questo fenomeno, detto ereditariet\`a \`e ci\`o che 
distingue la vita da altri processi. L'organismo vivente inoltre consuma energia libera per creare e mantenere la sua organizzazione che spinge un sistema 
complesso di processi chimici nel modo specificato dall'informazione ereditaria. Sia che l'organismo sia costituito da una singola cellula o da 
raggruppamenti di cellule specializzate collegati da sistemi complessi di comunicazione \`e stato generato da divisioni cellulari di una singola cellula.
La singola cellula \`e perci\`o il il veicolo dell'informazione ereditaria che definisce la specie e include il meccanismo necessario alla sua copia. 
\subsection{Tutte le cellule conservano la loro informazione ereditaria nello stesso codice chimico lineare (DNA)}
Tutte le cellule viventi sulla Terra conservano le loro informazioni sotto forma di molecole a doppio filamento di DNA, lunghe catene polimeriche accoppiate
senza ramificazioni, formate sempre dagli stessi quattro tipi di monomeri, con nomi derivati da un alfabeto a quattro lettere (A, C, G, T). Questi monomeri
sono attaccati in una lunga sequenza lineare che codifica l'informazione genica. Essendo il DNA una struttura utilizzata da tutte le cellule viventi il DNA
di un essere umano \`e leggibile, copiabile e interpretabile da una cellula batterica (e viceversa). Utilizzando metodi chimici si pu\`o leggere la sequenza
completa di monomeri in qualunque molecola di DNA e decifrare cos\`i l'informazione ereditaria contenuta in qualsiasi organismo. 
\subsection{Tutte le cellule replicano la loro informazione ereditaria mediante polimerizzazione su uno stampo}
I meccanismi che rendono possibile la vita dipendono dalla struttura della molecola di DNA a doppio filamento. Ciacun monomero nel DNA, detto nucleotide, 
\`e composto da uno zucchero (deossiribosio) con un gruppo fosfato attaccato e una base che pu\`o essere adenina (A), guanina (G), citosina (C) o timina 
(T). I nucleotidi creano una catena polimerica composta da un'ossatura ripetitiva zucchero-fosfato con una serie di basi che sporgono da un lato. Essendo 
l'unit\`a zucchero-fosfato asimmetrica la catena ha una polarit\`a che determina l'ordine di lettura. Il polimero di DNA viene esteso aggiungendo monomeri 
ad una estremit\`a. Essendo la base uguale per tutti in teoria qualsiasi base pu\`o essere aggiunta in qualsiasi momento. Essendo che nella cellula vivente 
il DNA viene sintetizzato su uno stampo formato da un filamento preesistente di DNA le basi che sporgono dal filamento esistente si legano a basi del 
filamento che viene sintetizzato, secondo una regola rigida definita da strutture complementari delle basi: A si lega a T e C a G. Questi accompiamenti di 
basi mentengono i nuovi monomeri in posizione e controllano la scelta del nuovo monomero da aggiungere. In questo modo si crea una struttura a doppio 
filamento composta da due catene complementari di nucleotidi e un'ossatura con polarit\`a inversa. I nucleotidi di ciascun filamento si uniscono tra di loro 
attraverso legami covalenti, mentre con i corrispettivi nell'altro con legami ad idrogeno. I due filamenti si avvolgono l'uno sull'altro formando la 
struttura a doppia elica. Essendo i legami tra le basi deboli i due filamenti possono separarsi in modo da fornire lo stampo per una nuova replicazione. 
Questo processo di replicazione del DNA avviene con ritmi, controlli e molecole ausiliarie diverse, ma le basi sono universali: il DNA \`e il depositario 
dell'informazione e la polimerizzaione a stampo \`e il modo con cui l'informazione viene copiata e propagata. 
\subsection{Tutte le cellule trascivono porzioni della loro informazione ereditaria nella stessa forma intermedia (RNA)}
Si rende necessario esprimere le informazioni del DNA in modo da guidare la sintesi di altre molecole nella cellula. Anche questo processo \`e universale e 
produce principalmente RNA e proteine. Il processo inizia con una polimerizzazione su stampo detta trascrizione, in cui segmenti del DNA sono usati come
stampo per la sintesi di molecole pi\`u corte di acido-ribonucleico o RNA. In seguito durante la traduzione queste molecole dirigono la sintesi di proteine.
Nell'RNA l'ossatura \`e formata da ribosio e la timina viene sostituita dall'uracile (U). Durante la trascrizione i monomeri dell'RNA sono allineati e 
scelti per la polimerizzazione su un filamento stampo di DNA. Il risultato \`e un polimero che rappresenta una parte dell'informazione genetica della 
cellula. Lo stesso segmento di DNA pu\`o essere utilizzato per la sintesi di molti trascritti identici di RNA. Si noti pertanto come se il DNA rimane unico 
e stabile per la cellula l'RNA \`e monouso e prodotto in massa. Questi trascritti sono intermedi nel trasferimento dell'informazione genetica: servono 
soprattutto da RNA messaggero (mRNA) che guida la sintesi di proteine secondo le istruzioni conservate nel DNA. Le molecole di RNA possiedono anche 
strutture caratteristiche che possono conferire capacit\`a chimiche specializzate. Essendo a filamento singolo possono ripiegarsi all'indietro su s\`e 
stesse per formare legami deboli tra le basi, situzione che avviene quando segmenti della sequenza sono localmente complementari. La forma viene pertanto 
dettata dalla sequenza e pu\`o permettere alla molecola di essere scelta selettivamente e di catalizzare modificazioni chimiche cruciali per alcuni dei 
processi pi\`u antichi e fondamentali della cellula.
\subsection{Tutte le cellule usano proteine come catalizzatori}
Anche le proteine sono lunghe catene polimeriche non ramificate formate dall'unione in serie di monomeri comuni a tutti gli esseri viventi. Portano 
un'informazione sotto forma di una sequenza lineare di simboli. Dopo l'acqua sono l'elemento pi\`u presente nella cellula. I monomeri sono detti amminoacidi
e ne esistono 20, sono tutti formati dalla stessa struttura centrale standard che permette la formazione di catene a cui \`e attaccato un gruppo laterale 
che determina il carattere chimico specifico. Ciacuna molecola proteica o polipeptide si ripiega in una forma tridimensionale precisa con siti reattivi 
sulla sua superficie. Questi polimeri di amminoacidi si legano con alta specificit\`a ad altre molecole e agiscono da enzimi che catalizzano reazioni in cui
vengono rotti o creati legami covalenti e dirigono pertanto la maggioranza dei processi chimici. Hanno anche funzione strutturale, motile, di rilevazione
segnali, che viene determinata in base alla sequenza di amminoacidi creata dalla sequenza genica. Il circuito a feedback di catalizzazione del processo di
duplicazione del DNA da parte delle proteine, che viene poi utilizzato per produrre proteine e RNA \`e alla base del comportamento autocatalitico e capace 
di autoriprodursi degli organismi viventi. 
\subsection{Tutte le cellule traducono RNA nello stesso modo}
La traduzione dell'informazione genica \`e un processo complesso. L'informazione contenuta in un mRNA \`e letta in gruppo di tre nucleotidi alla volta: 
ciascuna tripletta di nucleotidi (codone) codifica un singolo amminoacido. Questa codifica porta a ridondanza in quanto ci sono 64 codoni ma 20 amminoacidi.
Il codice \`e letto dal tRNA (RNA transfer). Ogni tipo di tRNA posside ad un'estremit\`a un amminoacido specifico e all'altra estremit\`a una sequenza di
tre nucleotidi (anticodone) che gli permette tramite l'accoppiamento di basi di riconoscere un gruppo di codoni nell'mRNA. Per la sintesi proteica una 
successione di molecole di tRNA cariche degli amminoacidi deve legarsi all'mRNA e gli amminoacidi devono essere uniti per espandere la catena proteica e i 
tRNA liberati dal loro carico devono essere rilasciati. Questo insieme di processi viene eseguito dal ribosoma formato da due catene principali di RNA
detto rRNA (RNA ribosomiale) e da un gran numero di proteine diverse. Questa struttura si attacca ad un'estremit\`a dell'mRNA e si sposta lungo di essa
catturando molecole di tRNA cariche e mettendo insieme gli amminoacidi in modo da formare una nuova catena proteica. 
\subsection{Il frammento di informazione genica che corrisponde ad una proteina \`e un gene}


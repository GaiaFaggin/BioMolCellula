\chapter{Replicazione, riparazione e ricombinazione del DNA}
L'abilit\`a della cellula di mantenere un alto grado di ordine dipende dall'accurata duplicazione di grandi quantit\`a di informazioni genetiche trasportate in forma chimica come DNA.
Il processo della replicazione del DNA deve avvenire prima che una cellula possa produrre due cellule figlie geneticamente indipendenti. Il mantenimento dell'ordine richiede anche una
continua sorveglianza e riparaizone dell'informazione genetica, continaumente danneggiata dall'ambiente chimicamente, attraverso radiazioni, calore e molecole attive generate nella 
cellula. Questi processi vengono svolti da proteine che catalizzano processi rapidi e accurati che avvengono all'interno della cellula. Se la sopravvivenza a breve termine di una cellula
dipende dalla prevenzione dei cambi nella sequenza, quella a lungo termine richiede che una sequenza di DNA cambi lungo le generazioni in modo da permettere adattamento evolutivo a un
ambiente dinamico. Nonostante gli sforzi della cellula avvengono dei cambi nel DNA che possono mettere a disposizioni varianti che la selelzione spinge durante l'evoluzione.
\section{Il mantenimento della sequenza di DNA}
La sopravviveza dell'individuo richiede un alto grado di stabilit\`a genetica. Solo raramente il processo di mantenimento del DNA fallisce causando cambi permenenti nel DNA o mutazioni, 
che possono distruggere un organismo se avvengono se in posizioni vitali della sequenza del DNA.
\subsection{I tassi di mutazione sono estremamenti bassi}
Il tasso di mutaziine pu\`o essere derminato direttamente attraverso esperimenti con batteri come l'Escherichia coli che si divide ogni $30$ minuti e una cellula singola pu\`o generare
una grande popolazione in meno di un giorno. In tale di popolazione \`e possibile individuare una piccola frazione di batteri con una mutazione dannosa in un gene particolare se on
\`e necessario alla sopravvivenza. Tale frazione \`e una sottostima delle mutazioni in quanto ne esistono silenti. Dopo aver corretto per queste mutazioni silenti si trova che un singolo
gene per una proteina di dimensione media ($10^3$ paia di nucleotidi) accumla una mutazione una volta ogni $10^6$ generazioni: il tasso di mutazione \`e pertanto di tre cambi 
nucleotidici per $10^{10}$ nucleotidi per generazione di cellule. Recentemente \`e stato possbile misurare il tasso di mutazione direttamente in organismi pi\`u complessi e la 
riproduzione sessuale. In questo caso la sequenza genomica di una famiglia si sequenzia e una comparazione determina circa $70$ mutazionidi singoli nucleotidi in ogni discendente. 
Normalizzando alla dimensione del genoma umano il tasso di mutazione di un nucleotide cambia per $10^8$ nucleotidi per generazione. Questa \`e una sottostima in quanto non considera le 
mutazioni letali che non sarebbero presenti nella progenie. Circa $100$ divisioni accadono tra il concepimento e il tempo di produzione di uova e sperma che producono una nuova 
generazione, pertanto il tasso umano di mutazione \`e di $1$ mutazione per $10^{10}$ divisioni cellulari. In entrambi gli esperimenti si nota come i tassi di mutazioni sono estremamente
bassi e con un fattore di tre l'uno dall'altro: sono infatti preservati i meccanismi base che garantiscono questi tassi bassi e sono stati conservati da cellule ancestrali molto antiche.
\subsection{I tassi di mutazione bassi sono necessari per la vita come la conosciamo}
Essendo che la maggior parte delle mutazioni sono deleterie, nessuna specie pu\`o permettersi di accumularle. Pur essendo il tasso di mutazioni basso si pensa limiti il numero di 
proteine che pu\`o dipendere a $30^.000$ in quanto per un numero maggiore la proabilit\`a che una componente chiave sia danneggiata da una mutazione diventa troppo elevata. Le cellule di
un organismo che si riproduce sessualmente sono di due tipi: le cellule germinali e quelle somatiche. Le prime trasmettano l'informazione tra genitore e prole, mentre le seconde formano
il corpo dell'organismo. Le cellule germinali devono esssere protette contro alti tassi di mutazione per mantenere la specie, cosa che deve avvenire anche nelle cellule somatiche per 
la corretta formazione di un corpo organizzato. I cambi nucleotidici nelle cellule somatiche possono creare cellule varianti, alcune delle quali attraverso selezione naturale locale
proliferano rapidamente. Nei casi estremi si genera un cancro, la cui probabilit\`a aumenta linearmente con il tasso di mutazione. Pertanto sia la perpetuazione di una specie con un 
gran numero di geni e la prevenzione di cancri che risultano da mutazioni delle cellule somatiche dipendono dall'alta fedelt\`a con cui le sequenze di DNA sono replicate e mantenute. 
\section{Meccanismi di replicazione del DNA}
Tutti gli organismi duplicano il proprio DNA con estrema accuratezza prima di ogni divisione cellulare.
\subsection{L'accoppiemaneto delle basi sottost\`a la replicazione e riparazione del DNA}
Il meccanismo che usa la cellula per copiare la sequenza di DNA \`e il DNA templating che richiede la separazione dell'elica di DNA in due filamenti stampi e il riconoscimento di ogni
nucleotide nel filamento stampo da parte del nucleotide libero complementarre. La separazione dell'elica espone i gruppi donatori e accettori con legami a idrogeno per ogni base, 
allineandolo per la polimerizzazione catalizzata da enizimi in una nuova catena di DNA. Il primo di tali enzimi \`e detto DNA polimerasi: i nucleotidi liberi che formano il substrato 
sono trifosfati deossiribonucleidici e la loro polimerizzazione in DNA richiede un singolo filamento come stampo. 
\subsection{La forcella di replicazione del DNA \`e asimmetrica}
Durante la replicazione ognuno dei due filamenti originali viene utilizzato come stampo per la formazione di un nuovo filamento. Siccome ognuna delle due cellule figlie eredita una nuova
doppia elica contenente un filamento originale e uno nuovo quest'ultima \`e detta replicata semiconservativamente. Si nota una regione di replicazione localizzata che si muove lungo
la doppia elica parentale. A causa della struttura a forma di Y tale regione \`e detta forcella di replicazione e vi si trova un complesso multienzima che contiene la DNA polimerasi che
sintetizza il DNA per entrambi i filamenti figli. Il meccanismo di replicazione del DNA sembra una continua crescita di entrambi i filamenti, ma a causa dell'orientamento antiparallelo
il meccanismo richiederebbe la polimerizzazione da $5'$-$3'$ per un filamento e da $3'$-$5'$ per l'altro e pertanto due tipi diversi di enzimi DNA polimerasi che in realt\`a riecsce a 
sintetizzare unicamente nella direzione $5'$-$3'$. Il filamento nella direzione opposta viene sintetizzato grazie all'esistenza di segmenti di DNA detti fragmenti di Okazaki ($1000-2000$
basi) transitori che si trovano alla forcella di replicazione. Questi sono polimerizzati unicamente nella direzione $5'$-$3'$ e sono uniti dopo la sintesi per creare lunghe catene di 
DNA. La forcella ha una struttura simetrica e il filamento figlio sintetizzato continuamente \`e detto il filamento principale la cui sintesi dipende leggermente dal filamento in ritardo
la cui sintesi \`e discontinua e ha direzione opposta alla crescita della sintesi del DNA.
\subsection{L'alta fedelt\`a del meccanismo di replicazione del DNA richiede molti meccanismi di correzione}
Le coppie complementari standard non sono le uniche possibili: con piccoli cambi nella geometria dell'elica \`e possibile formare legami tra G e T e esistono rare forme tautomeriche di C
che si accoppiano con A. L'alta fedelt\`a della duplicazione richiede meccanismi di controllo sequenziali che correggono ogni accoppiamento iniziale errato. Il primo passo \`e fatto 
dalla DNA polimerasi quando un nuovo nucleotide \`e aggiunto covalentemente all catena: quello corretto ha un'affinit\`a pi\`u alta con la polimerasi che si muove. Dopo il legame, ma
prima dell'addizione covalente alla catena l'enzima subisce un cambio conformazionale in cui si stringe lungo il sito attivo. Lo stringimento avviene pi\`u facilmente per le basi 
incorrette. La seconda reazione di correzione \`e detta correzione esonucleica avviene quando un nucleotide \`e aggiunto covalentemente alla catena. La DNA polimerasi \`e altamente 
discriminante nelle catene di DNA che allungano: richiedono una base $3'-OH$ acoppiata con un filamento primare. Queste molecole con un nucleotide sbagliato a tale treminazione del
primer non sono efficaci come stampi. Le molecole di DNA polimerasi correggono il primer attraverso un separato sito catalitico (in una diversa subunit\'a o dominio). Questo
$3'$-$5'$ esonucleasi di correzione elimina ogni residuo non accoppiato o malaccoppiato alla terminazione del primer, continuando vino a che abbastanza nucleotidi sono stati rimossi per
generare una terminazione correttamente accoppiata che pu\`o iniziare la sintesi. Queste propriet\`a di autocorrezione della DNA polimerasi dipendono dalla richiesta di un primer a 
terminazione perfettametne accoppiata, cosa non inclusa nell'RNA polimerasi. La frequenza di errore \`e di $1$ ogni $10^4$ eventi di polimerizzazione nella sintesi e traduzione dell'RNA.
\subsection{Solo la replicazione nella direzione $5'$-$3'$ permette una correzione efficiente}
La necessit\`a di accuratezza spiega perch\`e la replicazione del DNA avviene solo nella direzione $5'$-$3'$: se una DNA polimerasi aggiungesse trifosfati deossiribonucleici nella 
direzione opposta la terminazione $5'$ dovrebbe fornire il trifosfato necessario per il legame covalente e gli sbagli non potrebbero essere eliminati attraverso idrolizzazione in quanto
la terminazione $5'$ senza tale gruppo terminerebbe la sintesi. Nonostante tutti questi meccanismi la DNA polimerasi pu\`o creare degli errori, che possono essere indiviudati da un 
processo detto di riparazione direzionata al filo. 
\subsection{Un enzima che polimerizza i nucleotidi sintetizza corti primer di RNA sul filamento in ritardo}
Per il filamento principale \`e necessario un solo primer all'inizio della replicazione, mentre sul filamento in ritardo ogni volta che la DNA polimerasi completa un corto frammento di
Okazaki deve iniziare a sintetizzare un nuovo fragmento a un sito pi\`u in avanti. Un meccanismo produce il primer necessario e dipende dall'enzima detto DNA primasi, che usa trifosfati
ribonucleici per sintetizzare corti primer a RNA sul filamento in ritardo. Si noti come un filamento di RNA pu\`o formare legami con uno di DNA, generando una doppia elica ibrida se 
le sequenze sono complementari e pertanto lo stesso principio di sintesi del DNA guida la sintesi dei primer a RNA. Essendo che un primer a RNA contiene un nucleotide appropriamente
accoppiato con una terminazione $3'-OH$ a una fine pu\`o esser allungato dalla DNA polimerasi a questa fine per formare un frammento di Okazaki, sintesi che finisce quando la
polimerasi incontra il primer a RNA attaccatto alla terminazione $5'$ del semento precendente. Per produrre un filamento continuo di DNA si utilizza un sistema di riparazione del DNA
che elimina i primer a RNA e li sostituisce con DNA. L'enzima DNA ligasi successivamente unisce la terminazione $3'$ del nuovo segmento con quella $5'$ del vecchio. Il primer a RNA 
\`e necessario perm antenere basso il tasso di mutazioni in quanto marca i frammenti come copie sospette. 
\subsection{Proteine speciali aiutano l'apertura della doppia elica sopra la forcella di replicazione}
Per permettere la sintesi la doppia elica deve essere aperta prima della forcella di replicazione in modo che i rifosfati desossiribonucleici possano formare coppie di base con i 
filamenti. Essendo la doppia elica stabile in condizioni fisiologiche sono necessarie la DNA elicasi e un sinogolo filamento di DNA legato a proteine per aprire la doppia elica in tale
ambiente. La DNA elicasi sono state per la prima volta isolate come proteine che idrolizzano l'ATP quando sono legate a un singolo filamento del DNA, reazione che le permette di
spingersi lungo un filamento singolo. Quando incontrano una regone a doppia elica continuano a muoversi lungo il proprio filamento separandola a $1000$ nucleotidi al secondo. Esistono
elicasi che lavorano in entrambe le direzioni della polarit\`a. Proteine che si legano a singoli filamenti di DNA o proteine di destabilizzazione dell'elica si legano strettamente e
cooperativamente per esporre singoli filamenti di DNA senza coprire le basi aiutando l'elicasi stabilizzando la conformazione a singolo filamento e impedendo la formazione di corte 
eliche a forcina nel filamento in ritardo.
\subsection{Un anello che scivola mantiene una DNA polimerasi in movimento sul DNA}
Da sole le molecole di DNA polimerasi sintetizzerebbero una piccola stringa di nucleotidi prima di separarsi dallo stampo. Questa tendenza permette a una polimerasi che ha sintetizzato 
un frammento di Okazaki di separarsi e essere riciclata velocemente ma rende difficile la sintesi di sequenze lunghe se non fosse per una proteina detta PCNA che funziona come un
morsetto scorrevole che mantiene la polimerasi fermamente sul DNA metnre si muove e la rilascia appena incontra una regione a doppia elica. Tale proteina forma un grande anello intorno
alla doppia elica. Una faccia si lega al retro della DNA polimerasi, mentre l'intero anello scorre liberamente lungo il DNA. L'assemblaggio di tale proteina richiede un idrolidi 
dell'ATP da parte di un complesso proteico detto caricatore di morsetto che idrolizza l'ATP mentre carica il morsetto su una giunzione primer. Sul filamento principale la DNA polimerasi
\`e strettamente legata al morsetto e i due rimangono associati per molto tempo. Sul filamento in ritardo ogni volta che la polimerasi arriva alla terminazione $5'$ del frammento di 
Okazaki si rilascia dal morsetto e si dissocia dallo stampo. Questa molecola si associa successivamente con un nuovo morsetto assemblato sul primer a RNA del successivo frammento.
\subsection{Le proteine a una forcella di replicazione cooperano per formare una macchina di replicazione}
Le proteine coinvolte nella replicazione sono ordinate in un complesso multienzima che si mantiene stazionario rispetto all'ambiente con il DNA che si muove al suo interno. All'inizio
della forcella di replicazione la DNA elicasi apre l'elica dove due molecole di DNA polimerasi lavorano sul filamento principale e quello in ritardo, la prima in maniera continua la
seconda in piccoli intervalli. L'associazione stretta di queste proteine aumenta l'efficienza della replicazione ed \`e permessa da un piegamento all'indietro del filamento in 
ritardo. Questo ordinamento facilita il caricamento del morsetto polimerasi ogni volta che un frammento di Okazaki \`e sintetizzato. Le proteine di replicazione sono legate insieme in
in una singola grande unit\`a ($>10^6$ dalton) permettendo una sintesi efficiente e coordinata. 
\subsection{Un sistema di riparazione direzionato al filamento rimuove gli errori di replicazione che sfuggono alla macchina di replicazione}
Una classe di mutanti possiede alterazioni nei geni mutatori che aumentano il tasso di mutazioni spontanee: alcuni di essi possiedono forme difettive di esonucleasi di correzione. Questo
studio ha scoperto un meccanismo che rimuove errori di replicazioni e sfuggiti all'esonucleasi detto sistema di strand-directed mismatch repair che individua potenziali distorsioni 
nell'elica del DNA e corregge uno dei due nucleotidi scorretti: per essere corretto deve essere in grado di distinguere e rimuovere gli error solo nei nuovi filamenti. Nei procarioti
viene utilizzato un meccanismo di distinzione dei filamenti che dipende dalla metilazione di residui di A nel GATC gruppi metile sono aggiunti in utti i residui nella sequenza GATC, ma 
non fino a che \`e passato abbastanza tempo: pertanto le uniche sequenze GATC non metilate si trovano unicamente nel filamento dietro la forcella di replicazione. Il riconoscimento di 
GATC non metilati permette a nuovi filamenti di DNA di essere distinti dai vecchi. Il processo per questa correzione coinvolge pertanto il riconoscimento del filamento, l'escissione dell
a porzione che contiene la mancata corrispondenza e la risintesi del segmento escisso con il vecchio stampo. Questo sistema riduce il numero di errori di un fattore tra $100$ e $1000$. 
Nelle cellule eucariotiche viene utilizzato un altro metodo per riconoscere il nuovo filamento: i nuovi filamenti in ritardo contengono transientemente nicks o single-strand breaks che
forniscono il segnale che direziona il sistema di correzione. Questa strategia richiede che anche il filamento principale sia nicked. 
\subsection{La DNA topoisomerasi impedisce l'ingarbugliamento del DNA durante la replicazione}
Mentre la forcella di replicazione si muove lungo il DNA crea il problema dell'avvolgimento: due filamenti genitori devono essere srotolati in modo da permettere la replicazione. In 
principio questo srotolamento pu\`o essere ottenuto rapidemente ruotando il cromosoma lungo la forcella ma \`e sfavorevole per cromosomi lunghi, pertnato solo il DNA davanti alla 
forcella viene srotolato e il sovraarrotolamento \`e rilassato dalla DNA topoisomerasi, associabile a una nucleasi reversibile che si aggiunge al backbone fosfato rompendo i legami
fosfodiersteri nel filamento che si riforma quando la proteina va via. La topoisomerasi 1 produce una rottura a filamento singolo transiente che permette a due sezioni dell'elica del DNA
tra le due parti del nick di ruotare liberamente tra di loro utilizzano i legami fosfodiesteri come pivot. Tensione nell'elica guida la rotazione nella direzione che la elimina: la
replicazione avviene con la rotazione di piccole sezioni della sequenza. Siccome il legame con la proteina mantiene l'energia del legame fosfodiestere la ricreazione di esso \`e rapida
e non richiede energia esterna. La topoisomerasi II forma un legame covalente con entrambi i filamenti del DNA, creando una rottura in entrambi transiente. Questi enzimi sono attivati 
dai siti sul cromosoma dove le doppie eliche si incrociano come quelle generate da un sovraavvolgimento in una frocella di replicazione. Quando una molecola di topoisomerasi II si lega 
in un sito di incrocio usa l'idrolisi dell'ATP per rompere una doppia elica reversibilmente creando un gate di DNA, causa la seconda vicina doppia elica di passare nell'apertura e 
successivamente riunisce l'elica rotta e si dissocia dal DNA. Il passaggio della doppia elica nel gate avviene nella direzione che causa l'eliminazione del sovraavvolgimento. Pu\`o 
anche separare due cicli di DNA interbloccati. 
\subsection{La replicazione del DNA \`e fondalmente simile in eucarioti e batteri}
Molto di quello che \`e conosciuto dell'enzimologia della replicazione del DNA negli eucarioti  \`e stat conservata durante il processo evolutivo che ha separato batteri ed eucarioti. 
Pur essendoci pi\u proteine per i secondi le funzioni di base sono le stesse. 
\section{Inizializzazione e completamento della replicazione del DNA nei cromosomi}

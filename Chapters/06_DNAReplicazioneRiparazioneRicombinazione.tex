\chapter{Replicazione, riparazione e ricombinazione del DNA}
L'abilit\`a della cellula di mantenere un alto grado di ordine dipende dall'accurata duplicazione di grandi quantit\`a di informazioni genetiche trasportate in forma chimica come DNA.
Il processo della replicazione del DNA deve avvenire prima che una cellula possa produrre due cellule figlie geneticamente indipendenti. Il mantenimento dell'ordine richiede anche una
continua sorveglianza e riparaizone dell'informazione genetica, continaumente danneggiata dall'ambiente chimicamente, attraverso radiazioni, calore e molecole attive generate nella 
cellula. Questi processi vengono svolti da proteine che catalizzano processi rapidi e accurati che avvengono all'interno della cellula. Se la sopravvivenza a breve termine di una cellula
dipende dalla prevenzione dei cambi nella sequenza, quella a lungo termine richiede che una sequenza di DNA cambi lungo le generazioni in modo da permettere adattamento evolutivo a un
ambiente dinamico. Nonostante gli sforzi della cellula avvengono dei cambi nel DNA che possono mettere a disposizioni varianti che la selelzione spinge durante l'evoluzione.
\section{Il mantenimento della sequenza di DNA}
La sopravviveza dell'individuo richiede un alto grado di stabilit\`a genetica. Solo raramente il processo di mantenimento del DNA fallisce causando cambi permenenti nel DNA o mutazioni, 
che possono distruggere un organismo se avvengono se in posizioni vitali della sequenza del DNA.
\subsection{I tassi di mutazione sono estremamenti bassi}
Il tasso di mutaziine pu\`o essere derminato direttamente attraverso esperimenti con batteri come l'Escherichia coli che si divide ogni $30$ minuti e una cellula singola pu\`o generare
una grande popolazione in meno di un giorno. In tale di popolazione \`e possibile individuare una piccola frazione di batteri con una mutazione dannosa in un gene particolare se on
\`e necessario alla sopravvivenza. Tale frazione \`e una sottostima delle mutazioni in quanto ne esistono silenti. Dopo aver corretto per queste mutazioni silenti si trova che un singolo
gene per una proteina di dimensione media ($10^3$ paia di nucleotidi) accumla una mutazione una volta ogni $10^6$ generazioni: il tasso di mutazione \`e pertanto di tre cambi 
nucleotidici per $10^{10}$ nucleotidi per generazione di cellule. Recentemente \`e stato possbile misurare il tasso di mutazione direttamente in organismi pi\`u complessi e la 
riproduzione sessuale. In questo caso la sequenza genomica di una famiglia si sequenzia e una comparazione determina circa $70$ mutazionidi singoli nucleotidi in ogni discendente. 
Normalizzando alla dimensione del genoma umano il tasso di mutazione di un nucleotide cambia per $10^8$ nucleotidi per generazione. Questa \`e una sottostima in quanto non considera le 
mutazioni letali che non sarebbero presenti nella progenie. Circa $100$ divisioni accadono tra il concepimento e il tempo di produzione di uova e sperma che producono una nuova 
generazione, pertanto il tasso umano di mutazione \`e di $1$ mutazione per $10^{10}$ divisioni cellulari. In entrambi gli esperimenti si nota come i tassi di mutazioni sono estremamente
bassi e con un fattore di tre l'uno dall'altro: sono infatti preservati i meccanismi base che garantiscono questi tassi bassi e sono stati conservati da cellule ancestrali molto antiche.
\subsection{I tassi di mutazione bassi sono necessari per la vita come la conosciamo}
Essendo che la maggior parte delle mutazioni sono deleterie, nessuna specie pu\`o permettersi di accumularle. Pur essendo il tasso di mutazioni basso si pensa limiti il numero di 
proteine che pu\`o dipendere a $30^.000$ in quanto per un numero maggiore la proabilit\`a che una componente chiave sia danneggiata da una mutazione diventa troppo elevata. Le cellule di
un organismo che si riproduce sessualmente sono di due tipi: le cellule germinali e quelle somatiche. Le prime trasmettano l'informazione tra genitore e prole, mentre le seconde formano
il corpo dell'organismo. Le cellule germinali devono esssere protette contro alti tassi di mutazione per mantenere la specie, cosa che deve avvenire anche nelle cellule somatiche per 
la corretta formazione di un corpo organizzato. I cambi nucleotidici nelle cellule somatiche possono creare cellule varianti, alcune delle quali attraverso selezione naturale locale
proliferano rapidamente. Nei casi estremi si genera un cancro, la cui probabilit\`a aumenta linearmente con il tasso di mutazione. Pertanto sia la perpetuazione di una specie con un 
gran numero di geni e la prevenzione di cancri che risultano da mutazioni delle cellule somatiche dipendono dall'alta fedelt\`a con cui le sequenze di DNA sono replicate e mantenute. 
\section{Meccanismi di replicazione del DNA}
Tutti gli organismi duplicano il proprio DNA con estrema accuratezza prima di ogni divisione cellulare.
\subsection{L'accoppiemaneto delle basi sottost\`a la replicazione e riparazione del DNA}
Il meccanismo che usa la cellula per copiare la sequenza di DNA \`e il DNA templating che richiede la separazione dell'elica di DNA in due filamenti stampi e il riconoscimento di ogni
nucleotide nel filamento stampo da parte del nucleotide libero complementarre. La separazione dell'elica espone i gruppi donatori e accettori con legami a idrogeno per ogni base, 
allineandolo per la polimerizzazione catalizzata da enizimi in una nuova catena di DNA. Il primo di tali enzimi \`e detto DNA polimerasi: i nucleotidi liberi che formano il substrato 
sono trifosfati deossiribonucleidici e la loro polimerizzazione in DNA richiede un singolo filamento come stampo. 
\subsection{La forcella di replicazione del DNA \`e asimmetrica}
Durante la replicazione ognuno dei due filamenti originali viene utilizzato come stampo per la formazione di un nuovo filamento. Siccome ognuna delle due cellule figlie eredita una nuova
doppia elica contenente un filamento originale e uno nuovo quest'ultima \`e detta replicata semiconservativamente. Si nota una regione di replicazione localizzata che si muove lungo
la doppia elica parentale. A causa della struttura a forma di Y tale regione \`e detta forcella di replicazione e vi si trova un complesso multienzima che contiene la DNA polimerasi che
sintetizza il DNA per entrambi i filamenti figli. Il meccanismo di replicazione del DNA sembra una continua crescita di entrambi i filamenti, ma a causa dell'orientamento antiparallelo
il meccanismo richiederebbe la polimerizzazione da $5'$-$3'$ per un filamento e da $3'$-$5'$ per l'altro e pertanto due tipi diversi di enzimi DNA polimerasi che in realt\`a riecsce a 
sintetizzare unicamente nella direzione $5'$-$3'$. Il filamento nella direzione opposta viene sintetizzato grazie all'esistenza di segmenti di DNA detti fragmenti di Okazaki ($1000-2000$
basi) transitori che si trovano alla forcella di replicazione. Questi sono polimerizzati unicamente nella direzione $5'$-$3'$ e sono uniti dopo la sintesi per creare lunghe catene di 
DNA. La forcella ha una struttura simetrica e il filamento figlio sintetizzato continuamente \`e detto il filamento principale la cui sintesi dipende leggermente dal filamento in ritardo
la cui sintesi \`e discontinua e ha direzione opposta alla crescita della sintesi del DNA.
\subsection{L'alta fedelt\`a del meccanismo di replicazione del DNA richiede molti meccanismi di correzione}
Le coppie complementari standard non sono le uniche possibili: con piccoli cambi nella geometria dell'elica \`e possibile formare legami tra G e T e esistono rare forme tautomeriche di C
che si accoppiano con A. L'alta fedelt\`a della duplicazione richiede meccanismi di controllo sequenziali che correggono ogni accoppiamento iniziale errato. Il primo passo \`e fatto 
dalla DNA polimerasi quando un nuovo nucleotide \`e aggiunto covalentemente all catena: quello corretto ha un'affinit\`a pi\`u alta con la polimerasi che si muove. Dopo il legame, ma
prima dell'addizione covalente alla catena l'enzima subisce un cambio conformazionale in cui si stringe lungo il sito attivo. Lo stringimento avviene pi\`u facilmente per le basi 
incorrette. La seconda reazione di correzione \`e detta correzione esonucleica avviene quando un nucleotide \`e aggiunto covalentemente alla catena. La DNA polimerasi \`e altamente 
discriminante nelle catene di DNA che allungano: richiedono una base $3'-OH$ acoppiata con un filamento primare. Queste molecole con un nucleotide sbagliato a tale treminazione del
primer non sono efficaci come stampi. Le molecole di DNA polimerasi correggono il primer attraverso un separato sito catalitico (in una diversa subunit\'a o dominio). Questo
$3'$-$5'$ esonucleasi di correzione elimina ogni residuo non accoppiato o malaccoppiato alla terminazione del primer, continuando vino a che abbastanza nucleotidi sono stati rimossi per
generare una terminazione correttamente accoppiata che pu\`o iniziare la sintesi. Queste propriet\`a di autocorrezione della DNA polimerasi dipendono dalla richiesta di un primer a 
terminazione perfettametne accoppiata, cosa non inclusa nell'RNA polimerasi. La frequenza di errore \`e di $1$ ogni $10^4$ eventi di polimerizzazione nella sintesi e traduzione dell'RNA.
\subsection{Solo la replicazione nella direzione $5'$-$3'$ permette una correzione efficiente}
La necessit\`a di accuratezza spiega perch\`e la replicazione del DNA avviene solo nella direzione $5'$-$3'$: se una DNA polimerasi aggiungesse trifosfati deossiribonucleici nella 
direzione opposta la terminazione $5'$ dovrebbe fornire il trifosfato necessario per il legame covalente e gli sbagli non potrebbero essere eliminati attraverso idrolizzazione in quanto
la terminazione $5'$ senza tale gruppo terminerebbe la sintesi. Nonostante tutti questi meccanismi la DNA polimerasi pu\`o creare degli errori, che possono essere indiviudati da un 
processo detto di riparazione direzionata al filo. 
\subsection{Un enzima che polimerizza i nucleotidi sintetizza corti primer di RNA sul filamento in ritardo}
Per il filamento principale \`e necessario un solo primer all'inizio della replicazione, mentre sul filamento in ritardo ogni volta che la DNA polimerasi completa un corto frammento di
Okazaki deve iniziare a sintetizzare un nuovo fragmento a un sito pi\`u in avanti. Un meccanismo produce il primer necessario e dipende dall'enzima detto DNA primasi, che usa trifosfati
ribonucleici per sintetizzare corti primer a RNA sul filamento in ritardo. Si noti come un filamento di RNA pu\`o formare legami con uno di DNA, generando una doppia elica ibrida se 
le sequenze sono complementari e pertanto lo stesso principio di sintesi del DNA guida la sintesi dei primer a RNA. Essendo che un primer a RNA contiene un nucleotide appropriamente
accoppiato con una terminazione $3'-OH$ a una fine pu\`o esser allungato dalla DNA polimerasi a questa fine per formare un frammento di Okazaki, sintesi che finisce quando la
polimerasi incontra il primer a RNA attaccatto alla terminazione $5'$ del semento precendente. Per produrre un filamento continuo di DNA si utilizza un sistema di riparazione del DNA
che elimina i primer a RNA e li sostituisce con DNA. L'enzima DNA ligasi successivamente unisce la terminazione $3'$ del nuovo segmento con quella $5'$ del vecchio. Il primer a RNA 
\`e necessario perm antenere basso il tasso di mutazioni in quanto marca i frammenti come copie sospette. 
\subsection{Proteine speciali aiutano l'apertura della doppia elica sopra la forcella di replicazione}
Per permettere la sintesi la doppia elica deve essere aperta prima della forcella di replicazione in modo che i rifosfati desossiribonucleici possano formare coppie di base con i 
filamenti. Essendo la doppia elica stabile in condizioni fisiologiche sono necessarie la DNA elicasi e un sinogolo filamento di DNA legato a proteine per aprire la doppia elica in tale
ambiente. La DNA elicasi sono state per la prima volta isolate come proteine che idrolizzano l'ATP quando sono legate a un singolo filamento del DNA, reazione che le permette di
spingersi lungo un filamento singolo. Quando incontrano una regone a doppia elica continuano a muoversi lungo il proprio filamento separandola a $1000$ nucleotidi al secondo. Esistono
elicasi che lavorano in entrambe le direzioni della polarit\`a. Proteine che si legano a singoli filamenti di DNA o proteine di destabilizzazione dell'elica si legano strettamente e
cooperativamente per esporre singoli filamenti di DNA senza coprire le basi aiutando l'elicasi stabilizzando la conformazione a singolo filamento e impedendo la formazione di corte 
eliche a forcina nel filamento in ritardo.
\subsection{Un anello che scivola mantiene una DNA polimerasi in movimento sul DNA}
Da sole le molecole di DNA polimerasi sintetizzerebbero una piccola stringa di nucleotidi prima di separarsi dallo stampo. Questa tendenza permette a una polimerasi che ha sintetizzato 
un frammento di Okazaki di separarsi e essere riciclata velocemente ma rende difficile la sintesi di sequenze lunghe se non fosse per una proteina detta PCNA che funziona come un
morsetto scorrevole che mantiene la polimerasi fermamente sul DNA metnre si muove e la rilascia appena incontra una regione a doppia elica. Tale proteina forma un grande anello intorno
alla doppia elica. Una faccia si lega al retro della DNA polimerasi, mentre l'intero anello scorre liberamente lungo il DNA. L'assemblaggio di tale proteina richiede un idrolidi 
dell'ATP da parte di un complesso proteico detto caricatore di morsetto che idrolizza l'ATP mentre carica il morsetto su una giunzione primer. Sul filamento principale la DNA polimerasi
\`e strettamente legata al morsetto e i due rimangono associati per molto tempo. Sul filamento in ritardo ogni volta che la polimerasi arriva alla terminazione $5'$ del frammento di 
Okazaki si rilascia dal morsetto e si dissocia dallo stampo. Questa molecola si associa successivamente con un nuovo morsetto assemblato sul primer a RNA del successivo frammento.
\subsection{Le proteine a una forcella di replicazione cooperano per formare una macchina di replicazione}
Le proteine coinvolte nella replicazione sono ordinate in un complesso multienzima che si mantiene stazionario rispetto all'ambiente con il DNA che si muove al suo interno. All'inizio
della forcella di replicazione la DNA elicasi apre l'elica dove due molecole di DNA polimerasi lavorano sul filamento principale e quello in ritardo, la prima in maniera continua la
seconda in piccoli intervalli. L'associazione stretta di queste proteine aumenta l'efficienza della replicazione ed \`e permessa da un piegamento all'indietro del filamento in 
ritardo. Questo ordinamento facilita il caricamento del morsetto polimerasi ogni volta che un frammento di Okazaki \`e sintetizzato. Le proteine di replicazione sono legate insieme in
in una singola grande unit\`a ($>10^6$ dalton) permettendo una sintesi efficiente e coordinata. 
\subsection{Un sistema di riparazione direzionato al filamento rimuove gli errori di replicazione che sfuggono alla macchina di replicazione}
Una classe di mutanti possiede alterazioni nei geni mutatori che aumentano il tasso di mutazioni spontanee: alcuni di essi possiedono forme difettive di esonucleasi di correzione. Questo
studio ha scoperto un meccanismo che rimuove errori di replicazioni e sfuggiti all'esonucleasi detto sistema di strand-directed mismatch repair che individua potenziali distorsioni 
nell'elica del DNA e corregge uno dei due nucleotidi scorretti: per essere corretto deve essere in grado di distinguere e rimuovere gli error solo nei nuovi filamenti. Nei procarioti
viene utilizzato un meccanismo di distinzione dei filamenti che dipende dalla metilazione di residui di A nel GATC gruppi metile sono aggiunti in utti i residui nella sequenza GATC, ma 
non fino a che \`e passato abbastanza tempo: pertanto le uniche sequenze GATC non metilate si trovano unicamente nel filamento dietro la forcella di replicazione. Il riconoscimento di 
GATC non metilati permette a nuovi filamenti di DNA di essere distinti dai vecchi. Il processo per questa correzione coinvolge pertanto il riconoscimento del filamento, l'escissione dell
a porzione che contiene la mancata corrispondenza e la risintesi del segmento escisso con il vecchio stampo. Questo sistema riduce il numero di errori di un fattore tra $100$ e $1000$. 
Nelle cellule eucariotiche viene utilizzato un altro metodo per riconoscere il nuovo filamento: i nuovi filamenti in ritardo contengono transientemente nicks o single-strand breaks che
forniscono il segnale che direziona il sistema di correzione. Questa strategia richiede che anche il filamento principale sia nicked. 
\subsection{La DNA topoisomerasi impedisce l'ingarbugliamento del DNA durante la replicazione}
Mentre la forcella di replicazione si muove lungo il DNA crea il problema dell'avvolgimento: due filamenti genitori devono essere srotolati in modo da permettere la replicazione. In 
principio questo srotolamento pu\`o essere ottenuto rapidemente ruotando il cromosoma lungo la forcella ma \`e sfavorevole per cromosomi lunghi, pertnato solo il DNA davanti alla 
forcella viene srotolato e il sovraarrotolamento \`e rilassato dalla DNA topoisomerasi, associabile a una nucleasi reversibile che si aggiunge al backbone fosfato rompendo i legami
fosfodiersteri nel filamento che si riforma quando la proteina va via. La topoisomerasi 1 produce una rottura a filamento singolo transiente che permette a due sezioni dell'elica del DNA
tra le due parti del nick di ruotare liberamente tra di loro utilizzano i legami fosfodiesteri come pivot. Tensione nell'elica guida la rotazione nella direzione che la elimina: la
replicazione avviene con la rotazione di piccole sezioni della sequenza. Siccome il legame con la proteina mantiene l'energia del legame fosfodiestere la ricreazione di esso \`e rapida
e non richiede energia esterna. La topoisomerasi II forma un legame covalente con entrambi i filamenti del DNA, creando una rottura in entrambi transiente. Questi enzimi sono attivati 
dai siti sul cromosoma dove le doppie eliche si incrociano come quelle generate da un sovraavvolgimento in una frocella di replicazione. Quando una molecola di topoisomerasi II si lega 
in un sito di incrocio usa l'idrolisi dell'ATP per rompere una doppia elica reversibilmente creando un gate di DNA, causa la seconda vicina doppia elica di passare nell'apertura e 
successivamente riunisce l'elica rotta e si dissocia dal DNA. Il passaggio della doppia elica nel gate avviene nella direzione che causa l'eliminazione del sovraavvolgimento. Pu\`o 
anche separare due cicli di DNA interbloccati. 
\subsection{La replicazione del DNA \`e fondalmente simile in eucarioti e batteri}
Molto di quello che \`e conosciuto dell'enzimologia della replicazione del DNA negli eucarioti  \`e stat conservata durante il processo evolutivo che ha separato batteri ed eucarioti. 
Pur essendoci pi\u proteine per i secondi le funzioni di base sono le stesse. 
\section{Inizializzazione e completamento della replicazione del DNA nei cromosomi}
\subsection{La sintesi del DNA inizia alle origini di replicazione}
Essendo la doppia elica stabile per iniziare la replicazione deve essere aperta e i due filamenti separati, processo svolto da proteine iniziatrici che si legano alla doppia elica e 
separano i due filamenti rompendo i corrispettivi legami a idrogeno. Queste regioni in cui la doppia elica viene aperta sono dette origini di replicazione, specificate da seqenze di 
nucleotidi che attraggono proteine iniziatrici e altre lunghezze facili da aprire: A e T sono mantenute insieme da meno legami a idrogeno rispetto a C e G. Il processo alla base \`e lo
stesso per batteri ed eucarioti, ma differisce nel modo in cui il processo avviene e come \`e regolato. 
\subsection{I cromosomi batterici hanno un origine singola per la replicazione}
Il genoma dell'E. coli \`e contenuto in una molecola di DNA circolare. La replicazione inizia ad un singolo sito e le due forcelle di replicazione procedono in direzioni opposte fino
a che si incontrano. L'unico punto in cui la replicazione pu\`o essere controllata \`e l'inizio. Il processo inizia quando le proteine iniziatrici legate a ATP si legano in copie 
multiple in siti specifici del DNA avvolgendolo formando un complesso DNA-proteine che destabilizza la doppia elica vicina. Questo complesso attrae due DNA elicasi, legate a un 
caricatore elicasi, analogo al caricatore di morsetti ma che mantiene l'elicasi in una forma inattiva fino a che \`e propiamente caricata sulla forcella di replicazione nascente. Una
volta che \`e caricata si dissocia e l'elicasi inizia a svolgere il DNA esponendo i filamenti in modo che la DNA primasi possa sintetizzare il primo primer a RNA che porta all'
assemblaggio delle proteine necessarie per creare due forcelle di replicazione che continuano a sintetizzare il DNA fino a che tutto lo stampo \`e stato replicato. L'interazione tra le
proteine iniziatrici con l'origine di replicazione \`e regolata, permettendo l'inizio del proceso solo se ci sono nutrienti sufficienti per completarlo e che solo una replicazione 
avvenga per divisione cellulare. Dopo che la replicazione \`e iniziata la proteina iniziatrice \`e disattivata dall'idrolisi dell'ATP e l'origine di replicazione ha un periodo 
refrattario causato da un ritardo della metilazione dei nuovi nucleotidi A nell'origine. L'iniziazione non pu\`o accadere fino a che le A sono metilate e la proteina iniziatrice \`e 
riportata allo stato legato all'ATP. 
\subsection{I cromosomi eucariotici contengono multiple origini di replicazione}
Le forcelle di replicazione eucariotiche si muovono di circa $50$ nucleotidi al secondo a causa dello stato pi\`u denso della cromatina. Si \`e notato come sono presenti da $30^.000$ a 
$500^.000$ origini di replicazione per la divisione di una cellula umana in modo che una cellula possa coordinare le regioni attive con altre caratteristiche dei cromosomi come i geni
che sono espressi. Le forcelle di replicazione sono formate in coppie e creano una bolla di replicazione che si muove in  direzioni opposte da un punto di origine comune, che si
fermano quando collidono con un'altra forcella che si muove nella direzione opposta. Molte forcelle di replicazione operano indipendentemente su ogni cromosoma e formano due eliche di 
DNA figlie complete. 
\subsection{Negli eucarioti la replicazione del DNA avviene durante un'unica parte del ciclo vitale della cellula}
Se i batteri replicano il DNA quasi continuamente negli eucarioti questo avviene solo durante la fase S o di sintesi del DNA che in una cellula mammifera dura circa $8$ ore, al termine
delle quali ogni cromosoma \`e stato completamente replicato in due coppie che rimandono unite al centromero fino alla fase M. 
\subsection{Diverse regioni dello stesso cromosoma si replicano a tempi distinti durante la fase S}
Le orgini di replicazione non sono tutti attivate allo stesso momento, ma in cluster di circa $50$ adiacenti, ognuna delle quali \`e replicata durante una piccola parte della fase S.
L'ordine di attivazione dipende in parte dalla struttura cromatinica: l'eterocromatina nell'ultima parte della fase in quanto il ritardo potrebbe essere in relazione con la 
decondensazione della cromatina. Le forcelle di replicazione si muovono ad una velocit\`a costante durante la fase S.
\subsection{Un grande complesso multisubunit\`a si lega alle origini di replicazione eucariotiche}
La maggior parte delle sequenze di DNA che sevono come origini di replicazione contengono un sito di legame per una proteina iniziatrice formata da molte subunit\`a detta ORC (origin
recognition complex), una lunghezza di DNA ricca in A e T e almeno un sito di legame per proteine che facilitano il legame dell'ORC modificando la struttura cromatinica. Per garantire
che tutto il DNA sia copiato una e una sola volta le elicasi replicative vengono caricate sequenzialmente sulle origini e attivate per iniziare la replicazione del DNA. Durante la fase
$G_1$ le elicasi replicative sono caricate sul DNA vicino all'ORC per creare un complesso prereplicativo. Successivamente proteine chinasi specializzate attivano l'elicasi. L'apertura
della doppia elica permette il caricamento delle proteine di replicazione rimanenti come la DNA polimerasi. La chinasi che inizia la replicazione previene l'assemblaggio di nuovi 
complessi prereplicativi fino alla fase M successiva che risetta l'intero ciclo. Lo fanno fosforilando l'ORC rendendolo incapace di accettare nuove elicasi. Pertanto il complesso 
prereplicativo pu\`o formarsi unicamente nella fase $G_1$, mentre pu\`o essere attivato e disassemblato unicamente nella fase $S$, due fasi mutualmente esclusive. 
\subsection{Nuovi nucleosomi sno assemblati in coda alla forcella di replicazione}
La duplicazione dei cromosomi richiede la sintesi e assemblaggio di nuove proteine cromosomiche sul DNA in coda alla forcella. La cellula richiede un gran numero di nuove proteine 
istoni in massa uguale al nuovo DNA sintetizzato per creare nuovi nucleosomi in ogni ciclo della cellula, pertanto gli organismi possiedono multiple copie del gene per ogni istone. Gli
istoni sono sintetizzati principalmente nella fase S, quando il livello di mRNA istone aumenta di $50$ volte e degradati alla fine della fase. Le proteine istone invece sopravvivono per
l'intera vita della cellula. Il collegamento tra sintesi del DNA e degli istoni riflette un meccanismo di feedback che monitora il livello degli istoni liberi per garantire che le 
quantit\`a siano uguali. Quando la forcella di replicazione avanza e passa attraverso nucleosomi del genitore e pertanto \`e richiesto un complesso di rimodellamento della cromatina
che destabilizza le interfaccie DNA-istone in modo che la forcella possa superarle efficientemente. Mentre la forcella di replicazione passa attraverso la cromatina gli istoni sono
spostati. Quando un nucleosoma viene attraversato l'ottamero istonico viene rotto in un tetramero H3-H4 e due dimeri H2A-H2B. Il primo rimane associato debolmente con il DNA ed \`e
distribuito a caso tra uno dei due figli, mentre i dimeri sono completamente rilasciati. I tetrameri H3-H4 recentemente sintetizzati sono aggiunti per riempire i buchi e i dimeri, met\`a
vecchi e met\`a nuovi sono aggiunti a caso per completare il nucleosoma. La lunghezza dei frammenti di Okazaki \`e determinata dal punto in cui la DNA polimerasi \`e bloccata da un 
nuovo nucleosoma. Pertanto i frammenti hanno la stessa lunghezza della lunghezza di ripetizione dei nucleosomi. L'addizione di tetrameri e dimeri richiede accompagnatori istoni o 
fabbriche di assemblaggio di cromatina, complessi a multisubunit\`a che si legano agli istoni altamente basici e li rilasciano solo nel contesto appropriato e diretti verso il DNA 
appena replicato attraverso un'interazione specifica con il morsetto PCNA che rimangono sul DNA abbastanza a lungo per permettere ai primi di completare il loro compito.
\subsubsection{la telomerasi remplica la fine dei cromosomi}
Il meccanismo di replicazione del filamento in ritardo incontra problemi alla fine del cromosoma lineare: l'RNA primer finale non pu\`o essere sostituito dal DNA in quanto non c'\`e una
fine $3'-OH$ disponibile per la polimerasi di riparazione. I batteri risolvono il problema con DNA circolare, mentre gli eucarioti attraverso sequenze specializzate dette telomeri
che contengono molte ripetizioni di sequenze corte (negli umani GGGTTA), riconosciute dall'enzima telomerasi che le rifornisce ogni volta che la cellula si divida. La telomerasi 
riconosce la vine di un telomero e la allunga nella riezione $5'$-$3'$ utilizzando uno stampo a RNA componente dell'enzima stesso per sintetizzare nuove copie della ripetizione. Dopo
l'estensione del filo genitore della telomerasi la replicazione del filamento in ritardo pu\`o essere completata dalla DNA polimerasi standard che usa queste estensioni per sintetizzare
il filamento complementare. 
\subsection{I telomeri sono condensati in strutture specializzate che proteggono la fine dei cromosomi}
I telomeri devono essere in grado di distinguere tra le rotture accidentali del cromosoma: una nucleasi specializzate si attacca alla fine $5'$ dei cromosomi lasciando protrudere un 
singolo filamento che in combinazione con le ripetizioni di GGGTTA attrae un gruppo di proteine che formano un cappuccio detto shelterin che nasconde i telomeri dai rivelatori di 
danni che monitorano il DNA: la fine protrundente del DNA si infila nella sequenza ripetuta del telomero. Questi anelli a T sono regolati dal shelterin e proteggono ulteriormente la 
fine del cromosoma. 
\subsection{La lunghezza dei telomeri \`e regolata dalla cellula e dagli organismi}
Essendo i processi che fanno crescere e riducono ogni sequenza di telomeri sono bilanciati approssimativamente una fine cromosomica contiene un numero variabile di ripetizioni 
telomeriche. Molte cellule hanno un numero di meccanismi omeostatici che mantengono il numero di queste ripetizioni in intervallo. Nella maggior parte delle divisioni cellulari i 
telomeri si accorciano gradualmente per limitare la proliferazione di cellule ribelli nei tessuti adulti. Le ripetizioni telomeriche sono erose in quantit\`a diverse in tipi diversi di 
cellule grazie a un enzima che non pu\`o tenere il passo con la duplicazione. Dopo molte generazioni le cellule discendenti avranno cromosomi senza funzione telomerica e attivano una
risposta ad danni del DNA che causa la fine del ciclo cellulare e della duplicazione in un processo detto replicative cell senescence. 
\section{Riparazione del DNA}
Mantenere la stabilit\`a genetica per la vita richiede meccanismi per riparare le lesioni accidentali che continuano ad avvenire. La maggior parte dei cambi nel DNA sono immediatamente
corretti da un insieme di proteine dette DNA repair. Solo lo $0.02\%$ dei cambi del DNA si accumula come permutazioni permanenti. Una grande parte della capacit\`a di codifica dei genomi
\`e utilizzata per queste proteine. 
\subsection{Senza la riparazione del DNA, danni spontanei cambierebbero rapidamente la sequenza del DNA}
Nonostante il DNA sia un materiale altamente stabile \`e una molecola organica complessa suscettibile a cambi spontanei che porterebbero a mutazioni se lasciate non riparate. Il DNA 
della cellula umana perde circa $18^.000$ basi purine ogni giorno a causa dei legami N-glicosilici al deossiribosio si idrolizzano nella depurinazione. Similmente una deamminazione 
dela citosina nell'uracile accade a circa $100$ basi per cellula al giorno. Le basi del DNA possono essere danneggiate da una collisione con metaboliti reattivi prodotti nella cellula
come forme reattive di ossigeno o il donatore di metile S-adenosilmetionine o da radiazioni ultraviolette. Se lasciate non corrette queste modifiche porterebbero alla cancellazione di
basi o a sostituzioni durante la replicazione con conseguenze disastrose.
\subsection{La doppia elica del DNA \`e prontamente riparata}
La struttura a doppia elica del DNA \`e adatta alla riparazione in quanto porta due copie dell'informazione genetica, pertanto quando un filamento \`e danneggiato quello complementare
mantiene una copia intatta della stessa informazione che viene utilizzata durante la riparazione. I tipi di processi di riparazione presentati non possono operare su DNA o RNA a singolo
filamento. 
\subsection{Danni al DNA possono essere rimossi attraverso molti cammini}
Le cellule hanno diversi modi per riparare il DNA utilizzando diversi enzimi che agiscono in base alla lesione. Nei due cammini pi\`u comuni il danno \`e asportato e la sequenza 
originale \`e ripristinata dalla DNA polimerasi utilizzando il filamento non danneggiato come stampo. Differiscono riguardo a come eliminano il danno. Nel primo cammino detto riparazione
tramite asportazione della base coinvolge un insieme di enzimi detti DNA glicolasi che possono riconoscere un tipo specifico di base nel DNA e catalizzare la rimozione idrolitica. Ne
esistono al meno sei tipi, come quelli che rimuovono C e A deaminate, diversi tipi di basi alcalinate o ossidate, basi con anelli aperti e quelle in cui un doppio legame 
carbonio-carbonio \`e stato convertito in un legame singolo. Una base alterata viene riconosciuta in un processo con un passo base in cui avviene un flip-out del nucleotide danneggiato
dall'elica mediato da un enzima che permette alla DNA glicolasi di controllare tutte le parti della base per danni. Questi enzimi si muovono lungo tutto il DNA alla ricerca di danni. 
Quando viene trovato una base danneggiata la rimuove dallo zucchero. Il buco creato dalla DNA glicolasi \`e riconosciuto da un enzima detto AP endonucleasi che taglia il backbone a 
fosfodiestere e il vuoto lasciato \`e riparato. La depurinazione lascia uno zucchero con una base mancante e sono pertanto riparate direttamente con un AP endonucleasi. Il secondo 
cammino principale \`e detto riparazione a asportazione del nucleotida. Questo meccanismo pu\`o riparare danni causati da grandi cambi nella struttura della doppia elica. Un grande
complesso multienzima scansiona il DNA alla ricerca di una distorione della doppia elica e la DNA elicasi rimuove il singolo filamento contenente la lesione. Il grande gap prodotto \`e
riparato dalla DNA polimerasi e dalla DNA ligasi. Un'alternativa a questi meccanismi \`e la reversione chimica diretta, utilizzata selettivamente per la rapida rimozione di lesioni
altamente mutagenice o citotossiche.
\subsection{L'accoppiamento di riparazione a asportazione di nucleotidi con la trascrizione garantisce che il DNA pi\`u importante \`e riparato efficientemente}
Tutto il DNA \`e sotto costante sorveglianza per i danni, ma le cellule possiedono modi per direzionare la riparazione a sequenze che sono richieste urgentemente. Lo fanno legando la
polimerasi al cammino di riparazione a asportazione di nucleotidi. La RNA polimerasi stalla a queste lesioni e attraverso proteine di accoppiamento direzione il complesso di riparazione
a questi siti. La RNA polimerasi viene poi fatta riprendere da dove si era fermata con una reazione complessa.
\subsection{La chimica delle basi del DNA facilita l'individuazione dei danni}
La doppia elica del DNA sembra ottima per la riparazione e la natura delle quattro basi rende chiara la distinzione tra quelle danneggiate e non danneggiateL ogni possibile deamminazione
presente nel DNA presenta una base innaturale che pu\`o essere riconosciuta e rimossa da una DNA glicolasi specifica. L'uracile nell'RNA \`e stato sostituito dalla timina in quanto 
se no il sistema di riparazione non sarebbe stato in grado di sistinguere una C deamminata da una U naturale. 
\subsection{Speciali translesion della DNA polimerasi sono usate durante emergenze}
Se il DNA soffre di danni pesanti si rende necessaria una strategia diversa: la replicazione della DNA polimerasi si ferma quando incontra DNA danneggiato  e in emergenze vengono
utilizzate altre polimerasi, versatili ma meno accurate detter translesion polimerasi per replicare attraverso il danno. Tali polimerasi possono riconoscere un tipo specifico di danno 
e aggiungere i nucleotidi necessari per ripristinare la sequenza iniziale. Altri fanno congetture educate. Nono hanno un meccanismo di proofreading esonucleoco e sono 
meno discriminanti nella scelta dell'incorporazione del nucleotide e possono pertanto aggiugere solo pochi nucleotidi alla volta. Queste polimerasi portano rischi alla cellula a causa 
della loro imprecisione. 
\subsection{Rotture nel doppio filamento sono efficientemente riparate}
Un tipo di danno al DNA pericoloso avviene quando entrambi i filamenti della doppia elica sono rotti, lasciando nessun filamento stampo capace di riparare il danno accuratamente. Due
meccanismi vengono utilizzati per riparare a questi danni. Il primo \`e l'unione delle terminazioni non omologo in cui le terminazioni rotte sono riunite da DNA ligation con la 
perdita dei nucleotidi nel sito dell'untione. Questo meccanismo \`e comune delle cellule somatiche dei mammiferi. Ha come risultato una mutazione e pu\`o inoltre unire due parti 
del cromosoma non inizialmente insieme. Il secondo tipo \`e detto ricombinazinoe omologa. Entrambi vengon utilizzati, il secondo solo durante e poco dopo la replicazione quando cromatidi
sorelle sono disponibili come stampo. 
\subsection{Danno al DNA ritarda la progressione del ciclo cellulare}
Le cellule possono riconoscere molti tipi di danni e gli enzimi sono rendono efficiente la riparazione ritardando la progressione del ciclo cellulare fino a che la riparazione \`e 
completa. Questi ritardi facilitano la riparazione rendendo disponibile il tempo necessario affinch\`e i meccanismi si completino. I danni risultano anche in una maggiore sintesi di 
enzimi di riparazione attraverso speciali proteine di segnale che percepiscono i danni e regolano l'espressione genica. 
\subsection{Riparazione omologa}
Un ulteriore meccanismo di riparazione del DNA \`e detta ricombinazione omologa, la cui caratteristica fondamentale \`e uno scambio di filamenti di DNA tra un paio di duplex omologhi
della sequenza di DNA, segmenti di doppia elica che sono molto simili nella sequenza nucleotidica. Questo scambio permette a una lunghezza di DNA duplex di essere stampo per 
ripristinare informazioni perse o danneggiate. Essendo non legata al filamento complementare al danno \`e molto versatile. \`E il modo in cui vengono riparate rotture a doppio filamento.
Questi danno la maggior parte delle volte nascono quando la forcella di replicazione si stalla o viene rotta indipendentemente. La ricombinazione omologa corregge questi errori ed \`e
fondamentale per ogni cella proliferante. \`E il meccanismo pi\`u versatile. Inoltre durante la meiosi catalizza lo scambio di informazioni genetiche tra cromosomi omologhi materni e 
paterni creando nuove combinazioni di sequenza di DNA passata ai discendenti. 
\subsection{L'accoppiamento delle basi del DNA guida la ricombinazione omologa}
La ricombinazione omologa avviene unicamente tra duplex di DNA che hanno regioni di sequenza estensiva omologhe. I due duplex che stanno svolgendo ricombinazione mologa testano la 
sequenza dell'altro  da un duplex e un singolo filamento dell'altro. La corrispondenza non deve essere perfetta ma deve essere molto simile. L'interazione di accoppiamento di basi
pu\`o essere imitato permettendo a una doppia elica di riformarsi dai singoli filamenti nel processo di DNA rinaturazione o ibridazione che avviene quando rare collisioni giustappungono
nucleotidi complementari su due filamenti singoli, permettendo la creazione di corte lunghezze di doppia elica. Questa parte lenta \`e seguita da un zippering in qui la regione di doppia
elica si estende per massimizzare il numero di rezioni che accoppiano le basi. L'ibridazione pu\`o creare regioni di doppia elica consistenti di filamenti che si sono originati da 
molecole duplex diverse se sono similmente complementari. La ricombianzione omologa avviene attraverso un insieme di reazioni che permettono a due duplex del DNA si campionare le 
rispettive sequenze senza dissociarsi in singoli filamenti. 
\subsection{La ricombinazione omologa pu\`o riparare perfettamente rotture a doppio filamento nel DNA}
La ricombinazione omologa pu\`o riparare doppi filamenti accuratamente, senza alcuna perdita o alterazione al sito di riparazione. Per fare questo il DNA danneggiato deve essere 
avvicinato adl DNA omologo che serve da stampo per la riparazione. Per questa ragione il processo avviene dopo la replicazione, che pu\`o creare rischi di incidenti che richiedono questa
riparazione. Il cammino pi\`u semplice per la riparazione mostra come il duplex di DNA danneggiato e l'omologo si ingarbugliano in modo che uno dei filamenti danneggiati pu\`o usare il 
complementare del duplex intatto come stampo. Le terminazioni del DNA danneggiato sono resecati da nucleasi per produrre terminazioni $3'$ a filamento singolo protrundenti. 
Successivamente avviene uno scambio di filamenti o strand invasion in cui una terminazione a filamento singolo $3'$ dalla molecola danneggiata entra nel duplex stampo e cerca per la 
sequenza omologa attraverso accoppiamento delle basi. Dopo che questo \`e avvenuto una polimerasi estende il filamento invasore utilizzando l'informazione fornita dallo stampo 
ripristinando l'informazione. Infine i filamenti si separano, avviene altra sintesi di riparazione e legatura ripristano le doppie eliche originali e completano il processo di 
riparazione. 
\subsection{Il cambio di filamenti avviene grazie la proteina RecA/Rad51}
La proteina che compie lo scambio di filamenti avviene grazie alla proteina RecA in E. coli e Rad51 negli eucarioti. Per catalizzare lo scambio la proteina si lega cooperativamente
al filamento invasore, facendo in modo che l'insieme forzi i lDNA in gruppi di tre nucleotidi consecutivi mantenuti come in una doppia elica convenzionale ma tra triplette adiacenti
il backbone \`e svolto e allungato. Questo filamento si lega al duplex DNA in modo da allungarlo, destabilizzandolo e rendendo la separazione dei filamenti semplice. Il filamento 
invasore pu\`o poi campionare la sequenza attraverso accoppiamento di basi, che avviene in blocchi di triplette: se una corrispondenza \`e trovata si prova la prossima tripletta e cos\`i
via. RecA idrolizza l'ATP e i passaggi precedenti richiedono che ogni monomero di RecA siano legati con esso, ma la ricerca non richiede l'idrolisi, avvenendo per collisione molecolare.
Una volta che la reazione \`e completa l'idrolisi \`e necessaria per disassemblare RecA dal complesso delle molecole di DNA e la DNA polimerasi e DNA ligasi possono completare il 
processo di riparazione.
\subsection{La ricombinazione omologa pu\`o salvare forcelle di replicazione danneggiate}
Il ruolo pi\`u importante della ricombinazione omologa \`e che salva forcelle di replicazione in satllo o danneggiato. Una causa possono essere intervalli vuoti nell'elica di DNA 
sopra la forcella che quando sono raggiunti causano la sua rottura. 
\subsection{Le cellule regolano l'uso della ricombinazione omologa durante la riparazione del DNA}
La ricombinazione omologa presenta dei rischi alla cellula in quanto pu\`o riparare a danni usando bit di genoma sbagliati come stampo. Un cromosoma umano danneggiato pu\`o essere 
riparato usando l'omologo da altri parenti invece che dalla cromatide sorella, convertendo la sequenza del DNA riparato dalla sequenza materna a quella paterna o viceversa. Questo causa 
una perdita di eterozigosit\`a, passaggio critico nella formazione di molti cancri. Per minimizzare il rischio di errori ogni passaggio del processo \`e regolato. Il processamento delle
terminazioni danneggiate \`e coordinato con la divisione cellulare: gli enzimi di nucleasi che causano questo processo sono attivati in parte da fosforilazione solo nelle fasi S e $G_2$
quando un duplex filgio pu\`o essere utilizzato come stampo per la riparazione aiutato dalla prossimit\`a. Il caricamento di RecA e Rad52 \`e controllato attraverso una serie di 
proteine accessori come Rad52 in modo da rendere il processo accurato ed efficiente. Sono create ad alto livello negli eucarioti e trasportate nel nucleo in una forma inattiva. Quando
avvene un danno convergono rapidamente nel sito e attivate. 
\subsection{La ricombinazione omologa \`e cruciale per la meiosi}
La ricombinazione omologa viene usata come metodo per trasportare nuove combinazioni di geni come risultato di scambio di materiale genetico tra diversi cromosomi, parte necessaria per
la meiosi. Diventa parte integrale del processo dove cromosomi sono separati a cellule germinali producendo cromosomi attraverso crossing-over e conversione dei geni, creando 
cromosomi ibridi che contengono informazioni sia materne che paterne. Il processo di base \`e lo stesso.
\subsection{La ricombinazione meiotica comincia con una rottura a doppio filamento programmata}
La ricombinazione meiotica inizia con una proteina specializzata che rompe entrambi i filamenti dell'elica in un cromosoma ricombinante e come una topoisomerasi rimane legata 
covalentemente al DNA rotto. Una nucleasi specializzata degrada la terminazione legata dalla proteina, rimuovendola e lasciando una terminazione $3'$ protrundente. Molte proteine 
specifiche alla meiosi direzionano le altre in modo da svolgere il processo in maniera leggermente diversa. Inoltre la ricombinazione avviene tra cromosomi materni e paterni.
\subsection{Le giunzioni di Holliday sono formate durante la meiosi}
Nella meiosi un intermedio detto giunzione di Holliday o scambio cross-strand pu\`o adottare multiple conformazioni e un insieme speciale di proteine di ricombinazione che vi si legano
e stabilizzano l'isomero simmetrico aperto. Le proteine che legano le giunzioni possono catalizzare una reazione di branch migration dove il DNA \`e avvolto attravreso la giunzione 
rompendo e riformando accoppiamenti di basi. Le giunzioni utilizzano l'idrolisi dell'ATP per espandere le regioni di DNA eteroduplex inizialmente create dalla reazione di scambio di 
filamento. Tipicamente si formano in coppie.
\subsection{La ricombinazione omologa produce crossover e non-crossover durante la meiosi}
Durante la meiosi ci sono due risultati della ricombinazione omologa: negli umani il $90\%$ delle rotture del doppio filamento si risolvono come non-crossover: i due duplex si separano
in una forma non alterata ad eccezione di una regione eteroduplex ficino al sito della rottura. L'altro risultato \`e dovuto alla formazione e taglio di una doppia giunzione di Holliday 
con un enzima specializzato per la formzione di un crossover: le due porzioni originali di ciascun cromosoma sono scambiate, creando due cromosomi che hanno fatto crossover. I 
crossover che si formano sono distribuiti nel cromosoma in modo che il crossover in una posizione inibisce crossing-over nelle regioni vicine. Il meccanismo regolatorio associato non
\`e capito ma viene detto crossover control garantisce la distribuzione costante dei punti di crossover e assicura che ogni cromosoma svolga un crossover ogni meiosi. Circa due 
crossover avvengono per cromosoma e svolgono importante ruolo meccanico nella segregazione dei cromosomi durante la meiosi. La macchina di ricombinazione lascia una regione eteroduplex
con un singolo filamento con la sequenza di DNA dell'omologo paterno accoppiato con un filamento dell'omologo materno. Queste regioni possono tollerare un piccolo numero di coppie non
corrispondenti, marcando siti di conversione genica potenziale. 
\subsection{La ricombinazione omologa solitamente causa conversione genica}
\`E una regola fondamentale della genetica che a parte il DNA mitocondriale, ereditato dalla madre, ogni genitore faccia una contribuzione genetica uguale alla discendenza. Un insieme
completo di geni \`e ereditato dalla madre e uno dal padre. Sottostante a questa legge si trova l'accurata separazione di cromosomi a cellula germinali durante la meiosi. Quando una 
cellula diplide in un genitore fa la meiosi per produrre quattro cellule germinali aploidi met\`a dei geni dovrebbero essere materne e l'altra met\`a paterne. Versioni alternative dello
stesso gene sono dette alleli e la divergenza dalla distribuzione aspettata durante la meiosi \`e detta conversione genica, che avviene per una piccola paret del genoma e in molti casi
parte di un gene viene cambiata. Molti cammini possono causare la conversione genica, ma uno dei pi\`u importanti nasce come conseguenza della ricombinazione durante la meiosi: se i
due filamenti parte di una regione eteroduplex non possiedono sequenze identiche si formano coppie di basi errate e pertanto si utilizza a caso lo stampo materno o paterno per la 
riparazione e pertanto un allele sar\`a perso e l'altro duplicato. 
\section{Trasposizione e ricombinazione specifica al sito conservativa}
L'ordine dei geni sui cromosomi che subiscono la ricombinazione omologa \`e lo stesso. La trasposizione e ricombinazione conservativa specifica al sito sono ricombinazioni che non
richiedono regioni di DNA omologo sostanziale. Queste due reazioni di ricombinazioni possono alterare l'ordine dei geni in un cromosoma e tipi di mutaizoni che introducono nuovi blocchi
di DNA nel genoma. Sono dedicate a muovere una grande variet\`a di segmenti di DNA specializzati o elementi genetici mobili da una posizione all'altra. Spesso uno di questi geni codifica
un enizima che catalizza il bovimento dell'elemento rendendo questa ricombinazione possibile. TUtte le cellule contengono elementi genetici mobili che hanno avuto un ruolo evolutivo
profondo nella formazione del genoma odierno. Sono considerati parassiti molecolari che persistono in quanto la cellula non pu\`o eliminarli, ma possono anche creare benefici a essa. Il
loro movimento produce molte varianti genetiche da cui dipende l'evoluzione in quanto possono riordinare sequenze vicine dell'ospite. 
\subsection{Attraverso la transizione gli elementi genetici mobili possono inserirsi in ogni sequenza di DNA}
Gli elementi che si muovono per trasposizione sono detti trasposoni. In questo processo un enzima specifico codificato dal trasposone detto trasposasi agisce su una sequenza di DNA
specifica a ogni fine del trasposone causando il suo inserimento in un nuovo sito di DNA obiettivo. La maggior parte sono poco selettivi e possono inserirsi in locazioni diverse nel
genoma. La maggior parte si muovono raramente. I trasposoni possono essere classificati in base a struttura e meccanismo di trasposizione in trasposoni solo a DNA retrotrasposoni simili
ai retroviruso e retrotrasposoni non retrovirali. 
\subsection{Trasposoni solo a DNA si possono muovere attraverso un meccanismo a copia-incolla}
Questi trasposoni esitono unicamente come DNA durante il loro mocimento sono predominanti nei batteri e sono responsabili per la resistenza agli antibiotici. Nonostante questi elementi
mobili possono trasporsi solo all'interno della cellula si possono muovere tra cellule attraverso il trasferimento di geni orizzonatale. Una volta introdotti nella nuova cellula si 
possono inserirenel genoma ed essere passati a tutta la progenie. Si possono spostare da un sito donatore a uno obiettivo attraverso una trasposizione copia-incolla dove il trasposone 
\`e asportato da un luogo del genoma e inserito in un altro. La reazione produce una corta duplicazione della sequenza obiettivo al sito di inserzione, ripetizioni che affiancano il 
trasposone servono come records di eventi di trasposizione precedenti. Quando il trasposone \`e asportato dal luogo originale lascia un buco nel cromosoma la cui lesione pu\`o essere
riparata da un sistema di riparazione a doppio filamento se il cromosoma \`e stato appena replicato e esiste una copia dell'ospite. Alternativamente una reazione di unione delle 
terminazioni non omologa avviene e la sequenza del DNA a fianco del trasposone \`e modificata, producendo una mutazione nel sito da cui il trasposone si \`e asportato. 
\subsection{Alcuni virus usano un meccanismo di trasposizione per muoversi nel cromosoma della cellula ospite}
Alcuni virus sono considerati elementi genetici mobili in quanto usano meccanismi di trasposizioni per integrare i loro genomi nella cellula ospite, codificando proteine che incapsulano
la loro informazione genetica in particelle virali che possono infettare altre cellule. La trasposizione ha un ruolo fondamentale nel ciclo vitale dei virus come i retrovirus: al di
fuori della cellula esistono come un singolo filamento di RNA incapsulato in un capside proteico con un enzima trascrittasi inversa. Durante il processo di infezione l'RNA virale 
entra la cellula ed \`e convertito in un doppio filamento di DNA dall'enzima che pu\`o polimerizzare DNA su uno stampo a RNA o DNA. Il termine retrovirus si riferisce alla capacit\`a
del firus di invertire il flusso delle informazioni genetiche. Una volta che la trascrittasi inversa ha prodotto la molecola di DNA sequenze specifiche nelle sue terminazioni sono 
riconosciute da una trasposati chiamata integrasi che lo inserisce nel cromosoma con un meccanismo simile dai trasposoni solo a DNA.
\subsection{Retrotrasposoni simili ai retrovirus assomigliano ai retrovirus ma non possiedono un capside}
Una famiglia di trasposoni detta retrotrasposoni simili ai retrovirus si muove dentro e fuori i cromosomi con un meccanismo simile a quello dei retrovirus. Questi elementi sono presenti
in organismi diversi e non hanno la capacit\`a di lasciare la cellula residente. Il primo passo nella loro trasposizione \`e la trascrizione dell'intero trasposone, producendo una 
copia a RNA edll'elemento che \`e tradotto come un RNA messagiero nella cellula codifica un enzima di trascrittasi inversa che forma una copia di DNA a doppia elica della molecola di 
RNA attraverso un intermedio RNA-DNA ibrido che viene poi integrato nel cromosoma con un enzima di integrasi codificato dall'elemento. 
\subsection{Una grande frazione del genoma umano \`e composta di retrotrasposoni nonvirali}
Una frazione significativa di molti cromosomi dei vertebrati \`e composta da sequenze di DNA ripetute che, negli umani, sono principalmente versioni mutate e troncate di retrotrasposoni
nonvirali. La maggior parte sono ormai immobili, ma una piccola parte \`e ancora capace di muoversi. Questi trasposoni si muovono attraverso un meccanismo che richiede un complesso di
endonucleasi e trascrittasi inversa. L'RNA e la trascrittasi hanno un ruolo pi\`u diretto nell'evendo di ricombinazione. Alcuni non portano la propria endonucleasi o trascrittasi 
inversa.
\subsection{La ricombinazione conservativa specifica al sito pu\`o riordinare il DNA reversibilmente}
Questo tipo di meccanismo di ricombinazione riordina altri tipi di elementi di DNA mobili. In questo cammino rottura e riunione avvengono a due siti speciali, uno in ogni molecola di DNA
partecipante. In base alla posizione e al relativo orientamento dei due siti pu\`o accadere integraizone, esportazione o inversione del DNA. Questo processo avviene grazie a un enzima 
che rompe e riunisce due doppie eliche a sequenze specifiche. Lo stesso enizima che le unisce le pu\`o rirompere, ripristinando la sequenza delle due molecole di DNA originale. Viene
spesso utilizzato dai virus a DNA per muovere i loro genomi in quelli della cellula ospite. Quando \`e integrato il DNA virale viene replicato ed \`e passato a tutte le cellule 
discendenti. Se la cellula ospite subisce danni il virus pu\`o invertire la reazione di ricombinazione, asportare il proprio genoma e incapsularlo in una particella di virus. Sono
molte le differenze con la trasposizione: richiede sequenze specializzate sia sul donatore che sul recipiente che contengono siti di riconoscimento per la ricombinasi che catalizzer\`a
il riordinamento. I meccanismi di reazione sono fondamentalmente diversi: la ricombinasi che catalizza la reazione assomiblia alla topoisomerasi in quanto forma legami covalenti 
temporanei con il DNA e usa questa energia per completare il riordinamento del DNA. TUtti i legami fosfati utilizzati durante un evento di ricombinazione sono ripristinati al 
completamento. La ricombinazione conservativa specifica al sito viene utilizzata da molti batteri per controllare l'espressione di un gene specifico.

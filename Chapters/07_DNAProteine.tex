\chapter{Come una cellula legge il genoma, dal DNA alle proteine}
Il DNA nel genoma usa l'RNA come intermediario nella sintesi delle proteine. Quando una cellula necessita una proteina utilizza la sequenza appropriata della catena nucleotidica 
copiandola in RNA durante la trascrizione che direziona direttamente la sintesi della proteina durante la traduzione. Esistono varianti di questo processo in cui i trascritti a RNA 
vengono processati nel nucleo con processi come RNA splicing prima che posano uscire da esso. Questi cambi possono cambiare il significato di una molecola di DNA. Per molti geni inoltre
il prodotto finale \`e RNA. I genomi di organismi multicellulari sono disordinati con corti esoni e lunghi introni. Sezioni che codficano il DNA sono separate da lunghe sequenze senza 
apparente significato. 
\section{Dal DNA all'RNA}
Essendo che molte copie identiche dello stesso RNA possono essere completate dallo stesso gene ogni molecola di RNA pu\`o guidare la sintesi di molte proteine identiche, ma i geni sono
trascritti e tradotti a tassi diversi, permettendo la cellula di avere vaste quantit\`a di alcune proteine e piccole di altre. Inoltre la cellula regola l'espressione di ognuno dei suoi
geni secondo i suoi bisogni, controllando la produzione del suo RNA>
\subsection{Le molecole di RNA hanno un unico filamento}
Il primo passo nella lettura delle istruzioni geniche \`e la copia di una particolare sequenza della sequenza di nucleotidi in una a RNA. L'infromazione nell'RNA \`e scritta nello stesso
linguaggio del DNA e questo processo \`e pertanto detto trascrizione. L'RNA \`e un polimero lineare composto da quattro tipi di subunit\`a nucleotidiche legate da legami a fosfodiestere.
Differisce dal DNA in quanto i nucleotidi nell'RNA sono ribonucleici, ovvero contengono ribosio e contiene la base uracile invece della timina che si pu\`o legare all'adenina. La
struttura complessiva \`e molto diversa: l'RNA \`e a filamento singolo e una catena pu\`o piegarsi in una forma simile a una proteina permettendogli di avere precise funzioni strutturali
e catalitiche.
\subsection{La trascrizione produce RNA complementare a un filamento di DNA}
L'RNA \`e sintetizzato attraverso la trascrizione del DNA, che comincia con l'apertura e lo svolgimento di una piccola porzione della doppia elica che espone le basi sui filamenti, uno
dei quali agisce come stampo per la sintesi della molecola di RNA. La sequenza di nucleotidi \`e determinata dall'accoppiamento di basi complementari tra i nucleotidi che arrivano e lo
stampo. Quando avviene una corrispondenza il ribonucleide che arriva \`e legato covalentemente con la catena crescente in una reazione catalizzata da enzimi. La catena \`e allungata un
nucleotide alla volta e possiede una sequenza complementare allo stampo. Il filamento di RNA non rimane legato con lo stampo ma \`e separato dietro al regione dove i nucleotidi sono
aggiunti causando il rilasciamento come singolo filamento. Le molecole di RNA sono inoltre molto pi\`u corte rispetto le molecole di DNA. 
\subsection{L'RNA polimerasi causa la trascrizione}
Gli enzimi che svolgono la trascrizione sono detti RNA polimerasi e catalizzano la formazione del legame fosfodiestere che lega i nucleotidi muovendosi lungo il DNA, svolgendo l'elica
sopra il sito attivo per la polimerizzazione. La catena di RNA \`e estesa nella direzione $5'$-$3'$. I substrati sono ribonicleoside trifosfato la cui idrolizzazione fornisce l'energia
necessaria alla reazione. Il rilascio immediato dell'RNA significa che le copie possono essere create in poco tempo, con la sintesi di molecole addizionali che inizia prima che quelle
prime sian completate. L'RNA polimerasi catalizza l'unione di ribonucledi e pu\`o cominciare una catena di RNA senza un primer. L'RNA polimerasi fa un errore una volta ogni $10^4$ 
nucleotidi e le conseguenze di tale errore sono meno significative. Inoltre la stessa RNA polimerasi che comincia una molecola di RNA eve finirla senza dissociarsi dallo stampo. L'RNA
polimerasi contiene un meccanismo di proofreading: se un ribonucleotide \`e aggiungo la polimerasi pu\`o indietreggiare e il sito attivo svolge una reazione di asportaizone dove una
molecola d'acqua sostituisce il pirofostato ed \`e rilasciata una molecola di monofosfato. 
\subsection{Le cellule producono diverse categorie di molecole di RNA}
La maggior parte dei geni trasportati in un DNA della cellula specificano la sequenza di amminoacidi della proteina e le molecole di RNA che sono copiate da questi geni sono detti RNA
messaggeri o mRNA. Il prodotto finale di altri geni \`e la molecola di RNA, detti RNA non codificanti che servono come componenti strutturali, enzimatiche e regolatore per molti 
processi. Molecole di RNA piccolo nucleare o snRNA direzionano lo splicing di pre-mRNA per formare mRNA, l'RNA ribosomiale o rRNA forma il nucleo del ribosoma e i tranfer RNA o tRNA
forma gli adattatori che selezionano gli amminoacidi e li mantengono in posizione. I microRNA o miRNA e RNA piccolo interferente siRNA servono come regolatori per l'espressione genica e 
RNA piwi-interagente o piRNA protegge le linee germinali dai trasposoni. I long noncoding RNA o lncRNA con funzione di impalcature e regolano diversi procesi cellulari come
l'inattivazione del cromosoma X. Ogni segmento di DNA trascritto \`e detto unit\`a di trascrizione che tipicamente possiede le informazioni di un gene. La maggior parte dell'RNA nella
cellula \`e rRNA/
\subsection{Segnali codificati nel DNA indicano l'RNA polimerasi dove iniziare e dove finire}
Per trascrivere un gene accuratamente la RNA polimerasi deve riconoscere dove iniziare e finire sul genoma. QUesto avviene in maniera diversa rispetto a batteri ed eucarioti. 
L'iniziazione di una trascrizione \`e il punto in cui la cellula regola quali proteine devono essere prodotto e a quale velocit\`a. L'RNA polimerasi batterica \`e un complesso a 
multisubunit\`a che sintetizza l'RNA. Una subunit\`a addizionale detta fattore sigma $\sigma$ associa con l'enzima nucleo e lo assiste nella lettura dei segnali nel DNA che dicono dove
iniziare la trascrizione. Il fattore $\sigma$ e l'enzima di nucleo formano un oloenzima RNA polimerasi che aderisce debolemente al DNA batterico quando collidono e scivola rapidamente 
lungo il DNA fino a dissociarsi. QUando l'oloenzima arriva a una sequenza speciale che indica il punto di inizio per la sintesi di RNA detto protomero si lega fortemente in quanto il 
fattore $\sigma$  crea contatto specifico con i limiti della base esposti all'esterno nella doppia elica. L'oloenzima polimerasi al protromero apre la doppia elica esponendo una piccola
lunghezza di nucleotidi su ogni filamento chiamata bolla di trascrizione (di circa $10$ nucleotidi), stabilizzata dal legame con il fattore $\sigma$ con le basi non accoppiate. L'altro
filamento agisce come stampo per l'accoppiamento di basi con i ribonucleotidi che arrivano, uniti dalla polimerasi per iniziare la catena di RNA. I primi $10$ nucleotidi sono 
sintetizzati attraverso un meccanismo di ``scrunching" dove la RNA polimerasi rimane legata al protomero e tira il DNA nel suo sito attivo espandendo la bolla di trascrizione. Questo
processo genere stress e ke catene di RNA sono rilasciate e forzando la polimerasi a riiniziare la sintesi. Questo processo di iniziazione abortiva \`e superato e lo stress generato 
aiuta l'enzima a rompere l'interazione con il protomero e con il fattore $\sigma$. La polimerasi inizia a muoversi lungo il DNA sintetizzando l'RNA muovendosi di base in base espandendo
la bolla e contraendola al retro. Si continua l'allungamento della catena fino a che l'enzima incontra un terminatore dove la polimerasi si ferma e rilascia la molecola di RNA e lo 
stampo a RNA. La polimearsi si riassocia con il fattore $\sigma$ ed \`e libera di riiniziare un processo di trascrizione. La maggior parte dei segnali di terminazione nei batteri \`e
formata da una stringa di coppie A-T precedute da una sequenza due volte simmetrica  di DNA che quando trascritta forma una forcina attraverso l'accoppiamento di basi che aiuta il 
disengaggio dell'RNA trascritto dal sito attivo.
\subsection{I segnali di inizio e terminazione della trascrizione sono eterogenei nella sequenza nucleotidica}
Le sequenze di inizio e fine sono codificate da sequenze in relazione, che riflettono aspetti del DNA che sono riconosciuti direttamente dal fattore $\sigma$. Queste caratteristiche 
formano una sequenza di nucleotidi consenzianti, una media di un gran numero di sequenze. Si possono anche riconoscere attraverso la frequenza relativa di basi in ogni posizione. Tale
sequenza nei batteri varia in modo da determinare la forza dei geni (il numero di eventi di iniziazione per gene). Per i terminatori la struttura di base \`e quella che forma la 
forcina nell'RNA. Il filamento scelto per la sintesi dell'RNA dipende dall'orientamento del promotore. 
\subsection{L'iniziazione della trascrizione negli eucarioti richiede molte proteine}
Gli eucarioti possiedono la RNA polimerasi I, II e III, simili strutturalmente e con subunit\`a in comune, ma trascrivono diverse categorie di geni. La I e la III trascrivono geni che 
codificano tRNA, rRNA e piccoli RNA, la II trascrive la maggior parte dei geni, inclusi quelli che codificano le proteine. La RNA polimerasi II  richiede molti fattori detti fattori di 
trascrizione generali e l'iniziazione avviene su DNA condensato in nucleosomi e forme superiori di struttura cromatinica. 
\subsection{La RNA polimearsi II richiede un insieme di fattori di trascrizione generali}
I fattori di trascrizione generali aiutano a posizionare la polimerasi correttamente al promotore, a separare i due filamenti di DNA permettendo l'inizio della trascrizione e rilasciarla
dal promotore per iniziare la modalit\`a di allumgamento. Sono generali in quanto richieste da tutti i promotori utilizzati dalla polimerasi II. Sono un insieme di proteine che 
interagiscono dette TFIIA, TFIIB, TFIIC, TFIID e hanno funzione equivalente al fattore $\sigma$. Il processo di assemblaggio inizia quando TFIID si lega a una sequenza di DNA a doppia 
elica detta TATA box attraverso la subunit\`a TBP. Il legame causa una distorsione nel DNA della TATA box che crea una marcatura per la locazione di un promotore attivo. Altri fattori
insieme alla RNA polimerasi II formano un complesso di iniziazione della trascrizione. Dopo che si \`e formato sul DNA promotore la RNA polimerasi II ottiene l'accesso al filamento 
stampo e TFIIH che contiene una DNA elicasi idrolizza l'ATP svolgendo il DNA. La RNA polimearsi II rimane al promotore sintetizzando corte lunghezze di RNA fino a subire una serie di
cambi conformazionali che le permettono di spostarsi ed entrare nella fase di allungamento. In questa transizione viene aggiunto un gruppo fosfato alla coda della RNA polimerasi detto 
CTD (C-terminal domain). Durante l'iniziazione la serina nella quinta posizione della sequenza ripetuta \`e fosforilata da TFIIH che contiene una chinasi in una delle subunit\`a. La
polimerasi pu\`o poi disengaggiarsi dal cluster e subisce una serie di cambi conformazionali che le permettono di trascrivere per lunghe distanze senza dissociarsi dal DNA. Dopo che si
\`e entrati nella fase di allungamento i fattori di trascrizione generali si separano per iniziare un altro processo. 
\subsection{La polimerasi II richiede attivatori, mediatori e proteine per la modifica della cromatina}
L'inibizione della trascrizione negli eucarioti \`e complessa e richiede molte proteine. Innanzitutto delle proteine dette attivatori trascrizionali devono legarsi a speciali sequenze
del DNA dette enhancers e auitare ad attrarre l'RNA polimerasi II al punto d'inizio. SUccessitamente \`e necessario un complesso proteico detto mediatore che permette alle proteine 
attivatrici di comunicare con la Polimearsi II e con i fattori di trascrizione generali. Alla fine l'iniziazione della trascrizione richiede il reclutamento di enzimi modificatori della
cromatina, complessi di rimodellizazzione e enzimi modificatori degli istoni che aumentano l'accesso al DNA nella cromatina. L'ordine di assemblaggio di queste proteine non segue un 
cammino preciso e differisce per gene. Per cominciare la trascrizione la RNA polimerasi II deve essere rilasciata da questo complesso attraverso proteolisii insito delle proteine 
attivatrici.
\subsection{L'allungamento della trascrizione negli eucarioti richiede proteine accessorie}
Una volta che l'RNA polimerasi ha iniziato la trascrizione si muove a scatti. RNA polimerasi allunganti sono associate con una serie di fattori di allungamento, proteine che diminuiscono
la probabilit\`a che l'RNA polimearsi si dissoci prima di aver finito la trascrizione. SI associano con la polimerasi dopo l'iniziazione. Quando l'RNA polimerasi si muove lungo un gene
alcuni enzimi legati ad essa modificano gli istoni, lasciando una traccia, che potrebbe aiutare nelle trascrizioni successive e nella coordinazione dell'allungamento.
\subsection{La trascrizione crea tensione superelicale}
Il superavvolgimento del DNA \`e una conformazione che il DNA assume quando \`e presente tensione superelicale. Un grande superavvolgimento di DNA si forma per ogni $10$ paia di 
nucleotidi svolti, processo energeticamente favorevole in quanto ripristina una torsione normale nella regione accoppiata rimanente. La RNA polimerasi crea tensione superelicale mentre
si muove lungo una lunghezza di DNA ancorata alla terminazione, con tensione positiva davanti a lei e negativa dietro. Negli eucarioti questa tensione \`e rimossa dalla DNA 
topoisomerasi, mentre nei batteri una DNA girasi usa l'energia dell'idrolisi dell'ATP per imporre superavvolgimenti al DNA mantenendolo in costante tensione, ma in senso opposto da 
quello fornito dalla polimerasi. 
\subsection{L'allungamento della trascrizione \`e strettamente accoppiato con il processamento dell'RNA}
Negli eucarioti la trascrizione \`e il primo passo per la produzione di una molecola di mRNA matura. Un passo successivo \`e la modificazione covalente delle terminazioni dell'RNA e 
la rimozione delle sequenze di introni nel processo di RNA splicing. La terminazione $5'$ viene incappucciata e la $3'$ attraverso polidenilazione che permettono alla cellula di capire
se le terminazioni sono entrambe presenti prima che sia esportata dal nucleo e tradotta. L'RNA splicing permette di sintetizzare proteine diverse dallo stesso gene. La fosforilazione
della coda CTD della polimearsi permette la disassociazione delle altre proteine e l'associazione di nuove. Alcune di queste si legano all'RNA che si sta sintetizzando processandolo. 
\subsection{L'incappucciamento dell'RNA \`e la prima modifica dei pre-mRNA eucariotici}
Appena l'RNA polimerasi II ha prodotto circa $25$ nucleotidi di RNA alla terminazione $5'$ della molecola viene aggiunto un cappuccio formato di una guanina modificata. Tre enzimi in 
successione svolgono la reazione necessaria: una fosfatasi rimuove il fsofato dalla terminazione, una guanil trasferasi aggiunge un GMP in un legame inverso e un metil trasferasi
aggiunge un gruppo metile alla guanosina. Questi ensimi si trovano alla catena della RNA polimerasi fosforilata alla posizione Ser5. Questo cappuccio metile indica la terminazione
$5'$ dell'mRNA eucariotico e lo differenzia da altri tipi di RNA. 
\subsection{L'RNA splicing rimuove le sequenze di introni dal pre-mRNA}
Essendo i geni eucariotici dispersi in sequenze di introni che vengono trascritte insieme agli esoni si devono rimuovere i primi attraverso l'RNA splicing per produrre la proteina.
Questo processo avviene alla produzione dell'mRNA. Ogni splicing rimuove un introne attraverso due rezioni di trasferimento di fosforile o transesterificazioni sequenziali che legano 
insieme gli esoni rimuovendo gli introni. Il macchinario che lo svolge \`e un complesso consistente di $5$ molecole di RNA e centinaia di proteine. Idrolizza molte molecole di ATP per
evento. La complessit\`a assicura uno splicing accurato e flessibile. Oltre agli espetti evoluzionistici della divisione in domini delle proteine per la ricombinazione lo splicing
permette di produrre un insieme di proteine diverse dallo stesso gene. 
\subsection{Sequenze di nucleotidi segnalano dove avviene lo slicing}
Il macchinario dello splicing deve riconoscere tre porzioni della molecola di RNA precursoreL il sito di splice $5'$, quello $3'$ e il ramo nella sequenza di introni che forma la base
con il lazo. Ogni sito possiede una sequenza di nucleotidi simile per ogni introne che fornisce indizi sul luogo di splicing. Queste sono corte e possono avere variabilit\`a estesa e 
pertanto sono presenti altre informazioni per compiere la scelta ultima. 
\subsection{Lo splicing dell'RNA \`e svolto dallo spliceosoma}
I passi fondamentali dello splicing sono svolti da molecole di RNA specializzate che riconoscono la sequenza che riconosce il luogo dello splicing e ne catalizza le reazioni chimiche. 
Sono molecole di circa $200$ nucleotidi e sono U1, U2, U4, U5 e U6 e sono dette snRNA, insieme con almento sette subunit\`a proteiche per formare un snRNP (small nuclear 
ribonucleoprotein). Gli snRNP formano il nucleo dello spliceosoma, un complesso di RNA e proteine che svolge lo splicing. Durante la reazione di splicing il riconoscimento delle
giunzioni $5'$, $3'$ e del sito di ramificazione avviene attraverso accoppiamento di basi tra il snRNA e le sequenze di RNA nel substrato. All'interno della cellula il complesso esiste
come un assemblaggio vago di tutte le componenti che svolgono lo splicing come un unit\`a coordinata che continua a riordinarsi ogni volta che compie uno splice. 
\subsection{Lo spliceosoma usa l'idrolisi dell'ATP per produrre una sere di riordinamenti RNA-RNA}
L'idrolisi dell'ATP \`e necessaria per l'assemblaggio e riordinamento dello spliceosoma per rompere e formare interazioni RNA-RNA. Ogni splice richiede circa $200$ proteine. Questi
riordinamenti permettono l'esaminazione del pre-RNA dal snRNP. U1 riconosce il sito $5'$ attraverso l'accoppiamento di basi e successivamente questi legami sono rotti con l'idrolisi 
dell'ATP e viene sostituito con U6. Questi tipi di riordinamento avvengono molte volte e permettono un controllo da parte dello spliceosoma dei segnali di splicing aumentando la 
precisione del processo. Questi riordinamenti avvengono anche per creare i siti attivi nello spliceosoma per le due transesterificazioni, cosa che avviene sequenzialmente dopo che i 
segnali di splicing sono stati analizzati pi\`u volte. I siti catalitici sono formati da proteine e molecole di RNA e le seconde catalizzano la reazione chimica. Una volta che la 
chimica dello splicing \`e completata gli snRNP rimangono legati al lazo. Il loro disassemblaggio righiede un altra serie di riordinamenti RNA-RNA che richiedono l'idrolisi dell'ATP che
permettono il ritorno di snRNA alla loro conformazione originale che ne permette il riutilizzo. Al completamento lo spliceosoma direziona un insieme di rpoteine per legarsi all'mRNA
vicino la posiizone precedentemente occupata dall'introne detto complesso di giunzione degli esoni (EJC) che marcano il sito di uno splicing riuscito e influenzano il destino dell'nRNA.
\subsection{Altre propriet\`a del pre-mRNA e la sua sintesi spiega le scelte dei siti propri di splice}
Il meccanismo di riconoscimento dello spliceosoma sfrutta due strategie addizionali per aumentarne l'affidabilit\`a. Il primo \`e una conseguenza di essere accoppiato con la 
trascrizione: quando questa procede la coda fosforilata della RNA polimearsi porta varie componenti dello spliceosoma che sono direttamente trasferite dalla polimerasi all'RNA mentre 
emerge da essa, aiutando a atenere traccia di esoni e introni. La seconda strategia \`e detta definizione degli esoni: la dimensione degli esoni tende a essere pi\`u uniforme di quella
degli introni e attraverso questa definizione si possono ricercare sequenze di esoni di dimensione omogenea. Mentre la sintesi dell'RNA procede un gruppo di componenti addizionali come 
proteine SR si assemblano sulla sequenza di esoni e aiutano a marcare i siti di splice. Queste proteine reclutano U1 snRNA che marca il limite dell'esone downstream e U2 che specifica
l'upstream. Questi processi aumentano la precisione del deposito delle componenti iniziali dello splicing. Negli esoni sono presenti sequenze dette splicing enhancers. Le sequenze di
introni non sono rimosse dall'RNA nello stesso ordine in cui sono marcate. 
\subsection{La struttura cromatinica influenza l'RNA splicing}
I nucleosomi tendono a essere posizionati sugli esoni e causano la proteina responsabile per la definizione degli esoni di assemblarso all'RNA quando emerge dalla polimerasi. Cambi 
nella struttura cromatinica sono usati per cambiare i pattern di splicing in due modi. Siccome splicing e trascrizione sono accoppiate il tasso con cui la RNA polimerasi si muove 
lungo il DNA ha effetto sul tasso di splicing: minore la velocit\`a minore il salto di esoni: l'assemblaggio dello spliceosoma iniziale pu\`o essere completato prima che una scelta 
alternativa di sito di splicing venga presentata. I nucleosomi in cromatina condensata possono causare una pausa nella polimerasi. Inoltre specifiche modifiche agli istoni attraggono 
componenti dello spliceosoma che possono essere facilmente trasferiti all'RNA emergente. 
\subsection{Lo splicing di RNA mostra plasticit\`a notevole}
In confronto ad altri processi nell'espressione dei geni lo splicing \`e flessibile. Il macchinario di splicing si \`e evoluto in modo da cercare il pattern migliore per le giunzioni, ma
se uno di essi \`e stato danneggiato da una mutazione ricerca il prossimo migliore. Questa plasticit\`a del processo succerisce che cambi nei pattern di splicing sono stati importanti
nell'evoluzione di geni e organismi e che mutazioni che riguardano lo splicing possono essere distrutte per l'organismo. La cellula inoltre pu\`o facilmente regolare i pattern di RNA
splicing in modo da podurre diverse forme di una proteina a tempi e tessuti diversi. 
\subsection{Rna splicing catalizzato dallo spliceosoma si \`e probabilmetne evoluto da meccanismi di autosplicing}
Le cellule ancestrali utilizzavano RNA  per le catalisi principali e salvavano informazioni geniche nell'RNA rispetto al DNA. Queste reazioni di splicing hanno molto probabilmente avuto
un ruolo fondamentale. Come prova rimangono introni a RNA autosplicing. 
\subsection{Enzimi di processamento dell'RNA generano la terminazione $\mathbf{3'}$ degli mRNA eucariotici}
Qyabdi la RNA polimerasi II raggiunge la fine di un gene un meccanismo garantisce che la $3'$ fine del pre-mRNA sia processata. La posizione della terminazione $3'$ \`e specificata da
segnali codificati dal genoma, trascritti nell'RNA mentre la polimerasi si muove attraverso essi e successivamente riconosciuti da una serie di proteine che si legano all'RNA e 
enizmi di processamento dell'RNA. Due proteine a subunit\`a multiple dette CstF (cleavage simulation factor) e CPSG (cleavege and polyadenylation spcificity factor) hanno grande 
importanza. Entrambe queste proteine viaggiano con la coda dell'RNA polimerasi e sono trasferite alla terminazione $3'$ quando emerge dalla polimerasi stessa. Una volta che queste
proteine si legano con alle sequenze di riconoscimento altre proteine si assemblano con esse per creare la terminazione $3'$ dell'mRNA. L'RNA \`e rotto dalla polimerasi, successivamente
un enzima detto poly-A polimearsi (PAP) aggiunge sequenzalmente $200$ nucleotidi A alla terminazione appena prodotta. Il precursore delle addizioni \`e ATP. Mentre questa coda poly-A
viene sintetizzata (senza stampo) le proteine che si legano ad essa sono asseblate su di essa. Dopo che la terminazione \`e stata separata la polimerasi continua a trascrivere e l'RNA
che sintetizza non possiede un cappuccio $5'$ e viene degradato da una esonucleasi trasportata lungo la coda della polimerasi che causa l'eventuale separazione della polimerasi dallo
stampo e la terminazione della trascrizione.
\subsection{mRNA maturi eucariotici sono esportati selettivamente dal nucleo}
La sintesi e il processamento del pre-mRNA nel nucleo si svolge in maniera ordinata, ma solo una piccola percentuale di questo viene ulteriormente utilizzato dalla cellula, mentre il
resto \`e non solo inutile ma potenzilamnete dannoso. Per distinguere da questo pre-mRNA e l'mRNA maturo mentre una molecola di RNA viene processata perde delle proteine e ne 
acquisce altre: la presenza di una proteina snRNP significa splicing incompleto o errato. Solo quando le proteine presenti sulla molecola di mRNA segnalano collettivamente che il
suo processamento \`e stato completato con successo l'mRNA \`e esportato dal nucleo nel citosol, dove viene tradotto in proteine. I resti rimangono nel nucleo dove sono degradati
dall'esosoma nucleare, un complesso proteico ricco di RNA esonucleasi. Le molecole di pre-mRNA pi\`u comuni della trascrizione sono hnRNP (heterogeneous nuclear ribonuclear proteins) e 
alcune di esse svolgono le eliche a forcina nell'RNA in modo da permettere una facile lettura dei segnali e uno splicing pi\`u semplice. I mRNA maturi sono guidati attraverso il 
complesso di pori nucleici o NPC, canali acquosi nella membrana nucleare che connettono il nucleoplasma con il citosol. Il passaggio di macromolecole richiede energia per un trasporto
attivo in entrambe le direzioni attraverso il complesso. Le macromolecole sono mosse attraverso recettori di trasporto nucleari che le trasportano selettivamente. Affinch\`e avvenga 
l'esportazione dell'mRNA un recettore specifico deve essere caricato su di esso, un passo che avviene insieme alla rottura $3'$ e alla poliadenilazione. Una volta che l'esportazione
avviene il recettore si dissocia, rientra nel nucleo e viene riutilizzato. Alcune delle proteine possono influenzare il comportamento dell'RNA successivamente all'esportazione, come
la stabilit\`a, l'efficenza della traduzione e la destizione ultima. 
\subsection{Anche gli RNA non codificanti sono sintetizzati e processati nel nucleo}
La maggior parte dell'RNA nella cellula ha funzioni strutturali e catalitiche. L'RNA pi\`u presente \`e quello ribosomiale rRNA, per circa l'$80\%$. Questo RNA forma il nucleo del
ribosoma. Gli eucarioti possiedono la RNA polimerasi I per la sintesi degli rRNA. \`E strutturalmente simile alla RNA polimerasi II senza una coda C-terminale e pertanto i suoi 
trascritti non sono n\`e incappucciati n\`e poliadenilati. Essendo le componenti a RNA del ribosoma prodotti finali dei geni in ogni cella sono presenti multiple copie dei geni rRNA. Le
cellule umane ne contengono $200$ per genoma aploide in piccoli cluster su $5$ cromosomi diversi. Ci sono quattro tipi di rRNA eucariote, ognuno presente in una copia per ribosoma. Tre
dei quattro (18S, 5.8S e 28S) sono creati modificando e rompendo un rRNA precursore, mentre il quarto (5S) sintetizzato da un cluster separato dalla polimerasi III. Le modificazioni
che avvengono al precursore a $13^.000$ nucleotidi sono $100$ metilazioni delle posizioni $2'-OH$ sui zuccheri nucleotidici e $100$ isomerizzaazioni dei nucleotidi uridina a 
pseudouridina. Queste modifiche aiutano il piegamento e assemblaggio dei rRNA finali o alterano leggermente la funzione dei ribosomi. Ogni alterazione \`e fatta a una posizione specifica
determinata da RNA guida che si posizionano sul precursore tramite accoppiamento di basi e portano l'enzima modificatore alla posizione corretta. Tutti questi RNA sono detti piccoli RNA 
nucleari o snoRNA e svolgono le loro funzioni in sottocompartimenti del nucleo detti nucleoli. Molti sono codificati dagli introni di altri geni e sono sintetizzati da RNA polimerasi II
e processati da sequenze di introni esportate.
\subsection{Il nucleolo \`e una fabbrica di produzione di ribosomi}
Il nucleolo \`e il sito per il processamento di rRNA e il loro assemblaggio in subunit\`a del ribosoma. Non \`e confinato da una membrana ma consiste di un grande aggragato di 
macromolecole includenti i geni rRNA, rRNA precursori, maturi, enzimi per il loro processamento, snoRNP, fabbriche si assemblaggio come ATPasi, GTPasi, proteina chinasi e RNA elicasi, 
proteine ribosomiali e ribosomi parzialmente assemblati. Molti tipi di molecole di RNA svolgono un ruolo centrale nella chimica e struttura del nucleolo. I geni di rRNA, distribuiti in
$10$ cluster negli umani, durante l'interfase creano degli anelli che creano una parte del nucleolo, durante la fase M quando i cromosomi si condensano il nucleolo si frammenta e 
disappare. Nella parte di telofase della mitosi, quandi i cromosomi ritornano al loro stato semi-disperso riappare. La sua dimensione dipende dal numero di ribosomi che la cellula sta
producendo. L'assemblaggio del ribosoma \`e un processo complesso. Oltre al ruolo centrale nella biogenesi del ribosoma il nucleolo \`e il sito dove altri RNA non codificanti sono
prodotti e complessi RNA-proteine sono assemblati. Il nucleo pu\`o pertanto essere considerato come una fabbrica in cui RNA non codificanti sono trascritti, processati e assemblati con
le proteine formando una grande variet\`a di complessi ribonucleoproteici. 
\subsection{Il nucleo contiene una variet\`a di aggregati subnucleari}
Nel nucleo sono presenti altri corpi come i corpi di Cajal senza membrana e altamente dinamici in base alle necessit\`a della cellula. Il loro assemblaggio \`e mediato dall'associazione
di domini proteici semplici e la loro apparenza \`e il risultato di associazioni strette di elementi proteici e a RNA coinvolti nella sintesi, assemblaggio e conservazione di 
macromolecole coinvolte nell'espressione genica. I bodi di Cajal sono siti dove i snRNP e i snoRNP svolgono i loro ultimi passi di maturazione e dove i snRNP sono riciclati e i loro RNA
si resettando dopo i riordinamenti dello splicing. I clusteri di granuli intercromatinici sono proposti come pile di snRNP e altre componenti di processamento di RNA che sono 
usate nella produzione di nRNA. Sembra che la funzione principale di questi aggregati sia concentrare i componenti in modo da velocizzare il loro assemblaggio. I siti dello splicing
sono nell'ordine delle migliaia, altamente dinamici e il risultato dell'associazione di componenti di trascrizione e splicing per la creazione di piccole fabbriche, il nome dato ad
aggregati specifici contenenti un'alta concentrazione di componenti selezionate che creano catene di montaggio biochimiche. 
\section{Da RNA a proteine}
La maggior parte dei geni nella cellula producono molecole di mRNA che servono come intermediari sul cammino verso le proteine. 
\subsection{Una sequenza di mRNA \`e decodificata in insiemi di tre nucleotidi}
L'informazione di un nRNA maturo \`e utilizzata per sintetizzare una proteine. Questa sintesi \`e detta traduzione. Essendoci $4$ nucleotidi diversi e $20$ amminoacidi la traduzione non
avviene uno a uno. Le regole di traduzione sono dette codice genetico. La sequenza di nucleotidi viene letta in gruppi consecutivi di $3$ nucleotidi con $64$ possibili combinazioni. 
Il codice \`e ridondante e alcuni amminoacidi sono codificati da pi\`u triplette dette codoni che codificano un amminoacido o la terminazione del processo di traduzione. Questo codice
\`e utilizzato universalmente, con piccole differenze nei mitocondri, che hanno sistemi indipendenti. Una sequenza di RNA pu\`o essere tradotta in uno di tre diversi reading frame, 
dipendenti dal luogo di inizio della decodifica. SOlo uno dei tre codifica la proteina richiesta. 
\subsection{Molecole di tRNA combinano amminoacidi ai codoni}
I codoni non si legano direttamente all'amminoacido ma si richiede una molecola di adattamento che pu\`o riconoscere e legare sia il codone che l'amminoacido. Questi consistono di un
insieme di piccole molecole di tRNA lunghi $80$ nucleotidi piegandosi in una conformazione precisa. Quattro segmenti di tRNA sono a doppia elica producendo una molecola simile a un
quadrifoglio che viene successivamente piagata in una forma a L compatta tenuta insieme da legami a idrogeno tra le diverse regioni della molecola. Due regioni di nucleotidi non
accoppiati alle terminazioni della molecola sono fondamentali: una forma l'anticodone, tre nucleotidi che si accoppiano con il codone complementare e l'altra una regione a filamento
singolo alla terminazione $3'$ dove l'amminoacido si attacca al tRNA. La ridondanza indica che esistono pi\`u di un tRNA per amminoacido e che alcuni tRNA possono legarsi a pi\`u codoni.
Alcuni tRNA richiedono accoppiamento di basi accurato solo per due posizioni e possono tollerare una corrispondenza sbagliata (wobble) che spiega perc\`e molti codoni 
alternativi per un amminoacido differiscono solo nel terzo nucleotide. 
\subsection{tRNA sono modificati covalentemente prima che escano dal nucleo}
La sintesi del tRNA avviene dalla RNA polimearsi III. Sono tipicamente sintetizzati come precursori pi\`u grandi, che sono rifilati per produrre tRNA maturo. Contengono anche introni. Lo
splicing usa un meccanismo di copia-incolla catalizzato da proteine. Entrambi richiedono che il precursore sia correttamente piegato nella configurazione a quadrifoglio. Tutti i tRNA
sono modificati chimicamente ($1$ in $10$ nucleotidi \`e una versione alterata del ribonucleotide standard) che facilitano il riconoscimento del codone appropriato. 
\subsection{Specifici enzimi accoppiano amminoacidi con la molecola di tRNA appropriata}
Il riconoscimento e l'attaccamento dell'amminoacido corretto dipende dall'enzima amminoacil-tRNA sintetasi che lega covalentemente ogni amminoacido con l'insieme corretto di molecole di
tRNA. La maggior parte delle cellule possiedono una sintetasi diversa per ogni amminoacido. Queste reazioni attaccano alla terminazione $3'$ del tRNA l'amminoacido accoppiate con 
l'odrolisi dell'ATP producendo un legame ad alta energia tra il tRNA e l'amminoacido, energia utilizzata per legare l'amminoacido covalentemente con la catena polipeptidica crescente. 
Il codice genetico \`e tradotto pertanto da due insiemi di adattatori che agiscono sequenzialmente, ognuno dei quali corrisponde una superficie molecolare ad un altra con grande 
specificit\`a.
\subsection{La modifica da tRNA sintetasi assicura accuratezza}
La maggior parte delle sintetasi selezionano l'amminoacido corretto con un meccanismo a due fasi. Tale amminoacido ha la maggiore affinit\`a per il sito attivo della sintetasi ed \`e
favorito, ma la discriminazione tra amminoacidi simili avviene in un secondo passaggio dopo che l'amminoacido \`e stato legato covalentemente a AMP. Quando il tRNA si lega la sintetasi
prova a forzare l'amminoacido adenilato in una seconda tasca di modifica nell'enzima la cui dimensione permette l'accesso unicamente ad amminoacidi strettamente imparentati con quello
corretto. In questa tasca l'amminoacido \`e rimosso dall'AMP attraverso idrolisi, aumentando l'accuratezza a un errore ogni $40^.000$ accoppiamenti. La sintetasi deve anche essere in 
grado di riconoscere il corretto insieme di tRNA e la loro complementarit\`a chimica permette di investigare diverse cratteristiche del tRNA. La maggior parte viene riconosciuta 
direttamente: ci sono tre tasche adiacenti di legame con i nucleotidi, ognuna delle quali complementare in forma e carica ad un nucleotide nell'anticodone. 
\subsection{Gli amminoacidi sono aggiunti alla terminazione \ce{C-} di una catena polipeptidica crescente}
La reazione fondamentale nella sintesi di una proteina \`e la formazione di un legame peptide tra il gruppo carbossile alla fine di una catena polipeptidica e un gruppo ammino su un 
amminoacido in arrivo. Una proteina \`e sintetizzata per passaggi dalla terminazione \ce{N-} a quella \ce{C-} durante l'intero processo la terminazione carbossile crescente rimane
attivata da l'attacco covalente a una molecola di tRNA. Ogni addizione rompe il legame covalente ad alta energia, sostituiendolo con uno uguale con l'amminoacido pi\`u recente. In
questo modo ogni amminoacido trasporta con s\`e l'energia di attivazione necessaria per l'addizione del prossimo amminoacido. 
\subsection{Il messaggio di RNA \`e decodificato nei ribosomi}
La sintesi delle proteine \`e guidata da molecole di mRNA. Per mantenere un reading frame corretto e garantire accuratezza la sintesi avviene nel ribosoma, un complesso di proteine
ribosomiali e molecole di RNA ribosomiali rRNA. Sono presenti in milioni nel citoplasma. Le loro subunit\`a grande e pissola sono assemblate nel nucleo, dove rRNA si associano con le 
proteine ribosomiali trasportate l\`i. Queste due subunit\`a sono esportate nel citoplasma dove si uniscono per sintetizzare le proteine. I ribosomi eucariotici e batterici hanno
strutture e funzioni simili, composti da due subunit\`a, una grande e una piccola che si uniscono formando un ribosoma con una massa di milioni di danlton. La subunit\`a piccola 
crea il framework in cui i tRNA sono corrisposti ai codoni dell'mRNA, mentre la grande catalizza la formazione dei legami peptidi tra gli amminoacidi. Quando non sono attive le 
subunit\`a sono separate. Si uniscono su una molecola di mRNA vicino alla terminazione $5'$ per iniziare la sintesi di una proteina. L'mRNA viene poi tirato attraverso il ribosoma tre
nucleotidi alla volta. Mentre il codona entra la sequenza \`e tradotta in amminaocidi attraverso il tRNA. Quando si incontra un codone di stop il ribosoma rilascia la proteina finita e 
le due subunit\`a possono separarsi per un futuro riutilizzo. In un secondo si possono aggiungere $2$ amminoacidi per quelli aucariotici, $20$ per i baatterici. Il ribosoma contiene
quattro siti di legame per le molecole di RNA: uno per l'mRNA e tre (siti A, P e E) per i tRNA, collegate strettamente nei siti A e P solo se il suo anticodone forma accoppiamento di 
basi con il codone complementare nel ribosoma. I siti A e P sono abbastanza vicini da permettere la formazione tra le due molecole di tRNA di legami tra le basi con i codoni adiacenti
con la molecola di mRNA. Una volta che inizia la sintesi delle proteine ogni nuovo amminoacido \`e aggiunto alla catena crescente in un ciclo di reazioni a quattro passaggi: il legame
di tRNA, la formazione del legame peptide, la traslocazione della grande subunit\`a e la traslocazione della piccola subunit\`a. Come risultato dei due passi di traslocazione l'intero
ribosoma si muove di tre nucleotidi lungo l'mRNA.
\subsection{Fattori di allungamento portano avanti la traduzione e ne aumentano l'accuratezza}
Due fattori di allungamneto entrano e lasciano il ribosoma durante ogni ciclo idrolizzando GTP in GDP con conseguente cambi conformazionali. QUesti fattori sono chiamati EF-Tu e EF-G nei
batteri e EF1 e EF2 negli eucarioti. L'accoppiamento con questi fattori e i la loro transizione di conformazione durante il ciclo velocizza e rende la sintesi pi\`u accurata. I cicli di
associazione, idrolisi e disassociazione garantiscono che i cambi avvengono nella direzione corretta. EF-Tu aumenta l'accuratezza in quanto pu\`o legare contemporaneamente GTP e gli 
amminoacil-tRNA. In qeusta forma l'interazione codone-anticodone avviene nel sito A. A causa dei cambi di energia libera ssociati la corrispondenza corretta si lega pi\`u strettamente,
ma con differenze troppo lievi per garantire accuratezza. Per aumentare l'accuratezza della reazione il ribosoma e il EF-Tu lavorano insieme: i $16$ rRNA nella piccola subunit\`a 
determinano la correttezza della corrispondenza codone-anticodone piegandosi intorno ad esso e controllando i dettagli molecolari. Quando si trova una corrispondenza corretta il rRNA
si chiude strettamente intorno alla coppia causando un cambio conformazionale al ribosoma che causa l'idrolisi del GTP dall'EF-Tu. SOlo quando il GTP viene idrolizzato  ET-Tu rilascia 
la stretta sul amminoacil-tRNA e gli permette di essere utilizzato nella sinetsi. Se la corrispondenz anon avviene i tRNA escono dal ribosoma prima che possano essere utilizzati. Dopo
che il GTP viene idrolizzato e l'EF-Tu si disassocia c'\`e un ritardo mentre l'amminoacido si muove in posizione, impedendo ai tRNA incorretti di proseguire la sintesi in quanto si
disassociano troppo rapidamente. Un interazione codone-anticodone al sito P che avviene dopo l'incorporazione di un amminoacido errato causa un aumento del tasso di letture errate al 
sito A. Eventi scorretti successivi causano una terminazione prematura causata dai fattori di rilascio, che rilasciano la proteina erratat per la degradazione. 
\subsection{Molti processi biologici superano le limitazioni intrinseche all'accoppiamento di basi complementari}
Altri meccanismi vengono utilizzati per aumetnare la specificit\`a della sintesi. Il primo \`e l'adattamento indotto: l'interazione codone-anticodone \`e controllata dall'accoppiamento
di basi e dal piegamento del ribosoma, che dipende dalla correttezza della corrispondenza. Un secondo principio \`e il proofreading cineticoL l'idrolizzazione del GTP dopo l'
accoppiamento iniziale crea un passaggio irreversibile e inizia un tempo su un delay durante il quale l'amminacil-tRNA si muove nella posizione corretta per la catalisi. Durante questo
ritardo le coppie incorrette hanno una probabilit\`a maggiore di dissociarsi in quanto la relazione con il tRNA \`e pi\`u debole e il ritardo maggiore. 
\subsection{L'accuratezza nella traduzione richiede una spesa di energia libera}
La traudzione \`e un compromesso tra velocit\`a e accuratezza, inoltre la sintesi delle proteine richiede pi\`u energia libera di tutti gli altri processi biosinetici. Quattro legami
fosfati ad alta energia sono rotti per ogni nuovo legame peptidico: due per caricare il tRNA con l'amminoacido e due per il ciclo di reazioni durante la sintesi nel ribosoma. Altra
energia viene consumata ogni volta che un tRNA incorretto entra nel ribosoma, comincia l'idrolisi del GTP ed \`e rifiutato. 
\subsection{Il ribosoma \`e un ribozima}
Il ribosoma \`e un grande complesso formato per due terzi da RNA e per un terzo da proteine. Gli rRNA sono responsabili per la sua struttura, la sua abilit\`a di posizionare i tRNA sui 
mRNA e l'attivit\`a catalitica. Gli RNA ribosomiali sono piegati in strutture tridimensionali precise e compatte che formano il nucleo del ribosoma e ne determinano la forma. Le proteine
si trovano generalmente sulla superficie e riempioni i vuoti e fessure dell'RNA piegato. Alcune di queste mandano fuori regioni di catena polipeptidica che possono penetrare in buchi 
del nucleo, il loro ruolo principale sembra quello di stabilizzare il nucleo permettendo i cambi conformazionali necessari. Aiutano inoltre nell'assemblaggio degli rRNA che costituiscono
il nucleo. Anche il sito catalitico per la formazione del legame peptidico \`e costituito da RNA nonostante non contenga gruppi funzionali facilmente ionizzabili. Si crede che la 
struttura dei 23S rRNA formi una tastca altamente strutturata che orienta precisamente i due reagenti attraverso una rete di legami a idrogeno. Il tRNA nel sito P contribuisce con un
gruppo OH al sito attivo e partecipa direttamente nella catalisi, assicurando che la reazione avvenga solo quando il tRNA \`e posizionato correttamente. Le molecole di RNA con attivit\`a
catalitica sono dette ribozimi. 
\subsection{La sequenza nucleotidica negli mRNA segnala dove iniziare la sintesi delle proteine}
L'iniziazione e la terminazione della traduzione condividono caratteristiche con il ciclo di allunbamento. Il sito di inizio della sintesi della proteina sull'mRNA \`e cruciale in quanto
stabilisce il reading frame per l'intera lunghezza del messaggio. UN errore di uno dei nucleotidi a questo stato causerebbe ogni codone seguente di essere letto male e creando una 
proteina non funzionale ed \`e l'ultimo punto in cui la cellula pu\`o decidere se l'mRNA verr\`a tradotto. Il tasso di questo passaggio \`e determinante del tasso in cui una proteina 
viene sintetizzata. La traduzione inizia con il codone AUG e un tRNA iniziatore che trasporta l'amminoacido metionina, in ogni terminazione \ce{N-} della proteina che viene rimosso
da una proteasi specifica. Questo tRNA \`e riconosciuto da fattori di iniziazione in quanto ha una sequenza nucleotidica distinda da quello che usualmente trasporta la metionina. Il
complesso iniziatore tRNA-metionina \`e prima caricato nella piccola subunit\`a ribosomiale con i fattori eucariotici di iniziazione o eIF. Di tutti gli amminoacil-tRNA nella cellula 
solo l'iniziatore \`e in grado di legare la piccola subunit\`a ribosomiale senza che sia presente il ribosoma completo e si lega direttamente al sito P. Successivamente la piccola 
subunit\`a ribosomiale si lega alla terminazione $5'$ di un mRNA, riconosciuta grazie al cappuccio e poi si muove in avanti alla ricerca del primo AUG. Altri fattori di iniziazione 
agiscono come elicasi energizzate dall'ATP facilitandone il movimento. Quando trova AUG i fattori di iniziazione si disassociano permettendo alla subunit\`a di associarsi con il
complesso e completo ribosoma. Il tRNA iniziatore rimane al sito P, lasciando il sito A vuoto e la sintesi \`e pronta a cominciare. I nucleotidi che circondano il sito di inizio 
influenzano l'efficenza del riconoscimento dell'AUG durante lo scan. Se il sito di riconoscimento \`e diverso sostanzialmente dalle regioni adiacenti lo scan pu\`o avolte ignorare l'AUG
e passare ad un altro. Questo fenomeno detto leaky scanning viene utilizzato per produrre la stessa proteina con e senza una sequenza di sengale attaccata. Il meccanismo per selezionare 
il codone di inizio nei batteri \`e diverso: ogni mRNA contiene uno specifico sito di legame al ribosoma locato pochi nucleotidi prima dell'AUG dove inizia la traduzione. I ribosomi 
batterici possono assemblare direttamente su un codone di inizio interno all'mRNA che pertanto pu\`o essere utilizzato per codifiare pi\`u proteine e si dice policistronico. 
\subsection{I codoni di fine marcano la fine della traduzione}
La fine del messaggio di codifica \`e segnalata dalla presenza di uno dei tre codoni di fine UUA, UAG o UGA che non sono riconosciuti da un tRNA e non specificano un amminacido, ma 
segnalano al ribosoma di terminare la traduzione. I fattori di rilascio legano ogni ribosoma con un codone di fine posizionato al sito A, forzando la transferasi peptidica a catalizzare
l'addizione di una molecola di acuqa invece di un amminacido al peptidil-tRNA. Questa reazione libera la fine carbossilica della catena polipeptidica dalla molecola di tRNA e il suo
rilascio nel citoplasma. Il ribosoma poi rilascia la molecola di mRNA legata e si separa nelle due subunit\`a che si possono poi assemblare su un'altra molecola di mRNA. Durante la
traduzione il polipeptide nascente si muove attraverso un tunnel pieno d'acqua nella subunit\`a grande del ribosoma. Le pareti del tunnel composte dai 23S rRNA sono un insieme di 
piccole superfici idrofobiche incastrate in una superficie idrofilica pi\`u estensiva. Questa struttura non \`e complementare a nessun peptide  e mette a disposizione un incapsulamento
in cui il polipeptide pu\`o sciblare. Le proteine sono senza struttura mentre passano attraverso il ribosoma, anche se si possono creare delle regioni a $\alpha$ elica. Mentre lascia il
ribosoma la proteina deve piegarsi nella conformazione utile alla cellula. 
\subsection{Le proteine sono create su poliribosomi}
La sintesi della maggior parte delle proteine avviene tra i $20$ secondi e molti minuti. Durante questo periodo avvengono molte iniziazioni su ognuno dei mRNA tradotti. Appena il 
ribosoma precedente ha tradotto abbastanza sequenza nucleotidica per muoversi la terminazione $5'$ viene messa in un nuovo ribosoma. Le molecole di mRNA che sono tradotte si trovano 
nella forma di poliribosomi: assemblaggi citoplasmatici grandi composti da molti ribosomi vicini al massimo $80$ nucelotidi lungo un singolo mRNA. 
\subsection{Ci sono piccole variazioni nel codice genetico standard}
Il codice genetico possiede delle rare eccezioni: nei mitocondri dei mammiferi AUA \`e tradotta in metionina, mentre nel citosol della cellula come isoleucina. Questo tipo di deviazione
\`e specifica nell'organismo o nell'organello in cui avviente. In molte cellule avviene una recodifica di traduzione in cui altra informazione di sequenza nucleotidica presente 
nell'mRNA pu\`o cambiare il significato del codice genetico ad un sito particolare dell'mRNA in quanto esite un ventunesimo amminoacido, la selenocisteina, essenziale per molte funzioni
enzimatiche che contiene un atomo di selenio al posto dello zolfo della cisteina. \`E prodotta enzimaticamente da una serina attaccata a una molecola speciale di tRNA che si accoppia 
con il codone UGA. L'mRNA per le proteine che la contengono possiedono una sequenz anucleotida vicina che causa l'evendo di ricodifica. 
\subsection{Meccanismi di controllo della qualit\`a agiscono per prevenire la traduzione di mRNA danneggiati}
Gli mRNA possono lasciare il nucleo danneggiati o danneggiarsi nel processo. Sono presenti molti meccanismi per impedire la loro traduzione come il riconoscimento del cappuccio $5'$ e 
la catena poli-A prima della traduzione. Il meccanismo pi\`u potente \`e il decadimento di mRNA mediato da nonsense che elimina gli mRNA difettivi prima che si sposino dal nucleo. Questo
viene chiamato quando la cellula determina che una molecola di mRNA possiede un codone di terminazione nonsense nel posto errato. Avviene principalente in una molecola con uno splice
errato. Questo meccanismo inizia quanod una molecola di mRNA viene trasportat nel citosol. Mentre la terminazione $5'$ emerge dal poro l'mRNA si incontra con un ribosoma che comincia 
la traduzione. Mentre questa procede il complesso di giunzione degli esoni legati all'mRNA a ogni sito di splice sono mossi dal ribosoma. Il codone di fine normale si trova dentro 
l'ultimo esone e quando il ribosoma rriva ad esso e si ferma nessun EJC sar\`a legato all'mRNA. In questo caso l'mRNA passa l'ispezione e pu\o essere tradotto. Altrimenti la molecola
viene degradata. Il primo passo di traduzione permette alla cellula di verificare la capacit\` a della molecola di mRNA di produrre la proteina corretta mentre esce il nucleo. 
\subsection{ALcune proteine cominciano a piegarsi mentre vengono sintetizzate}
La catena polipeptidica per essere utile alla cellula deve piegarsi nella conformazione, legare ogni cofattore necessario alla sua attivit\`a, essere modificata dalla chinasi o altri 
enzimi e assemblarsi con altre subunit\`a proteiche con cui funziona. Queste informazioni sono contenute nella sequenza di amminoacidi. Quando una proteina si piega la parte idrofobica
si trova in un nucleo interno e un gran numero di interazioni non covalenti si formano tra varie parti della molecla. L'insieme di queste interazioni energeticamente favorevoli determina
la conformaizone finale della catena. Per alcune proteine il ripiegamento inizia appena la catena esce dal ribosoma cominciando dalla terminazione \ce{N-}. In questi casi si forma in 
pochi secondi una struttura compatta che contiene la maggior parte delle strutture secondarie. Per alcuni domini proteici si crea uno stato flessibile detto il molten globule, il 
processo di inizio verso l'arrivo alla conformazione corretta. 
\subsection{Accompagnatori molecolari aiutano a guidare il piegamento della maggior parte delle proteine}
La maggior parte delle proteine non si ripiegano correttamente durante la loro sintesi e richiedono proteine dette accompagnatori molecolari, utili in quanto stabiliscono il cammino
di piegamento che la proteina deve compiere. Riconoscono specificatamente configurazioni inforrette dall'esposizione delle superfici idrofobiche causato da legami reciproci. Gli 
accompagnatori evitano questo legandosi con tali superfici.
\subsection{La cellula utilizza diversi tipi di accompagnatori}
Molti accompagnatori sono detti proteine a shock terminco (hsp) in quanto sono sintetizzati in quantit\`a maggiore dopo brevi esposizioni della cellula a temperature elevate che
riflette all'operazione di un sistema di feedbak che risponde a un aumento in proteine malformate aumentando la sintesi degli accompagnatori che aiutano a ripiegarsi. Ci sono diverse
famiglie di accombagnatori e diversi membri funzionano in diversi organelli. Le proteine hsp60 e hsp70 lavorano con il loro piccolo insieme di proteine associate, hanno un affinit\`a
per superfici idrofobiche sposte e idrolizzano l'ATP legandosi e rilasciando il substrato proteico ad ogni ciclo di idrolisi. hsp70 lavora con cellule appena sintetizzate con ogni
suoi monomero legandosi a una stringa di quattro o cinque amminoacidi idrofobici. Sull'ATP legante rilascia la proteina da una struttura a basrile che agisce dopo che la proteina si \`e
completamente sintetizzata. Detto chaperonina forma una camera di isolamento per il processo di piegamento. Per entrare la camera la proteina substrato deve essere catturata attraverso
l'entrata idrofomibca e viene rilasciata nella camera piena di superfici idrofiliche e la camera \`e chiusa attraverso idrolisi dell'ATP. Qui il substrato si piega nella conformazione
finale in isolamento. Quando l'ATP viene idrolizzato il coperchio della camera si separae la proteina substrato esce dalla camera. L'energia dell'idrolisi viene utilizzta per 
movimenti meccanici che convertono gli accompagnatori dalla forma di cattura a quella di rilascio. 
\subsection{Regioni idrofobiche esposte forniscono segnali critici per il controlli di qualit\`a delle proteine}
L'azione della proteina hsp70 inizia quando una proteina sta venendo ancora sintetizzata. La cellula riconosce le proteine piegate male che richiedono turni addizionali di ripiegamento
catalizzati dall'ATP. Se una proteina ha una superficie di amminoacidi idrofobixi \`e anormale e pu\`o essere pericolosa per la cellula. Le proteine che si piegano velocemente da sole
non mostrano questi pattern e possono evitare gli accompagnatori, mentre le altre sono riparate da essi. Quando questo non funziona un meccanismo la distrugge completamente. Qusto 
cammino comincia con il riconoscimento di una superficie idrofobica anormale e finisce con la consegna della proteina a un complesso di proteasi detto proteosoma che la distrugge. 
\subsection{Il proteosoma \`e una proteasi compartimentalizzata con siti attivi reclusi}
Il macchinario proteolitico e gli accompagnatori competono per il riconoscimento di una proteina piegata male. Se si ripiega rapidamente solo una piccola frazione viene degradata. 
L'apparato che distrugge le proteine errate \`e il proteosoma, una proteasi dipendente dall'ATP. \`E presente in molte copie disperse nel citosol e nel nucleo e distrugge anche le 
proteine che sono entrate nel reticolo endoplasmatico. In queto caso sono riconosciute da un sistema di sorveglianza e le retrotrasloca nel citosol per la degradazione da parte del
proteosoma. Ogni proteosoma consiste di un cilindro gentrale vuoto formato da multiple subunit\`a proteiche che si assemblano come uno stack di quattro anelli eptamerici. Alcune delle
subunit\`a sono proteasi i cui siti attivi si trovano all'interno della camera impedendo la loro azione incontrollata. Ogni fine del cilindro \`e associata a un complesso proteico che
contiene sei anelli attraverso i quali le proteine obiettivo sono portate verso il nucleo dove sono degradate. La reazione di importaizone \`e guidata dall'idrolisi dell'ATP e svolge
la proteina obiettivo mentre si muove lungo il cappuccio esonendola alla proteasi nel nucleo. Le proteine appartengono alle unfoldases o proteina AAA che funzionano come esameri
e hanno caratteristiche comuni alla DNA elicasi. Una propriet\`a cruciale del proteosoma \` la processivit\`a del meccanismo: il substrato rimane legato fino a che non \`e completamente
convertito in peptidi corci. I 19S cappucci del proteosoma agiscono come cancelli all'entrata del nucleo proteolitico interno e solo le proteine marcate per la distruzione possono
passarci altradetto. La marcatura \`e il legame con l'ubiquitina, una piccola proteina che in questo caso legate in una catena alla lisina $48$. Un insieme speciale di molecole E3 \`e
responsabile per l'ubiquitilazione di proteine denaturate, mal piegate o contenenti amminoacidi anormali o ossidati attraverso una superficie idrofobica esposta che agisce come segnale
per queste molecole. In ogni caso si deve distinguere tra proteine completate con la conformazione errata e proteine che si stanno formando.
\subsection{Molte proteine sono controllate da distruzione regolata}
Un'altra funzione di questi cammini proteolitici \`e di conferire corte vite a proteine spcecifiche la cui concentrazione deve cambiare rapidamente con stati alterati della cellula. 
Alcune di queste sono degradate velocemente sempre, alte solo in certe condizinoi. Questo viene controllato in una classe di meccanismi dall'attivit\`a di un ubiquitina ligasi che viene
accesa da fosforilazione dell'E3 o da una transizione allosterica dell'E3 causata dal suo legame con una moleocla. Il complesso di promozione dell'anafase (APC) \`e una ligasi a pi\`u
subunit\`a che \`e attivata da addizione di subunit\`a durante la mitosi temporizzato dalla cellula. In risposta ad altri segnali si pu\`o creare un segnale di degradazione nella 
proteina causando una rapida ubiquitazione. Un modo comune \`e la fosforilazione di un  sito specifico che mostra un segnale di degradazione o una disassociazione di una subunit\`a. 
Segnali di degradazione potenti possono essere creati rompendo un legame peptidico se questo crea una nuova terminazione \ce{N-} riconosciuta da una specifica proteina E3 come un
residuo destabilizzante. 

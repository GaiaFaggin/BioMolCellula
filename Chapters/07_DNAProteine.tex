\chapter{Come una cellula legge il genoma, dal DNA alle proteine}
Il DNA nel genoma usa l'RNA come intermediario nella sintesi delle proteine. Quando una cellula necessita una proteina utilizza la sequenza appropriata della catena nucleotidica 
copiandola in RNA durante la trascrizione che direziona direttamente la sintesi della proteina durante la traduzione. Esistono varianti di questo processo in cui i trascritti a RNA 
vengono processati nel nucleo con processi come RNA splicing prima che posano uscire da esso. Questi cambi possono cambiare il significato di una molecola di DNA. Per molti geni inoltre
il prodotto finale \`e RNA. I genomi di organismi multicellulari sono disordinati con corti esoni e lunghi introni. Sezioni che codficano il DNA sono separate da lunghe sequenze senza 
apparente significato. 
\section{Dal DNA all'RNA}
Essendo che molte copie identiche dello stesso RNA possono essere completate dallo stesso gene ogni molecola di RNA pu\`o guidare la sintesi di molte proteine identiche, ma i geni sono
trascritti e tradotti a tassi diversi, permettendo la cellula di avere vaste quantit\`a di alcune proteine e piccole di altre. Inoltre la cellula regola l'espressione di ognuno dei suoi
geni secondo i suoi bisogni, controllando la produzione del suo RNA>
\subsection{Le molecole di RNA hanno un unico filamento}
Il primo passo nella lettura delle istruzioni geniche \`e la copia di una particolare sequenza della sequenza di nucleotidi in una a RNA. L'infromazione nell'RNA \`e scritta nello stesso
linguaggio del DNA e questo processo \`e pertanto detto trascrizione. L'RNA \`e un polimero lineare composto da quattro tipi di subunit\`a nucleotidiche legate da legami a fosfodiestere.
Differisce dal DNA in quanto i nucleotidi nell'RNA sono ribonucleici, ovvero contengono ribosio e contiene la base uracile invece della timina che si pu\`o legare all'adenina. La
struttura complessiva \`e molto diversa: l'RNA \`e a filamento singolo e una catena pu\`o piegarsi in una forma simile a una proteina permettendogli di avere precise funzioni strutturali
e catalitiche.
\subsection{La trascrizione produce RNA complementare a un filamento di DNA}
L'RNA \`e sintetizzato attraverso la trascrizione del DNA, che comincia con l'apertura e lo svolgimento di una piccola porzione della doppia elica che espone le basi sui filamenti, uno
dei quali agisce come stampo per la sintesi della molecola di RNA. La sequenza di nucleotidi \`e determinata dall'accoppiamento di basi complementari tra i nucleotidi che arrivano e lo
stampo. Quando avviene una corrispondenza il ribonucleide che arriva \`e legato covalentemente con la catena crescente in una reazione catalizzata da enzimi. La catena \`e allungata un
nucleotide alla volta e possiede una sequenza complementare allo stampo. Il filamento di RNA non rimane legato con lo stampo ma \`e separato dietro al regione dove i nucleotidi sono
aggiunti causando il rilasciamento come singolo filamento. Le molecole di RNA sono inoltre molto pi\`u corte rispetto le molecole di DNA. 
\subsection{L'RNA polimerasi causa la trascrizione}
Gli enzimi che svolgono la trascrizione sono detti RNA polimerasi e catalizzano la formazione del legame fosfodiestere che lega i nucleotidi muovendosi lungo il DNA, svolgendo l'elica
sopra il sito attivo per la polimerizzazione. La catena di RNA \`e estesa nella direzione $5'$-$3'$. I substrati sono ribonicleoside trifosfato la cui idrolizzazione fornisce l'energia
necessaria alla reazione. Il rilascio immediato dell'RNA significa che le copie possono essere create in poco tempo, con la sintesi di molecole addizionali che inizia prima che quelle
prime sian completate. L'RNA polimerasi catalizza l'unione di ribonucledi e pu\`o cominciare una catena di RNA senza un primer. L'RNA polimerasi fa un errore una volta ogni $10^4$ 
nucleotidi e le conseguenze di tale errore sono meno significative. Inoltre la stessa RNA polimerasi che comincia una molecola di RNA eve finirla senza dissociarsi dallo stampo. L'RNA
polimerasi contiene un meccanismo di proofreading: se un ribonucleotide \`e aggiungo la polimerasi pu\`o indietreggiare e il sito attivo svolge una reazione di asportaizone dove una
molecola d'acqua sostituisce il pirofostato ed \`e rilasciata una molecola di monofosfato. 
\subsection{Le cellule producono diverse categorie di molecole di RNA}
La maggior parte dei geni trasportati in un DNA della cellula specificano la sequenza di amminoacidi della proteina e le molecole di RNA che sono copiate da questi geni sono detti RNA
messaggeri o mRNA. Il prodotto finale di altri geni \`e la molecola di RNA, detti RNA non codificanti che servono come componenti strutturali, enzimatiche e regolatore per molti 
processi. Molecole di RNA piccolo nucleare o snRNA direzionano lo splicing di pre-mRNA per formare mRNA, l'RNA ribosomiale o rRNA forma il nucleo del ribosoma e i tranfer RNA o tRNA
forma gli adattatori che selezionano gli amminoacidi e li mantengono in posizione. I microRNA o miRNA e RNA piccolo interferente siRNA servono come regolatori per l'espressione genica e 
RNA piwi-interagente o piRNA protegge le linee germinali dai trasposoni. I long noncoding RNA o lncRNA con funzione di impalcature e regolano diversi procesi cellulari come
l'inattivazione del cromosoma X. Ogni segmento di DNA trascritto \`e detto unit\`a di trascrizione che tipicamente possiede le informazioni di un gene. La maggior parte dell'RNA nella
cellula \`e rRNA/
\subsection{Segnali codificati nel DNA indicano l'RNA polimerasi dove iniziare e dove finire}
Per trascrivere un gene accuratamente la RNA polimerasi deve riconoscere dove iniziare e finire sul genoma. QUesto avviene in maniera diversa rispetto a batteri ed eucarioti. 
L'iniziazione di una trascrizione \`e il punto in cui la cellula regola quali proteine devono essere prodotto e a quale velocit\`a. L'RNA polimerasi batterica \`e un complesso a 
multisubunit\`a che sintetizza l'RNA. Una subunit\`a addizionale detta fattore sigma $\sigma$ associa con l'enzima nucleo e lo assiste nella lettura dei segnali nel DNA che dicono dove
iniziare la trascrizione

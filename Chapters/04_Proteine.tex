\chapter{Proteine}
Le proteins costituiscono la maggior parte della massa secca della cellula. Svolgono le principali funzioni strutturali e la maggior parte delle funzioni
della cellula: gli enzimi mettono a disposizione le superfici che catalizzano le reazioni chimiche. Quelle incororate nella membrana plasmatica formano 
canali e pompe per il controllo del passaggio di piccole molecole nella cellula, hanno funzione di messaggeri intra e inter cellulari, si occupano del
movimento della cellula, sono anticorpi, tossine, ormoni e miuulle altre funzioni. 
\section{Forma e struttura delle proteine}
\subsection{La forma di una proteina \`e specificata dalla sua sequenza degli amminoacidi}
Esistono $20$ amminoacidi diversi che vengono codificati dal DNA di un organismo, ognuno con propriet\`a chimiche diverse. Una molecola protena \`e
costituista da una lunga catena senza ramificazioni di questi amminoacidi, ognuno legato ai suoi vicini attraverso un legame peptidico covalente. Sono
pertanto dette polipeptidi. Ogni tipo di proteina possiede una sequenza di amminoacidi unica e ne esistono migiaia di tipi. La sequenza ripetuti di atomi 
lungo il nucleo della catena polipeptidica \`e detto il polypeptide backbone, al quale si attaccano quelle porzioni di amminoacidi non coinvolte nel 
legame peptidico e che conferiscono all'amminoacido le sue propriet\`a uniche. Le $20$ catene laterali differenti degli amminoacidi differiscono nelle
propriet\`a: alcune sono nonpolari e idrofobiche, altre negativamente o positivamente cariche, altre formano legami covalenti. Le proteine formano una 
catena flessibile che pu\`o ripiegarsi in infiniti modi. Tale struttura pu\`o essere determianta da molti legami non covalenti che si formano tra una 
parte della catena e l'altra e sono: legami idrogeno, attrazioni elettrostatiche e forze di Van der Waals che in parallelo possono mantenere insieme due 
regioni della catena. La forza di questo gran numero di legami non covalenti determina la sabilit\`a di ogni forma ripiegata. Pu\`o entrare anche in gioco
una forza apparente di attrazione idrofobica. La forma viene pertanto influenzata fortemente dalla distribuzione degli amminoacidi polari e non polari. I 
non polaritendono a trovarsi all'interno della molecola in modo da evitare il contatto con l'acqua, mentre i gruppo polari verso l'esterno. Amminoacidi
polari all'interno della proteina sono tipicamente legati con altri amminoacidi polari o al backbone polipeptidico attraverso legami a idrogeno.
\subsection{Le proteine si ripiegano nella conformazione a minore energia}
La maggior parte delle proteine possiedono una particolare struttura tridimensionale. Tale conformazione \`e quella che minimizza la sua energia libera. 
La sequenza di amminoacidi contiene tutte le informazioni necessarie alla struttura della proteina. La maggir parte si piegano in una singola 
conformazione stabile, che pu\`o variare leggermente quando interagiscono con altre molecole. Nelle cellule una proteina detta molecular chaperone assiste
il processo di piegatura legandosi a catene polipeptidiche parizalmente piegate e aiutandole verso il cammino pi\`u favorevole. Sono richiesti per 
prevenire la formazione di aggregati proteici attraverso regioni idrofobiche temporaneamente esposte. Rendono pertanto il processo pi\`u affidabile. La
maggior parte delle proteine si trovano ad una lunghezza da $50$ a $2000$ amminoacidi. Proteine grandi consistono di domini proteici, unit\`a strutturali 
che si piegano indipendentemente. 
\subsection{L'$\alpha$-elica e il $\beta$-foglietto sono pattern di piegamento comuni}
Analizzando la struttura tridimensionale delle proteine diventa chiaro come esistano due pattern di piegamento regolari. L'$\alpha$-elica e il 
$\beta$-foglietto e sono causati dal legame a idrogeno tra i gruppi \ce{N-H} e \ce{C=O} nel backbone polipeptidico, senza coinvolgere elementi della 
catene secondarie degli amminoacidi. Pertanto, nonostante siano incompatibili rispetto ad alcuni amminoacidi molte sequenze le possono formare. In 
entrambi i casi la proteina adotta una conformazione regolare e ripetuta. La parte centrale di molte proteine contiene regioni estese di 
$\beta$-foglietti che possono fermare segmenti vicini di backbone polipeptidico con la stessa orientazione (catene parallele) o che si piega su s\`e 
stessa (catene antiparallele). Un $\alpha$-elica viene generata quanto una singola catena polipeptidica si torce su s\`e stessa per formare un cilindro
rigido. Un legame a idrogeno si forma ogni quarto legame peptidico, legando il \ce{C=O} di un peptide con il \ce{N-H} di un altro. L'elica pertanto
compie un giro completo ogni $3.6$ amminoacidi. Questa conformazione \`e abbondante nelle proteine di membrana come quelle di trasporto e i recettori, 
specialmente nella parte che attraversa la stessa e composta da amminoacidi non polari. In altre proteine le $\alpha$-eliche si avvolgono su s\`e stesse
formando una bobina arrotolata, che si forma quando due o quattro $\alpha$-eliche hanno la maggior parte delle catene laterali non polari da una parte, in
modo che possano avvolgersi tra di loro. 
\subsection{I domini proteici sono unit\`a modulari da cui le proteine pi\`u grandi sono costruite}
Vengono distinti quattro livelli di organizzazione nella struttura di una proteina: la struttura primaria \`e la sequenza di amminoacidi, lunghezze di
catena polipeptidicha che formano $\alpha$-eliche e $\beta$-foglietti sono la struttura secondaria, la completa organizzazione tridimensionale viene
detta struttura terziaria e se la proteina \`e costituita da un complesso di pi\`u catene polipeptidiche la strtuttura \`e detta quaternaria. Si intende
per dominio proteico una sottostruttura prodotta da una quasiasi parte contigua di catena polipeptidicha che pu\`o piegarsi indipententemente rispetto 
alle altre. Contengono solitamente tra i $40$ e i $350$ amminoacidi e sono le unit\`a modulari che costituiscono proteine pi\`u grandi. Diversi domini di
una proteina sono solitamente associati con diverse funzioni. Le proteine pi\`u piccole contengono un singolo dominio, mentre le pi\`u grandi anche a
dozzine, spesso connessi da corte e non strutturate lunghezze di catena polipeptidicha che formano cardini flessibili tra i domini.
\subsection{Poche delle catene polipeptidiche possibili saranno utili alla cellula} 
Le possibili combinazioni di una catena polipeptidica di lunghezza $n$ sono $20^n$ e pertanto solo una piccola frazione di questo insieme crea una conformazione tridimensionale stabile, 
circa una in un miliardo. La selezione naturale ha portato la cellula ha scegliere quelle proteine che, oltre a possedere tale conformazione, possiedono propriet\`a chimiche finemente
regolate in modo da permettere alla proteina di catalizzare una particolare reazione o per svolgere la funzione strutturale richiesta. 
\subsection{Le proteine possono essere classificate in molte famiglie}
Una volta che la proteina si \`e voluta per formare una conformazione stabile con un'utilit\`a pu\`o essere modificata attraverso meccanismi genetici in modo da creare nuove proteine
con diverse funzioni. Questo processo porta alla nascita di famiglie di proteine con sequenza e conformazione simili ma con funzione distinte. La struttura di diversi membri di una 
famiglia di proteine \`e conservata maggiormente rispetto alla sequenza di amminaocidi. Molti cambi di amminoacidi sono neutri, senza effetto sulla struttura e funzione della proteina.
Le proteine che subiscono cambi maligni vengono scartate durante il processo evolutivo. 
\subsection{Alcuni domini proteici si trovano in molte proteine diverse}
Le proteine formate da multipli domini si sono formate dall'unione accidentale di sequenze di DNA che codificano ogni dominio. Nel processo evolutivo del mescolamento del dominio, 
molte larghe proteine si sono evolute attraverso l'unione di domini preesistenti in nuove combinazione. Come risultato si sono create nuove superfici leganti alla giustapposizione dei
domini, dove si trovano molti dei siti funzionali della proteina. Un sottoinsieme di domini \`e stato molto mobile durante l'evoluzione, con strutture versatili dette moduli proteici. 
Alcuni domini possiedono un nucleo stabile formato da $\beta$-foglietti con anelli sporgenti di catena polipeptidica. GLi anelli sono situati per formare siti di legame per altre 
molecole. Il loro successo evolutivo \`e dovuto al fatto che mettono a disposizione una base per la generazione di siti di legame, richiedendo unicamente picoli cambi agli anelli 
esterni. Si possono inoltre integrare facilmente all'interno di altre proteine in quanto possiedono alle terminazioni \ce{N-} e \ce{C-}. Quando il DNA codifica tale dominio svolge una
duplicazione a tandem. I domini duplicati con questo ordinamento in linea possono essere collegati per formare strutture estese con s\`e stessi o altri domini. Queste strutture estese
rigide sono comuni in matrici di molecole extracellulari e nella porzione extracellulare di proteine recettrici sulla superficie della cellula. Altri tipi di domini sono detti plug-in
con i legami \ce{N-} e \ce{C-} vicini. Dopo il riordinamento genetico sono messi come inserimenti in una regione ad anello di una sequenza proteina. La frequenza di utilizzo dei domini
differisce tra tipi di organismi. Il complesso maggiore di istocompatibilit\`a (MHC) possiede un dominio di riconoscimento degli antigeni \`e presente unicamente negli umani, con 
funzioni specializzate e sono stati selezionati fortemente durante evoluzioni recenti. Molte coppie di domini si trovano insieme in molte proteine: la maggior parte delle proteine che
contengono coppie di due domini si sono sviluppate relativamente tardi durante l'evoluzione.
\subsection{Il genoma umano codifica un complesso insieme di proteine, rivelando che molto rimane sconosciuto}
La sequenziazione del genoma umano ha rivelato che contiene circa $21000$ geni che codificano proteine e che i vertebrati hanno ereditato la maggior parte delle proteine dagli 
invertebrati, nonostante in media ogni proteina sia pi\`u complessa. Il mescolamento dei domini durante l'evoluzione ha portato alla creazoine di molte nuove combinazioni di domini. La
maggior variet\`a delle proteine permette pi\`u possibli interazioni proteina-proteina.
\subsection{Proteine pi\`u grandi contengono pi\`u di una catena polipeptidica}
Gli stessi legami non covalenti che permettono il piegamento della proteina le permettono di legarsi con altre proteine per formare strutture pi\`u grandi nella cellula. Ogni regione di
una superficie di una proteina che pu\`o interagire con altre molecole si dice sito di legame. Una proteina ne pu\`o contenere diversi. Se tale risto riconosce la superficie di una 
seconda proteina il legame tra le due catene polipeptidiche crea una proteina pi\`u larga con una geometria definita. Ogni catena polipeptidica in tale proteina \`e detta subunit\`a 
proteicha. Nel caso pi\`u semplice due catene polipeptidiche con la stessa conformaizone possono legarsi testa-a-testa, formando un complesso simmetrico di du subunit\`a mantenuto 
dall'interazione tra due siti di legame identici. Molte proteine contengono due o pi\`u tipi di catene polipeptidiche.
\subsection{Alcune proteine globulari formano lunghi filamenti elicoidali}
Una proteina si dice globulare se la catena polipeptidica si ripiega su s\`e stessa formando una forma compatta simile ad una palla con una superficie irregolare. Alcune di queste 
proteine si possono combinare formando lunghi filamenti se ogni molecola possiede un sito di legame complementare ad un'altra regione sulla superficie della tessa molecola. Essendo che
ogni subunit\`a si lega alle altre allo stesso modo e il legame non \`e mai una retta la struttura complessiva assumer\`a una forma ad elica. 
\subsection{Molte proteine hanno forme allungate e fibrose}
Gli enzimi tendono ad essere proteine globulari: nonostante molte sono larghe e complicate con multiple subunit\`a, la maggior parte hanno una forma arrotondata. Ci sono funzioni che
richiedono che ogni molecola proteica occupi una lunga distanza. Queste proteine possiedono generalmente una struttura semplice allungata e sono dette proteine fibrose. Il citoscheletro
\`e formato da forme chiamate filamenti intermedi simili a corde. Sono abbondanti all'esterno della cellula, dove formano la maggior parte della struttura del gel della matrice
extracellulare che aiuta  a legare collezioni di cellule insieme per formare tessuti. Le proteine di matrice sono secrete dalle cellule e si assemblano in fibrilli lunghi.
\subsection{Le proteine contengono una grande quantit\`a di catene polipeptidiche intrinsicamente disordinate}
Un'altra molecola abbondante nella matrice proteica \`e l'elastina, un polipeptide altamente disordinato, il cui disordine \`e fondamentale per la sua funzione di produrre una mesh che
pu\`o essere spostata da una conformazione all'altra. Hanno funzioni importanti nei siti di legame, prendendo una forma solo quando incontrano la molecola che legano. Una funzione 
predominante di queste parti \`e per l'appunto formare siti di legami con altre proteine ad alta specificit\`a ma alterati da fosforilazione o defosforilazione o modifiche iniziate da
eventi di segnale. Vengono utilizzate come legame per mantenere due domini proteici in prossimit\`a in modo da permettere al substrato di muoversi tra i siti attivi in un complesso 
multienzima. Creano microregioni con una consistenza simile a gel che limita la diffusione. 
\subsection{Legami incrociati covalenti stabilizzano proteine extracellulari}







\begin{Huge}
	TO DO
\end{Huge}
